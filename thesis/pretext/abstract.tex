\chapter*{\centering Abstract}
Das Ziel dieser Bachelorarbeit ist es, die mobile Bilbearbeitung als Ansatz für die effiziente Aufmaßerfassung im Gerüstbau zu untersuchen.
Hierzu wird eine Software-Lösung in Form einer Android-Applikation für die Aufmaßerfassung im Gerüstbau entwickelt.
Um die Entwicklung dieser App möglichst nah am Anwender durchführen zu können, wird die Forschungsmethode des ``Human-Centered Design Processes'' von Norman (2013) angewandt. \\

Eingangs werden drei Apps aus dem Google Play-Store als mögliche Lösungsalternativen mit Hilfe einer für mobile Geräte erweiterten Form der Nielsen-Heuristiken bzgl. ihrer Usability evaluiert.
Anschließend wird aufbauend auf diesen Evaluationsergebnissen ein eigener Prototyp konzipiert.
Dieser Prototyp wird in drei weiteren Iterationen des ``Human-Centered Design Processes'' verbessert und auftretende Usability-Probleme gelöst.  \\

Ein abschließender Vergleich soll Auskunft darüber geben, ob die mobile Aufmaßerfassung eine Verbesserung in Hinblick auf die Effizienz der Benutzung im Vergleich zum traditionellen, analogen Prozess ist.
