\chapter{Evaluation vorhandener Lösungsalternativen}
Im folgenden Kapitel werden drei verschiedene Lösungsalternativen zur Aufmaßerfassung vorgestellt und bewertet.
Hierzu werden zunächst die Bewertungskriterien vorgestellt und gewichtet.
Anschließend werden die ausgewählten Apps iterativ vorgestellt und mit Hilfe der Kriterien evaluiert.
\todo{apps hier schon benennen?}

\section{Bewertungskriterien}
Zur Evaluation der Lösungsalternativen wird die erweiterte Version der Nielsen-Heuristiken \citep{Nielsen94}, sowie eine ausgewählte Menge selbst aufgestellter Kriterien benutzt.
Da sich die erweiterten Nielsen-Heuristiken aus einer Menge von 18 Kriterien zusammensetzen und nicht jede einzelene für die Bewertung der Lösungsalternativen gleich relevant ist, werden diese jeweils mit einem Bewertungsschema (A, B, C) so gewichtet, dass für die Evaluation der Apps nur für uns relevante Kriterien berücksichtig werden. \\

\subsection{Die erweiterten Nielsen-Heuristiken}\label{subsec:nielsen}
 Nach \citeauthor{Nielsen94} gibt es zehn Heuristiken, die auf einer Grundlage von allgemein anerkannten Prinzipien beruhen, und sich zur Evaluation von Usability-Problemen besonders gut eignen \citep[Seite.~25--62]{Nielsen94}: 

\begin{enumerate}
    \di{Sichtbarkeit des Systems}{N1}{Angemessene Rückmeldung in einem vernünftigen zeitlichen Rahmen}
    \di{Übereinstimmung zwischen System und realer Welt}{N2}{Sprache des Benutzers, natürliche und logische Reihenfolge}
    \di{Benutzerkontrolle und -freiheit}{N3}{Undo und Redo}
    \di{Konsistenz und Standards}{N4}{Konsistente Nutzung und Beachtung der Plattform-Konventionen}
    \di{Fehlervorbeugung}{N5}{Vermeide Situationen in denen Fehler entstehen können}
    \di{Wiedererkennung statt Erinnern}{N6}{Objekte, Optionen und Aktionen sollen sichtbar sein}
    \di{Flexibilität und Effizienz der Benutzung}{N7}{Häufige Aktionen anpassbar, Experten schnellere Bedienung erlauben}
    \di{Ästethisches und minimalistisches Design}{N8}{Keine irrelevanten Informationen}
    \di{Erkennbarkeit, Diagnose und Erholung von Fehlern}{N9}{Verständliche Fehlermeldungen}
    \di{Hilfe und Dokumentation}{N10}{Leicht zu finden, abgestimmt auf Aufgabe, konkrete Schritte zur Lösung}
\end{enumerate} 

\noindent
Zur Bewertung mobiler Endgeräte reichen diese zehn Bewertungskriterien jedoch nicht vollständig aus, sodass weitere acht Heuristiken, die speziell für mobile Geräte ausgelegt sind, hizugezogen werden \todo{ref auf ue, 239}:

\begin{enumerate}
	  \setcounter{enumi}{10}
    \di{Adäquater Umgang mit Unterbrechnungen}{N11}{Kein Verlust von Informationen bei Unterbrechung der Aktivität}
    \di{Fokussieren der Informationen}{N12}{Hervorheben der wichtigen Informationen, um schnelles Scannen zu erlauben}
    \di{``Joy of Use''}{N13}{Positives Nutzungserlebnis bei Bedienung/Verwendung}
    \di{``Don't lie to the user''}{N14}{Keine falschen oder ungültigen Zahlen, Fakten bzw. Nachrichten}
    \di{Unterstützung verschiedener Bildschirmausrichtungen}{N15}{Keine Verwirrung bei verschiedenen Ausrichtungen des Bildschirms}
    \di{Ergonomische Gestaltung der physischen Interatkion}{N16}{Kein unabsichtliches Auslösen von Funktionen}
    \di{Einfache Eingabe, Bildschirmlesbarkeit und Übersichtlichkeit}{N17}{Einfache Navigations- und Eingabetechniken bei einhändiger Benutzung}
    \di{Stelle Privatheit sicher}{N18}{Vermeide den Verlust/Diebstahl von privaten Daten z.b. durch Passwörter}
\end{enumerate}

\subsection{Weitere Kriterien}
Zusätzlich zu den in \autoref{subsec:nielsen} vorgestellen Heuristiken werden noch zwei weitere Kriterien zur Evaluation hinzugezogen, die für die Benutzung der App aus Sicht der Aufmaßerfassung im Gerüstbau wichtig sind: \todo{anders formulieren}

\begin{itemize}
    \di{Integration der App in eine vorhandene Systemarchitektur}{integration}{Die Software-Lösung sollte sich in eine bereits vorhandene Systemarchitektur integrieren lassen}
    \di{Export des annotierten Bildes und der Metadaten zur Weiterverarbeitung in einem nachgelagerten Dienst (z.B. API)}{export}{Die App sollte die eingetragenen Meta-Daten, wie zum Beispiel die Längen einer Linie, zur Weiterverarbeitung exportierbar machen}
	  \todo{mehr Kriterien raussuchen}
\end{itemize}

\section{Vorstellung und Evalutation ausgewählter Apps}
Als Lösungsalternativen stehen die drei Android-Applikationen \emph{Measuring Master}, \emph{Measures \& Sketch} und \emph{Photo Measure} zur Verfügung.

Bosch über 80 Standorte in Deutschland, und 14.5 Mrd. Euro Jahresumsatz \\
Photo Measures gute Bewertung mit 4/5 Sternen bei 10k-50k Downloads \\
Measure \& Sketch mit ~100k-500k Downloads sehr populär \\

\subsection{Measuring Master von Bosch}
Bosch Vorstellen mit Bilder + Evaluation der wichtigsten Punkte
\subsection{Measure \& Sketch von SameBits}
2 Vorstellen mit Bilder + Evaluation der wichtigsten Punkte
\subsection{Photo Measures von }
3 vorstellen mit Bilder + Evaluation der wichtigsten Punkte

\section{Auswertung der Evaluationsergebnisse}
Fazit und Überleitung zur Konzeption eigener App

