\chapter{Evaluation vorhandener Lösungsalternativen}
Im folgenden Kapitel werden drei verschiedene Lösungsalternativen zur Aufmaßerfassung vorgestellt und bewertet.
Hierzu werden zunächst die Bewertungskriterien vorgestellt und gewichtet.
Anschließend werden die ausgewählten Apps iterativ vorgestellt und mit Hilfe der Kriterien evaluiert.
\todo{apps hier schon benennen?}

\section{Bewertungskriterien}
Zur Evaluation der Lösungsalternativen wird die erweiterte Version der Nielsen-Heuristiken \citep{Nielsen94}, sowie eine ausgewählte Menge selbst aufgestellter Kriterien benutzt.
Da sich die erweiterten Nielsen-Heuristiken aus einer Menge von 18 Kriterien zusammensetzen und nicht jede einzelene für die Bewertung der Lösungsalternativen gleich relevant ist, werden diese jeweils mit einem Bewertungsschema (A, B, C) so gewichtet, dass für die Evaluation der Apps nur für uns relevante Kriterien berücksichtig werden. \\

\subsection{Die erweiterten Nielsen-Heuristiken}\label{subsec:nielsen}
 Nach \citeauthor{Nielsen94} gibt es zehn Heuristiken, die auf einer Grundlage von allgemein anerkannten Prinzipien beruhen, und sich zur Evaluation von Usability-Problemen besonders gut eignen \citep[Seite.~25--62]{Nielsen94}: 

\begin{enumerate}
    \di{Sichtbarkeit des Systems}{N1}{Angemessene Rückmeldung in einem vernünftigen zeitlichen Rahmen}
    \di{Übereinstimmung zwischen System und realer Welt}{N2}{Sprache des Benutzers, natürliche und logische Reihenfolge}
    \di{Benutzerkontrolle und -freiheit}{N3}{Undo und Redo}
    \di{Konsistenz und Standards}{N4}{Konsistente Nutzung und Beachtung der Plattform-Konventionen}
    \di{Fehlervorbeugung}{N5}{Vermeide Situationen in denen Fehler entstehen können}
    \di{Wiedererkennung statt Erinnern}{N6}{Objekte, Optionen und Aktionen sollen sichtbar sein}
    \di{Flexibilität und Effizienz der Benutzung}{N7}{Häufige Aktionen anpassbar, Experten schnellere Bedienung erlauben}
    \di{Ästethisches und minimalistisches Design}{N8}{Keine irrelevanten Informationen}
    \di{Erkennbarkeit, Diagnose und Erholung von Fehlern}{N9}{Verständliche Fehlermeldungen}
    \di{Hilfe und Dokumentation}{N10}{Leicht zu finden, abgestimmt auf Aufgabe, konkrete Schritte zur Lösung}
\end{enumerate} 

\noindent
Zur Bewertung mobiler Endgeräte reichen diese zehn Bewertungskriterien jedoch nicht vollständig aus, sodass weitere acht Heuristiken, die speziell für mobile Geräte ausgelegt sind, hizugezogen werden \todo{ref auf ue, 239}:

\begin{enumerate}
	  \setcounter{enumi}{10}
    \di{Adäquater Umgang mit Unterbrechnungen}{N11}{Kein Verlust von Informationen bei Unterbrechung der Aktivität}
    \di{Fokussieren der Informationen}{N12}{Hervorheben der wichtigen Informationen, um schnelles Scannen zu erlauben}
    \di{``Joy of Use''}{N13}{Positives Nutzungserlebnis bei Bedienung/Verwendung}
    \di{``Don't lie to the user''}{N14}{Keine falschen oder ungültigen Zahlen, Fakten bzw. Nachrichten}
    \di{Unterstützung verschiedener Bildschirmausrichtungen}{N15}{Keine Verwirrung bei verschiedenen Ausrichtungen des Bildschirms}
    \di{Ergonomische Gestaltung der physischen Interatkion}{N16}{Kein unabsichtliches Auslösen von Funktionen}
    \di{Einfache Eingabe, Bildschirmlesbarkeit und Übersichtlichkeit}{N17}{Einfache Navigations- und Eingabetechniken bei einhändiger Benutzung}
    \di{Stelle Privatheit sicher}{N18}{Vermeide den Verlust/Diebstahl von privaten Daten z.b. durch Passwörter}
\end{enumerate}

\subsection{Weitere Kriterien}
Zusätzlich zu den in \autoref{subsec:nielsen} vorgestellen Heuristiken werden noch zwei weitere Kriterien zur Evaluation hinzugezogen, die für die Benutzung der App aus Sicht der Aufmaßerfassung im Gerüstbau wichtig sind: \todo{anders formulieren}

\begin{itemize}
    \di{Integration der App in eine vorhandene Systemarchitektur}{integration}{Die Software-Lösung sollte sich in eine bereits vorhandene Systemarchitektur integrieren lassen}
    \di{Export des annotierten Bildes und der Metadaten zur Weiterverarbeitung in einem nachgelagerten Dienst (z.B. API)}{export}{Die App sollte die eingetragenen Meta-Daten, wie zum Beispiel die Längen einer Linie, zur Weiterverarbeitung exportierbar machen}
	  \todo{mehr Kriterien raussuchen}
\end{itemize}

\section{Vorstellung und Evalutation ausgewählter Apps}
Als Lösungsalternativen stehen die drei Android-Applikationen \emph{Measuring Master}, \emph{Measures \& Sketch} und \emph{Photo Measure} zur Verfügung.

Bosch über 80 Standorte in Deutschland, und 14.5 Mrd. Euro Jahresumsatz \\
Photo Measures gute Bewertung mit 4/5 Sternen bei 10k-50k Downloads \\
Measure \& Sketch mit ~100k-500k Downloads sehr populär \\

\subsection{Measuring Master von Bosch}

\subsubsection{Vorstellung}
\todo{Version, Downloaddatum, Playstore-Link}
Die App \emph{Measuring Master} von der Bosch GmbH ist im Play-Store unter der Rubrik ``Effizienz'' aufgelistet.
Selbst beschreibt der App-Hersteller seine Software wie folgt \citep{BoschMM}:

\begin{quote}
  ``Measuring Master ist eine multifunktionale App, die es ermöglicht, Aufmaße, Grundrisse und Temperaturmesswerte an einem Ort zu dokumentieren und zu verwalten.\\
  Die App ist besonders geeignet für Architekten, Maler, Bodenleger, Heizungsbauer und Elektriker, aber auch alle anderen Handwerker profitieren von der umfangreichen Funktionalität''
\end{quote}

\noindent
Nach dem Start der Applikation bietet sich die Möglichkeit ein neues Projekt anzulegen, oder bereits vorhandene Projekte zu bearbeiten.
Sobald das gewünschte Projekt ausgewählt wurde, bieten sich dem Benutzer über das Menü an der linken Seite diverse Funktionen (siehe \autoref{fig:bmenu}) an. \\

\begin{figure}[h]
  \centering
	\begin{subfigure}[b]{0.4\textwidth}
		\includegraphics[keepaspectratio, width=\textwidth]{bosch/menu}
		\caption{Navigationsmenü}
		\label{fig:bmenu}	
	\end{subfigure}
	\begin{subfigure}[b]{0.4\textwidth}
		\includegraphics[keepaspectratio, width=\textwidth]{bosch/bar}
		\caption{Aufmaße-Funktion}
		\label{fig:bbar}	
	\end{subfigure}
  \caption{Measuring Master bei ausgeklapptem Navigationsmenü und in der Aufmaße-Funktion}
\end{figure}

Im Folgenden wird der Menüpunkt ``Aufmaße'' und die damit verbundene Funktionalität weiter vorgestellt und evaluiert.
So bietet sich dem Benutzer nach Auswahl der Aufmaße-Funktion die Möglichkeit, ausgewählte Bilder direkt mit Messwerten zu beschriften.
Hierzu können entweder Bilder direkt aus der Gallerie importiert, oder mit der Kamera aufgenommen werden.
Sobald der Benutzer den Foto-Import erfolgreich abgeschlossen hat, öffnet sich eine neue Ansicht, welche das ausgewählte Bild und weitere Bedienelement, in Form einer Statusleiste am unteren Rand, zeigt (siehe \autoref{fig:bbar}). \\

In dieser Benutzeroberfläche kann der Nutzer mit Hilfe von vier verschiedenen Formen (Linie, Viereck, Winkel, Freiform) Aufmaße ins Bild anzeichnen, und über einen Klick auf die gewünschte Form, Messwerte eintragen.
Außerdem bietet die App die Option, eine ausgewählte Reihe von Laserentfernungsmesser mit der App zu verbinden.
Dies ermöglicht eine Übertraung der gemessenen Distanzen über eine Bluetooth-Schnittstelle direkt an die App. \\

Um das annotierte Bild außerhalb der App weiter zu benutzen, bietet diese den Export als \emph{PDF} und \emph{JPEG} an.
Die exportierte \emph{PDF} enthält im Gegensatz zu der \emph{JPEG} zusätzlich zu dem annotierten Bild auch noch eine Tabelle mit allen eingetragenen Messwerten \autoref{fig:bexport}. 
Zudem lassen sich annotierte Bilder in der App speichern und zu einem späteren Zeitpunkt wieder bearbeiten.

\begin{figure}[h]
  \centering
  \includegraphics[keepaspectratio, width=\textwidth]{bosch/export}
  \caption{Exportierte PDF}
  \label{fig:bexport}
\end{figure}

\subsubsection{Evaluation}

Die App zeigt in einer Art Statusleiste am unteren Rand des Bildschirms den aktuellen Modus an, und gibt über einen auffordernden Text am oberen Bildschirmrand dem Nutzer eine Hilfestellung, was er im gerade ausgewählten Modus machen kann. \todo{screens} Hiermit deckt die App Nielsen \ref{itm:1} und \ref{itm:10} ausreichend ab. \\

Des Weiteren benutzt auch diese App universell verständliche Icons, um die wichtigsten Aktionen wiedererkennbar zu machen. So hat beispielsweise das Mülleimer-Icon in jedem Modus die Löschfunktion. \\

Im Gegensatz zu \textsc{Photo Measures} bietet diese App dem Benutzer die Möglichkeit seine Aktionen rückgängig zu machen, oder sie zu wiederholen. Dies ist ein deutlicher Vorteil seitens der Usability, da Fehler nicht so hart bestraft werden, als wenn keine Undo/Redo-Button vorhanden wären. \\

Fehler werden hier durch das Deaktivieren von Buttons, die im aktuellen Systemzustand nicht benutzbar sind, präventiv verhindert. Das Löschen von Formen ist beispielsweise nur dann möglich, wenn zuvor eine Form ausgewählt wurde.

Negativ fällt auch in dieser Alternative die fehlerhafte Gesten-Unterstützung auf. So sorgen Zoom-Gesten per Doppel-Tap zum unabsichtlichen Zeichnen einer Form, welche im Nachhinein wieder gelöscht werden muss. Außerdem verletzt die App Nielsen~\ref{itm:15}, da Änderungen in der Bildschirmausrichtung dafür sorgen, dass das Bild nicht wie erwartet seine Ursprungsausrichtung beibehält, sonder auch rotiert wird. 

\begin{figure}[h]
	\begin{subfigure}[b]{0.5\textwidth}
		\includegraphics[keepaspectratio, width=0.9\linewidth]{bosch/portrait}
		\caption{App im Portrait-Modus}
		\label{fig:bportait}	
	\end{subfigure}
	~
	\begin{subfigure}[b]{0.5\textwidth}
		\includegraphics[keepaspectratio, height=0.7\linewidth]{bosch/landscape}
		\caption{App im Landscape-Modus}
		\label{fig:blandscape}	
	\end{subfigure}
	\caption{Bildschirmrotation Bosch-App}
	\label{fig:borientation}
\end{figure}

\subsection{Measure \& Sketch von SameBits}
2 Vorstellen mit Bilder + Evaluation der wichtigsten Punkte
\subsection{Photo Measures von }
3 vorstellen mit Bilder + Evaluation der wichtigsten Punkte

\section{Auswertung der Evaluationsergebnisse}
Fazit und Überleitung zur Konzeption eigener App

