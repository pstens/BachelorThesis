\chapter{Evaluation vorhandener Lösungsalternativen}\label{chap:eval}
Im folgenden Kapitel werden drei Lösungsalternativen aus dem Google Play-Store zur Aufmaßerfassung vorgestellt und evaluiert.
Hierzu werden die Bewertungskriterien zunächst vorgestellt, und anschließend iterativ auf die ausgewählten Apps angewandt.
Dieses Kapitel schließt die \emph{Observation} Phase ab, welche in \autoref{chap:problem} begonnen wurde. 

\section{Bewertungskriterien}
Zur Evaluation der Lösungsalternativen wird die erweiterte Version der Nielsen-Heuristiken \citep{Nielsen94}, sowie eine zusätzliche Menge selbst aufgestellter Kriterien benutzt \todo{anders schreiben}.
Da sich die erweiterten Nielsen-Heuristiken aus 18 verschiedenen Kriterien zusammensetzen, von denen nicht jede bezüglich des Einsatzbereiches im Gerüstbau gleich relevant ist, werden die Kriterien nach Absprache mit der Geschäftsleitung in zwei Klassen eingeteilt (R = als relevant beurteil, I = als nicht relevant beurteilt).
Für die Evaluation der Apps werden anschließend nur Kriterien benutzt, die in diesem Kontext als relevant klassifiziert wurden.
\todo{Gewichtung überarbeiten}

\subsection{Die erweiterten Nielsen-Heuristiken}\label{subsec:nielsen}
\citeauthor{Nielsen94} führt in seinem Buch ``Usability inspection methods'' zehn Heuristiken auf, die auf einer Grundlage von allgemein anerkannten Prinzipien beruhen, und sich zur Evaluation von Usability-Problemen besonders gut eignen \citep[Seiten 25--62]{Nielsen94}: 

\begin{enumerate}
    \di{Sichtbarkeit des Systemzustandes (R)}{N1}{Angemessene Rückmeldung in einem vernünftigen zeitlichen Rahmen}
    \di{Übereinstimmung zwischen System und realer Welt (R)}{N2}{Sprache des Benutzers, natürliche und logische Reihenfolge}
    \di{Benutzerkontrolle und -freiheit (R)}{N3}{Undo und Redo}
    \di{Konsistenz und Standards (R)}{N4}{Konsistente Nutzung und Beachtung der Plattform-Konventionen}
    \di{Fehlervorbeugung (R)}{N5}{Vermeide Situationen in denen Fehler entstehen können}
    \di{Wiedererkennung statt Erinnern (R)}{N6}{Objekte, Optionen und Aktionen sollen sichtbar sein}
    \di{Flexibilität und Effizienz der Benutzung (R)}{N7}{Häufige Aktionen anpassbar, Experten schnellere Bedienung erlauben}
    \di{Ästethisches und minimalistisches Design (I)}{N8}{Keine irrelevanten Informationen}
    \di{Erkennbarkeit, Diagnose und Erholung von Fehlern (R)}{N9}{Verständliche Fehlermeldungen}
    \di{Hilfe und Dokumentation (R)}{N10}{Leicht zu finden, abgestimmt auf Aufgabe, konkrete Schritte zur Lösung}
\end{enumerate} 

\noindent
Zur Bewertung von Software auf mobilen Endgeräten reichen diese zehn Bewertungskriterien jedoch nicht vollständig aus, sodass acht weitere Heuristiken, die speziell für mobile Geräte ausgelegt sind, hinzugezogen werden \citep{Bertini06}.
\citeauthor{MachadoNeto13} schlagen hierzu in ihrem Artikel ``Heuristics for the assessment of interfaces of mobile devices'' die folgenden weiteren Kriterien vor:

\begin{enumerate}
    \setcounter{enumi}{10}
    \di{Adäquater Umgang mit Unterbrechnungen (R)}{N11}{Kein Verlust von Informationen bei Unterbrechung der Aktivität}
    \di{Fokussieren der Informationen (R)}{N12}{Hervorheben der wichtigen Informationen, um schnelles Scannen zu erlauben}
    \di{``Joy of Use'' (R)}{N13}{Positives Nutzungserlebnis bei Bedienung/Verwendung}
    \di{``Don't lie to the user'' (I)}{N14}{Keine falschen oder ungültigen Zahlen, Fakten bzw. Nachrichten}
    \di{Unterstützung verschiedener Bildschirmausrichtungen (R)}{N15}{Keine Verwirrung bei verschiedenen Ausrichtungen des Bildschirms}
    \di{Ergonomische Gestaltung der physischen Interatkion (R)}{N16}{Kein unabsichtliches Auslösen von Funktionen}
    \di{Einfache Eingabe, Bildschirmlesbarkeit und Übersichtlichkeit (R)}{N17}{Einfache Navigations- und Eingabetechniken bei einhändiger Benutzung}
    \di{Stelle Privatheit sicher (I)}{N18}{Vermeide den Verlust/Diebstahl von privaten Daten z.b.\ durch Passwörter}
\end{enumerate}

\noindent
Bei der Klassifizierung der Kriterien wurden ``8. Ästethisches und minimalistisches Design'', ``14. Don't lie to the user'' und ``18. Stelle Privatheit sicher'' als nicht relevant gewichtet, da diese drei Punkte nach Absprache mit der Geschäftsleitung bei der Entwicklung einer eigenen Android-App für die Zielgruppe nicht wichtig seien.
\todo{warum wurden gerade diese als I markiert}

\subsection{Weitere Kriterien}
Zusätzlich zu den in \autoref{subsec:nielsen} vorgestellten Heuristiken werden noch zwei weitere Kriterien zur Evaluation hinzugezogen, die für die Einbindung in eine bereits bestehende Systemarchitektur wichtig sind: 

\begin{itemize}
    \di{Integration der App in eine vorhandene Systemarchitektur}{integration}{Die App sollte sich in eine bereits vorhandene Systemarchitektur leicht integrieren lassen}
    \di{Export des annotierten Bildes und der Metadaten zur Weiterverarbeitung durch einen nachgelagerten Dienst (z.B. \emph{API})}{export}{Die App sollte die eingetragenen Meta-Daten, wie zum Beispiel die Längen einer Linie, zur Weiterverarbeitung exportierbar machen}
    \todo{Hier noch mehr Kriterien auflisten}
\end{itemize}
\todo{Punkte beschreiben}

\section{Vorstellung und Evaluation ausgewählter Apps}\label{sec:evaluation}
Als Lösungsalternativen werden im folgenden drei Android-Applikationen vorgestellt, die unter dem Suchbegriff ``Aufmaße'' im Google Play-Store gelistet werden.
Bei der Auswahl der Lösungsalternativen wurden verschiedene Auswahlkriterien berücksichtigt.
So achtet, nach einer Studie aus dem Jahr 2014 von \citeauthor{Dogruel14}, in der 49 Smartphone-Nutzer bei ihrer Entscheidungsfindung zum Download von Apps aus dem Google-Play Store beobachten wurden, ein Großteil (74\%) der Testpersonen auf eine gute Bewertung der App.
33\% der Testpersonen stützen sich laut \citeauthor{Dogruel14} bei ihrer Entscheidungsfindung außerdem auch noch auf Rezensionen über die App.
Für 9\% der Testpersonen sei die Anzahl der Downloads bzw. die Popularität der Applikation ein weiteres Entscheidungskriterium \citep{Dogruel14}. \\

Um die Lösungsalternativen möglichst repräsentativ für das Angebot im Play-Store zu wählen, wurden Apps ausgewählt, die jeweils eine der oben genannten Auswahlkriterien besonders gut erfüllen.
Hierbei handelt es sich um die folgenden drei Apps:

\begin{itemize}
  \item \mm{} - Bosch GmbH (\textit{Popularität})
  \item \im{} - Dirk Farin (\textit{Bewertungen})
  \item \pm{} - Blue Big Pixel Inc. (\textit{Rezensionen})
\end{itemize}

\noindent
Die App \mm{} der Bosch GmbH stammt von einem internationalen Großkonzern mit mehr als 80 Standorten in Deutschland, und einem Jahresumsatz von 14,5 Mrd. Euro \citep{Bosch18}.
Mit zwischen 100 000 bis 500 000 Installationen gehört die App zu einer der Populärsten unter dem Suchbegriff ``Aufmaße'' im Google Play-Store.

\im{} von Dirk Farin wurde ausgewählt, da die App mit einer Gesamtbewertung von 3,9 von 5 Sternen bei 2 764 abgegebenen Bewertungen eine der beliebtesten Apps unter dem Suchbegriff ``Aufmaße'' ist.

Im Gegensatz dazu zeichnet sich die App \pm{} von Blue Bixel Inc. durch eine Vielzahl positiver Rezension aus, welche in der App-Beschreibung im Google Play-Store rezitiert werden.
So bewertet die Review-Seite ``App-Safari'' die App mit den Worten ``So incredibly convenient'' \citep{AppSafari18}, und auf der deutschen Seite ``appgefahren.de'' wird die App weiter gelobt:
``[...] für seinen Zweck eine richtige Empfehlung - die App macht genau das, was sie verspricht.'' \citep{Appgefahren18}.
Das Architektur- und Designmagazin ``Architectural Digest'' bewertet die App als ``very useful when shopping or meeting with contractors'' \citep{Architect18}. \\

Die Bewertungskriterien werden in den nachfolgenden Unterkapiteln jeweils in der Reihenfolge evaluiert, wie sie während des normalen Nutzungsablaufs der entsprechenden App aufgetreten sind.
\subsection{Measuring Master}

\subsubsection{Vorstellung}
\todo{Version, Downloaddatum, Playstore-Link}
Die App \mm{} von der Bosch GmbH ist im Play-Store unter der Rubrik ``Effizienz'' aufgelistet.
Selbst beschreibt der App-Hersteller seine Software wie folgt \citep{BoschMM}:

\begin{quote}
  ``Measuring Master ist eine multifunktionale App, die es ermöglicht, Aufmaße, Grundrisse und Temperaturmesswerte an einem Ort zu dokumentieren und zu verwalten.\\
  Die App ist besonders geeignet für Architekten, Maler, Bodenleger, Heizungsbauer und Elektriker, aber auch alle anderen Handwerker profitieren von der umfangreichen Funktionalität''
\end{quote}

\noindent
Nach dem Start der Applikation bietet sich die Möglichkeit ein neues Projekt anzulegen, oder bereits vorhandene Projekte zu bearbeiten.
Sobald das gewünschte Projekt ausgewählt wurde, bieten sich dem Benutzer über das Menü an der linken Seite diverse Funktionen (siehe \autoref{fig:bmenu}) an. \\

\begin{figure}[h]
  \centering
  \begin{subfigure}[t]{0.4\textwidth}
    \includegraphics[keepaspectratio, width=\textwidth]{bosch/menu}
    \caption{Navigationsmenü}\label{fig:bmenu}	
  \end{subfigure}
  \begin{subfigure}[t]{0.4\textwidth}
    \includegraphics[keepaspectratio, width=\textwidth]{bosch/bar}
    \caption{Aufmaße-Funktion}\label{fig:bbar}
  \end{subfigure}
  \caption{\mm{} bei ausgeklapptem Navigationsmenü und in der Aufmaße-Funktion}
\end{figure}

Im Folgenden wird der Menüpunkt ``Aufmaße'' und die damit verbundene Funktionalität weiter vorgestellt und evaluiert.
So bietet sich dem Benutzer nach Auswahl der Aufmaße-Funktion die Möglichkeit, ausgewählte Bilder direkt mit Messwerten zu beschriften.
Hierzu können entweder Bilder direkt aus der Gallerie importiert, oder mit der Kamera aufgenommen werden.
Sobald der Benutzer den Foto-Import erfolgreich abgeschlossen hat, öffnet sich eine neue Ansicht, welche das ausgewählte Bild und weitere Bedienelement, in Form einer Statusleiste am unteren Rand, zeigt (siehe \autoref{fig:bbar}). \\

In dieser Benutzeroberfläche kann der Nutzer mit Hilfe von vier verschiedenen Formen (Linie, Viereck, Winkel, Freiform) Aufmaße ins Bild anzeichnen, und über einen Klick auf die gewünschte Form, Messwerte eintragen.
Außerdem bietet die App die Option, eine ausgewählte Reihe von Laserentfernungsmesser mit der App zu verbinden.
Dies ermöglicht eine Übertraung der gemessenen Distanzen über eine Bluetooth-Schnittstelle direkt an die App. \\

Um das annotierte Bild außerhalb der App weiter zu benutzen, bietet diese den Export als \emph{PDF} und \emph{JPEG} an.
Die exportierte \emph{PDF} enthält im Gegensatz zu der \emph{JPEG} zusätzlich zu dem annotierten Bild auch noch eine Tabelle mit allen eingetragenen Messwerten \autoref{fig:bexport}. 
Zudem lassen sich annotierte Bilder in der App speichern und zu einem späteren Zeitpunkt wieder bearbeiten.

\begin{figure}[h]
  \centering
  \includegraphics[keepaspectratio, width=\textwidth]{bosch/export}
  \caption{Exportierte PDF}\label{fig:bexport}
\end{figure}

\subsubsection{Evaluation}

Durch die Anzeige der Statusleiste am unteren Bildschirmrand (siehe \autoref{fig:bbar}) gibt die App dem Benutzer zu jeder Zeit eine klare und sichtbare Rückmeldung über den aktuellen Systemzustand (Nielsen~\autoref{itm:N1}).

Zusätzlich bietet ein erklärender Text im oberen Bereich des Bildschirms eine konkrete Hilfestellung zu den möglichen Aktionen, die der Benutzer im aktuellen Systemzustand ausführen kann (Nielsen~\autoref{itm:N10}).

Darüber hinaus kann der Nutzer über ein Undo- bzw. Redo-Symbol fehlerhafte oder unabsichtliche Eingaben rückgängig machen, oder rückgängig gemachte Aktionen wiederholen (Nielsen~\autoref{itm:N3}).

Dadurch, dass Aktionen nur dann ausführbar sind, wenn man den richtigen Modus in der Statusleiste ausgewählt hat, werden Fehler präventiv vermieden. So lassen sich Messwerte erst dann einfügen, wenn zuvor eine Form markiert wurde, oder Aktionen rückgängig machen, wenn zuvor eine Benutzeraktion stattgefunden hat (Nielsen~\autoref{itm:N5}).

Schafft man es trotzdem einen Fehlerzustand des Systems herzustellen, wird dieser durch einen ``Toast'' \todo{def} erkennbar gemacht, und gibt dem Benutzer Feedback darüber, was zum Fehler geführt hat \todo{Bild vom Toast beim Bildimport} (Nielsen~\autoref{itm:N9}).

Durch die konsistente Benutzung bekannter Icons lassen sich die dahinter befindlichen Aktionen intuitiv erkennen, und die Gedächtnislast des Benutzers reduzieren. 
So wird zum Beispiel sowohl im ``Malen-Modus'' als auch im ``Text-Modus'' das Mülleimer-Icon in der oberen Statusleiste als Löschfunktion der markierten Form bzw.\ des ausgewählten Textes benutzt (Nielsen~\autoref{itm:N4} \& \autoref{itm:N6}).

Einen ``Expertenmodus'', der dem Benutzer voreingestelle Aktionen ändern, oder Abkürzungen für bestimmte Schritte nehmen lässt, gibt es in dieser App nicht (Nielsen~\autoref{itm:N7}).
So muss der Nutzer jedesmal, wenn er ein eine zuvor eingezeichnete Form beschriften will, in den ``Text-Modus'' wechseln, die gewünschte Form markieren, und anschließend die Messwerte eintragen. 
Alternativ hätte man hier zum Beispiel die Benutzung einer langen Klick-Aktion auf die gewünschte Form zum Beschrfiten nutzen können .

Durch das Hervorheben der markierten Form durch eine andere Farbe ermöglicht die App ein schnelles und effizientes Scannen des Bildschirminhaltes (Nielsen~\autoref{itm:N12}).

Das Malen von Formen durch das Klicken und anschließende Ziehen mit einem Finger auf dem Bildschirm ermöglicht eine einfache und effiziente Benutzung mit einer Hand (Nielsen~\autoref{itm:N17}).

Des Weiteren kann die App zu jeder Zeit pausiert werden, ohne dass eingetragene Messwerte oder Formen verloren gehen (Nielsen~\autoref{itm:N11}).

Im Kontrast dazu führen Bildschirmrotationen vom Hoch- ins Querformat zu einer verwirrenden Drehung des Bildes, welches sich entgegen aller Intuotion mit der Bildschirmausrichtung dreht, und nicht in der Ausrichtung, in der es aufgenommen wurde bleibt (Nielsen~\autoref{itm:N15}).

Zudem führt die teilweise fehlerhafte Umsetzung der Gesten-Unterstützung dazu, dass der Nutzer beim Zoomen des Bildes unabsichtlich eine Form malt, wenn er zuvor im ``Malen-Modus'' ist (Nielsen~\autoref{itm:N16}).
Dieser Punkt schmälert eine positive Benutzerfahrung enorm, da dies nach fast jeder Zoom-Aktion eine Undo-Aktion erzwingt, um die unabsichtlich-gemalte Form wieder zu entfernen (Nielsen~\autoref{itm:N13}).

Der Export des bearbeiteten Bildes führt dazu, dass alle eingetragenen Messwerte nicht mehr trivial auslesbar sind (da nicht als Meta-Daten verfügbar). Zudem würde sich die App nur schwer in eine bestehende Systemarchitektur integrieren lassen, da der Punkt der Aufmaßerfassung nur eine von vielen Funktionen der App ist. \todo{ref auf weitere Kriterien}

Zusammenfassend lässt sich also festhalten, dass die App \mm{} von der Bosch GmbH die Heuristiken nach Nielsen größtenteils erfüllt, sich jedoch nicht in eine bestehende Systemarchitektur integrieren lässt, und alle eingetragenen Messwerte nach Export des Bildes mühsam aus dem Bild extrahiert werden müssen, da diese dann nicht mehr als Meta-Daten vorliegen.

\subsection{Image Meter}

\subsubsection{Vorstellung}
Die App \im{} von \emph{Dirk Farin} hat zur Zeit des Downloads (20. Januar 2018) bei insgesamt 2 764 abgegeben Bewertungen eine durchschnittliche Bewertung von 3,9 von 5 Sternen im Google Play-Store \citep{FarinIM}.
Hierbei haben 72\% (1978) der Bewertungen vier oder fünf, und nur 28\% (786) 3 oder weniger Sterne.
Dies ist eine überdurchschnittlich hohe Bewertung, und macht die App zu einer der Beliebtesten unter dem Suchbegriff ``Aufmaße''.
Der Entwickler selbst beschreibt die App im Play-Store wie folgt \citep{FarinIM}:

\begin{quote}
  ``ImageMeter erlaubt das Beschriften Ihrer Fotos mit Längen-, Winkel- und Flächenmaßen sowie Text.
  Das ist viel einfacher und anschaulicher als aufwändig eine Skizze zu zeichnen.''
\end{quote}

\noindent
Beim Start der App wird dem Benutzer der sogenannte ``Tipp des Tages'' angezeigt.

\begin{wrapfigure}{R}{0.5\textwidth}
  \centering
  \includegraphics[keepaspectratio, width=0.5\textwidth]{image_meter/tip}
  \caption{``Tipp des Tages'' beim Start der App}
  \label{fig:imtip}
\end{wrapfigure}

Dieser enthält Informationen zu bestimmten Funktionen der App, wie zum Beispiel dem Exportieren von Bildern (siehe \autoref{fig:imtip}).
Hier hat der Nutzer die Möglichkeit sich weitere Tipps anzusehen, oder diese durch das Entfernen des Hakens in der Checkbox ``Tipp des Tages beim Start zeigen'' dauerhaft zu deaktivieren. \\

Sobald der Dialog zum ``Tipp des Tages'' geschlossen wurde, bietet sich über die Statusleiste am unteren Bildschirmrand die Möglichkeit, ein neues Bild aufzunehmen, oder direkt eines aus der Galerie zu importieren (siehe \autoref{fig:immenu}). \\

Das ausgewählte Bild wird nach erfolgreichem Import in die App im Hauptmenü in einer Liste mit alleren weiteren Bildern angezeigt.
Hier gelangt der Benutzer durch einen Klick auf das gewünschte Bild in eine neue Bildschirmoberfläche, in der das Bild beschriftet werden kann (siehe \autoref{fig:imdraw}).
In dieser Oberfläche wird dem Nutzer das zuvor ausgewählte Bild und eine neue Statusleiste am unteren Bildschirm angezeigt.
Oberhalb der Statusleiste befindet sich ein ``Floating Action Button'' (\emph{FAB}), der durch ein Fragezeichen-Icon gekennzeichnet ist.
Der \emph{FAB} öffnet beim Klick auf sich eine separate Hilfe-Seite, in der häufig gestellte Fragen und deren Antworten zu den Funktionen der App beschrieben stehen. \\

Zusätzlich zu dieser Hilfestellung, wird dem Benutzer beim Auswählen des Zeichen-Modus ein erklärender Text über der Statusleiste angezeigt (siehe \autoref{fig:imdraw}).
Dieser beschreibt genau, welche Aktion der Nutzer im aktuellen Systemzustand durchführen kann bzw. soll. \\

\begin{figure}[h]
  \centering
  \begin{subfigure}[t]{0.4\textwidth}
    \includegraphics[keepaspectratio, width=\textwidth]{image_meter/menu}
    \caption{Hauptansicht der App}
    \label{fig:immenu}	
  \end{subfigure}
  \begin{subfigure}[t]{0.4\textwidth}
    \includegraphics[keepaspectratio, width=\textwidth]{image_meter/draw}
    \caption{Aufmaße-Funktion mit eingezeichneter Form} 
    \label{fig:imdraw}	
  \end{subfigure}
  \caption{\im{} nach dem Start der App und in der Aufmaß-Funktion}
\end{figure}

\noindent
Die Statusleiste bietet dem Nutzer außerdem die Möglichkeit, die gewünschte Form und deren Farbe festzulegen.
Die Farbe bereits eingezeichneter Formen kann auch noch im Nachhinein angepasst werden.
Zu jeder Zeit sind nur die Funktionen in der Statusleiste auswählbar, die im aktuellen Systemzustand durchführbar sind.
So ist bspw. das Löschen von Formen nur dann möglich, wenn zuvor eine Form markiert wurde. \\

Insgesamt kann der Nutzer aus bis zu sieben verschiedene Formen (zehn in der Pro-Version) frei auswählen.
Unter diesen sieben Formen der Free-Version befinden sich auch zwei Referenz-Formen.
Diese können dazu genutzt werden, um Referenzlängen im Bild festzulegen, und ermöglichen so, dass alle nachträglich eingezeichneten Formen automatisch mit Messwerten versehen werden. \\

Eine Undo- bzw. Redo-Funktion ist über das \emph{Overflow-Menü} in der Statusleiste erreichbar.
Hierdurch können ausgeführte Aktionen rückgängig gemacht oder wiederholt werden. \\

Auch in dieser App lassen sich gespeicherte Bilder zu einem späteren Zeitpunkt weiter bearbeiten.
Außerdem können mehrere Bilder gleichzeitig zusammen in einer \emph{PDF} exportiert werden.

\subsubsection{Evaluation}\label{subsec:imeva}
Hilfe und Dokumentation (Nielsen~\autoref{itm:N10}) werden in dieser App in Form des ``Tipp des Tages'' (siehe \autoref{fig:imtip}) und die dedizierte Hilfe-Oberfläche bereits gestellt.

\begin{wrapfigure}{R}{0.5\textwidth}
  \centering
  \includegraphics[keepaspectratio, width=0.5\textwidth]{image_meter/faq}
  \caption{Hilfeoberfläche der App}
  \label{fig:imfaq}
\end{wrapfigure}

Hierbei bietet die Hilfe-Oberfläche Antworten auf eine vorgegebene Menge an Fragen zu den verschiedenen Funktionen der App an (siehe \autoref{fig:imfaq}). \\

Die Statusleiste zeigt zu jedem Zeitpunkt den ausgewählten Modus und die ausführbaren Aktionen an.
Hierdurch gibt die App dem Nutzer eine angemessene und verständliche Rückmeldung über den aktuellen Systemzustand der App (Nielsen~\autoref{itm:N1} \& \autoref{itm:N5}). \\

Formen können, nachdem sie in das Bild gezeichnet wurden, zu einem späteren Zeitpunkt in ihrer Farbe und Größe verändert werden.
Zusätzlich dazu gibt es in den Einstellungen der App weitere Optionen, um die App an die eigenen Bedürfnisse anzupassen.
So kann zum Beispiel eingestellt werden, ob Maßeinheiten angezeigt werden sollen, welche metrischen Einheiten benutzt werden sollen, oder wie viele Dezimalstellen für Messwerte verwendet werden solle.
Dies erlaubt nicht nur eine flexible Benutzung der App, sondern kann gleichzeitig zu einer Effizienzsteigerung führen, da die App nur einmal zu Beginn der Benutzung konfiguriert werden muss (Nielsten~\autoref{itm:N7}) \\

Die App versucht Situationen, in denen Fehler auftreten könnten, präventiv zu vermeiden.
Hierzu werden Aktionen, die beim Ausführen im aktuellen Systemzustand zu einem Fehler führen würden, ausgegraut und sind nicht auswählbar (Nielsen~\autoref{itm:N5} \& \autoref{itm:N9}).

\begin{wrapfigure}{R}{0.5\textwidth}
  \centering
  \includegraphics[keepaspectratio, width=0.5\textwidth]{image_meter/bar}
  \caption{Statusleiste in der Aufmaßfunktion}
  \label{fig:imbar}
\end{wrapfigure}

Das Ausgrauen der unbenutzbaren Icons kann jedoch in Kombination mit den anderen verwendeten Farben in der Statusleiste zu Verwirrung führen.
So werden hier zwei verschiedene Grautöne benutzt, die sich nur minimal unterscheiden (siehe \autoref{fig:imbar}). 
\todo{bisschen mehr hierzu} \\

Über einen Undo- bzw. Redo-Button hat der Benutzer die Möglichkeit, fehlerhafte Eingaben zu verbessern, oder Aktionen zu wiederholen (Nielsen \autoref{itm:N3}).
Bei der Benutzung im Hochformat sind beide Buttons jedoch nicht sichtbar, da sie in dem \emph{Overflow-Menü} an der rechten Seite der Statusleiste versteckt sind (siehe \autoref{fig:imbar}).
Der Benutzer ist hier also darauf angewiesen, das \emph{Overflow-Menü} anzuklicken, um zu wissen, dass sich dort die Optionen zum Undo bzw. Redo befinden.
Generell fühlt sich die Statusleiste mit den sieben Icons (inklusive \emph{Overflow-Menü}) zu überladen an.
Hier wäre die Verwendung einer zusätzlichen Menüleiste sinnvoll, die einen Teil der Icons übernimmt. \\
\todo{anders schreiben}

Ein positiver Aspekt dagegen liegt beim adäquaten Umgang mit Unterbrechungen, sowie der Unterstützung verschiedener Bildschirmausrichtungen (Nielsen~\autoref{itm:N11} \& \autoref{itm:N15}).
Beim Pausieren und Drehen der App gehen keine Informationen, wie bereits eingezeichnete Formen oder eingetragene Messwerte verloren, und das Bild bleibt stets in der erwarteten Ausrichtung. 

\begin{wrapfigure}{R}{0.5\textwidth}
  \centering
  \includegraphics[keepaspectratio, width=0.5\textwidth]{image_meter/lensebug}
  \caption{Zoom-Linse verdeckt Zeichenbereich}
  \label{fig:imlense}
\end{wrapfigure}

Hinzukommend wird der zusätzliche Bildschirmplatz der sich im Querformat ergibt genutzt, um alle Elemente der Statusleiste anzuzeigen, ohne Aktionen im Overflow-Menü zu verstecken. \\

Eine positive Benutzererfahrung ergibt sich aus der einfachen und effizienten einhändigen Handhabung in Kombination mit einer fehlerfreien Gesten-Unterstützung zur Navigation im Bild. (Nielsen~\autoref{itm:N13}, \autoref{itm:N16} \& \autoref{itm:N17}). \\

Außerdem bedient sich auch diese App beim Zeichnen von Formen einer Zoom-Linse, welche den Bereich um die Fingerposition vergößert darstellt.
Negativ fällt bei der Umsetzung dieser Funktion jedoch auf, dass die Zoom-Linse statisch in der oberen linken Ecke angezeigt wird, und sich bei Kollision mit dem Zeichenfinger nicht bewegt.
Dies kann dazu führen, dass der gezoomte Bereich beim Zeichnen die Form verdeckt und seine eigentlichen Aufgabe als Hilfestellung zur genaueren und schnelleren Zeichnung verfehlt (siehe \autoref{fig:imlense}). \\

Die App ermöglicht das Teilen von vorhandenen Bildern in diese, bietet beim Exportieren jedoch keine Möglichkeit die eingetragenen Messwerte als Meta-Daten beizubehalten.
So wäre es zwar möglich, Bilder aus der bestehenden Android-App an \im{} zu teilen, diese zu bearbeiten und anschließend abzuspeichern. 
Aber es gäbe keine Möglichkeit aus der bestehenden Android-App auf die eingetragenen Messwerte zuzugreifen, und diese für einen nachgeschalteten Diesnt aufzubereiten. \\

Zusammenfassend kann gesagt werden, dass die App \im{} von \emph{Dirk Farin} die Nielsen-Heuristiken überwiegend positive erfüllt.
Die versteckten Undo/Redo-Funktionalität und die Verwirrung des Nutzers durch die Verwendung ähnlicher Grautöne für unterschiedliche Aktionen in der Statsubar sind bei der Evaluation als Negativaspekte identifiziert worden.
Besonders diese beiden Punkte in Kombination mit der fehldenden Einführung bzw. Hilfe-Stellung beim ersten Start der Aufmaß-Funktion, wirken sich negativ auf die initiale Benutzererfahrung der App aus.
So muss der Benutzer, falls sich Fragen während des Einzeichnens von Formen ergeben, die aktuelle Oberfläche verlassen, in eine andere Oberfläche wechseln, und dort die passende Frage suchen. 

Hierdurch können Fehler entstehen, die mit der Nutzung einer Android-App für die Aufmaßerfassung verhindert werden sollten.

\subsection{Photo Measures}

\subsubsection{Vorstellung}
Die App \emph{Photo Measures} von \emph{Big Blue Pixel Inc.} hat zur Zeit des Downloads (20. Januar 2018) bei insgesamt 425 abgegeben Bewertungen eine durchschnittliche Bewertung von 4,3 von 5 Sternen im Google Play-Store \citep{PixelPM}.
Auch diese App wird wie die beiden zuvor vorgestellten Apps unter der Kategorie ``Effizienz'' gelistet.
Im Gegensatz zu den anderen Apps ist diese jedoch nicht kostenlos erhältlich, sondern kann für einen Preis von 3,99 Euro erworben werden. \todo{kaufen?} 
Die Beschreibung im Play-Store selbst lautet wie folgt \todo{cite aus playstore}:

\begin{quote}
  ``Photo Measures is the best and easiest way to save measures on your own photos on Android.
  [...] Whenever you need to save dimensions, sizes, angles or write down a detail you need to remember, Photo Measures will help you to be more efficient and more accurate.''
\end{quote}

\subsubsection{Evaluation}
\begin{wrapfigure}{R}{0.4\textwidth}
	\centering
	\includegraphics[keepaspectratio, width=0.4\textwidth]{photo_measures/help}
	\caption{Initialer Start}	
	\label{fig:pmhelp}
\end{wrapfigure}

Die App zeigt beim ersten Start ein helfendes Overlay, welches dem Nutzer genau erklärt, wie die App zu benutzen ist. Dieser Punkt fällt nach Nielsen unter \ref{itm:10} (Siehe Abbildung\ref{fig:pmhelp}) \\
 
Aktionen sind nur dann verfügbar, wenn sie benutzbar sind. Somit werden Fehler vorgebeugt, und der Benutzer weiß zu jeder Zeit, in welchem Systemzustand er sich befindet. Dies korrespondiert zu den Heuristiken~\ref{itm:1} und \ref{itm:5}. \todo{2 bilder mit bottom-bars} \\

Die App bedient sich einer Reihe universell bekannter Icons, sodass intuitiv erkennbar ist, welche Aktion sich hinter welchem Button verbirgt, ohne groß darüber nachdenken zu müssen. Zusätzlich stehen die entsprechenden Aktionen als Text unter den Icons. Dies kann nützlich sein, wenn ein Icon nicht auf Anhieb wiedererkannt wird. \ref{itm:4} \todo{ref auf pic von bottom-bars} \\

Des Weiteren gibt die App dem Benutzer die Möglichkeit, Formen, Größen und Farben anzupassen. Dies fördert eine flexible und effizente Benutzung. \ref{itm:7} \\

Ein durchaus schwerwiegender negativer Punkt liegt bei der Benutzerkontrolle \ref{itm:3} der App. So ist es dem Benutzer nicht möglich, über einen Undo- bzw. Redo-Button seine Aktionen zu revidieren. Dies ist gerade bei der Bearbeitung von Bildern, wo es viele aneinandergereihte Aktionen des Benutzers gibt, eine entscheidende Funktionalität, welche nicht nur die Gedächtnisbelastung des Benutzer senken, sondern auch den ``Joy of Use'' deutlich steigern kann. \\

Zudem bietet die App keine für den Nutzer erkennbare Ausstiegsmöglichkeit \ref{itm:6} an. Es gibt weder einen Zurück-Button, noch einen Button um das annotierte Bild explizit zu speichern. Die einzige Ausstiegsmöglichkeit erfolgt über die Zurück-Navigationstaste des Smartphones, welche das Bild auch zusätzlich speichert. Diese Lösungsvariante ist für den Nutzer nicht intuitiv verständlich. \todo{bild}

Die App erfüllt nahezu alle acht (\ref{itm:11}-\ref{itm:18}) Heuristiken für mobile Geräte. Zu jeder Zeit ist auf dem Bildschirm erkennbar, welche Form zur Zeit ausgewählt ist. Das Smartphone kann während der Benutzung pausiert bzw. gedreht werden, ohne dass Informationen verloren gehen, oder der Benutzer durch unbekannte Bausteine überrascht wird. \todo{screens} Hier fällt als einziger negativer Punkt die unzureichende Gesten-Unterstützung auf \ref{itm:13}. So malt der Benutzer unabsichtlich mit jeder Zoom-Geste eine Form in das Bild, welche danach wieder gelöscht werden muss, da es keine Undo-Funktion gibt. \\



\section{Auswertung der Evaluationsergebnisse}
\begin{sidewaystable}[ht]
  \centering
  \caption{Vergleich der Lösungsalternativen}
  \vspace*{10px}
  \label{tab:nielsen}
  \begin{tabular}{r|l|c|c|c|}
    \cline{2-5}
    &								& Photo Measures 	& Measuring Master 	& Measure \& Sketch \\ \cline{2-5} 
    Nach \cite{Nielsen94} 	& \autoref{itm:1}				&       \po 		&    \po 			&       \xmark      \\ \cline{2-5} 
    & \autoref{itm:2} 				&       \po  		&    \po  			&       \po		    \\ \cline{2-5}
    & \autoref{itm:3} 				&       \xmark 		&    \po			&       \xmark      \\ \cline{2-5} 
    & \autoref{itm:4} 				&       \po  		&    \po			&       \xmark      \\ \cline{2-5}
    & \autoref{itm:5} 				&       \po  		&    \xmark			&       \xmark      \\ \cline{2-5} 
    & \autoref{itm:6} 				&       \xmark 		&    \po  			&       \xmark      \\ \cline{2-5} 
    & \autoref{itm:7} 				&       \po  		&    \xmark			&       \xmark      \\ \cline{2-5} 
    & \autoref{itm:8} 				&       \nl  		&    \po  			&       \xmark      \\ \cline{2-5} 
    & \autoref{itm:9} 				&       \po   		&    \po  			&       \nl	        \\ \cline{2-5} 
    & \autoref{itm:10} 				&       \po  		&    \po 			&       \xmark      \\ \cline{2-5} 
    & \autoref{itm:11} 				&       \po   		&    \po 			&       \xmark      \\ \cline{2-5} 
    & \autoref{itm:12} 				&       \po   		&    \po 			&       \xmark      \\ \cline{2-5} 
    & \autoref{itm:13}			 	&       \xmark  	&    \xmark			&       \xmark      \\ \cline{2-5} 
    & \autoref{itm:14} 				&       \po   		&    \po  			&       \po		    \\ \cline{2-5}
    & \autoref{itm:15} 				&       \po   		&    \xmark			&       \xmark  	\\ \cline{2-5}   
    & \autoref{itm:16} 				&       \po   		&    \po  			&       \xmark      \\ \cline{2-5} 
    & \autoref{itm:17} 				&       \po  		&    \po  			&       \xmark		\\ \cline{2-5} 
    & \autoref{itm:18} 				&       \nl  		&    \nl 			&       \nl		    \\ \cline{2-5} 
    Eigene Kriterien 				& \autoref{itm:integration}		&      	\xmark		&    \xmark			&       \xmark      \\ \cline{2-5}
    & \autoref{itm:export}   		&      	\xmark		&    \xmark			&       \xmark      \\ \cline{2-5}


  \end{tabular}
  \\
  \vspace*{10px}
  \begin{tabular}{l}
    \po~wird erfüllt \\
    \nl~wurde nicht berücksichtigt \\
    \xmark~wird nicht erfüllt
  \end{tabular}
\end{sidewaystable}


\todo{Überarbeiten da kein MS mehr}
Wie sich in \autoref{sec:evaluation} gezeigt hat, haben sich die beiden Apps \mm{} und \pm{} bezüglich der Usability-Heuristiken nach \citeauthor{Nielsen94} deutlich besser als \ms{} identifizieren lassen. 
Hierbei ist besonders aufgefallen, dass sich alle drei Apps einer Zoom-Linse bedienen, die während des Zeichnens von Formen auf dem Bildschirm angezeigt wird, und das Bild im Bereich der aktuellen Zeichenposition vergrößert darstellt.
Außerdem benutzen die Apps alle eine Statusbar, die am unteren Rand des Bildschirms verankert ist, und den aktuellen Systemzustand der App anzeigt.
Bei der Alternative der \emph{Bosch GmbH} werden Aktionen, die zur Zeit nicht ausführbar sind, ausgegraut dargestellt. Dies gibt dem Benutzer eine klare und intuitive Rückmeldung über alle Aktionen, die im aktuellen Systemzustand möglich sind.
Zudem fällt auf, dass alle Apps Probleme mit der Gesten-Unterstützung haben, da sie entweder gar nicht (\todo{ref auf measure sketch}) oder fehlerhaft (\todo{ref andere beiden}) umgesetzt wurde. \\
Besonders dieser negative Punkt wirkt sich negativ auf das Benutzererlebnis aus, da eine intuitive und effiziente Navigation gerade bei Geräten mit kleineren Bildschirmgrößen essentiell sind. \todo{ref auf papers}
So schreibt bla in seinem Paper blub, dass sich bla \dots

Ein weiterer wichtiger Aspekt, der von keiner der evaluierten Apps erfüllt wurde, ist die Integration in die bereits bestehende Android-Applikation bzw. der Export der bearbeiteten Bilder ohne die eingetragenen Messwerte als Meta-Daten zu verlieren. \\

Diese Evaluationsergebnisse sollen im Nachfolgenden Kapitel auf die eigene Konzeption einer Android-Applikation angewandt werden.
Hierbei Aspekte, die besonders positiv aufgefallen sind, in die Konzeption einfließen, und Punkte, die sich bereits bei den Alternativen als Fehlerquellen identifizieren lassen haben, vermieden werden.
\todo{ein bisschen ausbauen}
