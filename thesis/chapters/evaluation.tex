\chapter{Evaluation vorhandener Lösungsalternativen}
Im folgenden Kapitel werden drei verschiedene Lösungsalternativen zur Aufmaßerfassung vorgestellt und bewertet.
Hierzu werden zunächst die Bewertungskriterien vorgestellt und gewichtet.
Anschließend werden die ausgewählten Apps iterativ vorgestellt und mit Hilfe der Kriterien evaluiert.
\todo{apps hier schon benennen?}

\section{Bewertungskriterien}
Zur Evaluation der Lösungsalternativen wird die erweiterte Version der Nielsen-Heuristiken \citep{Nielsen94}, sowie eine ausgewählte Menge selbst aufgestellter Kriterien benutzt.
Da sich die erweiterten Nielsen-Heuristiken aus einer Menge von 18 Kriterien zusammensetzen und nicht jede einzelene für die Bewertung der Lösungsalternativen gleich relevant ist, werden diese jeweils mit einem Bewertungsschema (A, B, C) so gewichtet, dass für die Evaluation der Apps nur für uns relevante Kriterien berücksichtig werden. \\

\subsection{Die erweiterten Nielsen-Heuristiken}\label{subsec:nielsen}
 Nach \citeauthor{Nielsen94} gibt es zehn Heuristiken, die auf einer Grundlage von allgemein anerkannten Prinzipien beruhen, und sich zur Evaluation von Usability-Problemen besonders gut eignen \citep[Seite.~25--62]{Nielsen94}: 

\begin{enumerate}
    \di{Sichtbarkeit des Systems}{N1}{Angemessene Rückmeldung in einem vernünftigen zeitlichen Rahmen}
    \di{Übereinstimmung zwischen System und realer Welt}{N2}{Sprache des Benutzers, natürliche und logische Reihenfolge}
    \di{Benutzerkontrolle und -freiheit}{N3}{Undo und Redo}
    \di{Konsistenz und Standards}{N4}{Konsistente Nutzung und Beachtung der Plattform-Konventionen}
    \di{Fehlervorbeugung}{N5}{Vermeide Situationen in denen Fehler entstehen können}
    \di{Wiedererkennung statt Erinnern}{N6}{Objekte, Optionen und Aktionen sollen sichtbar sein}
    \di{Flexibilität und Effizienz der Benutzung}{N7}{Häufige Aktionen anpassbar, Experten schnellere Bedienung erlauben}
    \di{Ästethisches und minimalistisches Design}{N8}{Keine irrelevanten Informationen}
    \di{Erkennbarkeit, Diagnose und Erholung von Fehlern}{N9}{Verständliche Fehlermeldungen}
    \di{Hilfe und Dokumentation}{N10}{Leicht zu finden, abgestimmt auf Aufgabe, konkrete Schritte zur Lösung}
\end{enumerate} 

\noindent
Zur Bewertung mobiler Endgeräte reichen diese zehn Bewertungskriterien jedoch nicht vollständig aus, sodass weitere acht Heuristiken, die speziell für mobile Geräte ausgelegt sind, hizugezogen werden \todo{ref auf ue, 239}:

\begin{enumerate}
	  \setcounter{enumi}{10}
    \di{Adäquater Umgang mit Unterbrechnungen}{N11}{Kein Verlust von Informationen bei Unterbrechung der Aktivität}
    \di{Fokussieren der Informationen}{N12}{Hervorheben der wichtigen Informationen, um schnelles Scannen zu erlauben}
    \di{``Joy of Use''}{N13}{Positives Nutzungserlebnis bei Bedienung/Verwendung}
    \di{``Don't lie to the user''}{N14}{Keine falschen oder ungültigen Zahlen, Fakten bzw. Nachrichten}
    \di{Unterstützung verschiedener Bildschirmausrichtungen}{N15}{Keine Verwirrung bei verschiedenen Ausrichtungen des Bildschirms}
    \di{Ergonomische Gestaltung der physischen Interatkion}{N16}{Kein unabsichtliches Auslösen von Funktionen}
    \di{Einfache Eingabe, Bildschirmlesbarkeit und Übersichtlichkeit}{N17}{Einfache Navigations- und Eingabetechniken bei einhändiger Benutzung}
    \di{Stelle Privatheit sicher}{N18}{Vermeide den Verlust/Diebstahl von privaten Daten z.b. durch Passwörter}
\end{enumerate}

\subsection{Weitere Kriterien}
Zusätzlich zu den in \autoref{subsec:nielsen} vorgestellen Heuristiken werden noch zwei weitere Kriterien zur Evaluation hinzugezogen, die für die Benutzung der App aus Sicht der Aufmaßerfassung im Gerüstbau wichtig sind: \todo{anders formulieren}

\begin{itemize}
    \di{Integration der App in eine vorhandene Systemarchitektur}{integration}{Die Software-Lösung sollte sich in eine bereits vorhandene Systemarchitektur integrieren lassen}
    \di{Export des annotierten Bildes und der Metadaten zur Weiterverarbeitung in einem nachgelagerten Dienst (z.B. API)}{export}{Die App sollte die eingetragenen Meta-Daten, wie zum Beispiel die Längen einer Linie, zur Weiterverarbeitung exportierbar machen}
	  \todo{mehr Kriterien raussuchen}
\end{itemize}

\section{Vorstellung und Evalutation ausgewählter Apps}
Als Lösungsalternativen stehen die drei Android-Applikationen \emph{Measuring Master}, \emph{Measures \& Sketch} und \emph{Photo Measure} zur Verfügung.

Bosch über 80 Standorte in Deutschland, und 14.5 Mrd. Euro Jahresumsatz \\
Photo Measures gute Bewertung mit 4/5 Sternen bei 10k-50k Downloads \\
Measure \& Sketch mit rund 100k-500k Downloads sehr populär \\

\subsection{Measuring Master}

\subsubsection{Vorstellung}
\todo{Version, Downloaddatum, Playstore-Link}
Die App \mm{} von der Bosch GmbH ist im Play-Store unter der Rubrik ``Effizienz'' aufgelistet.
Selbst beschreibt der App-Hersteller seine Software wie folgt \citep{BoschMM}:

\begin{quote}
  ``Measuring Master ist eine multifunktionale App, die es ermöglicht, Aufmaße, Grundrisse und Temperaturmesswerte an einem Ort zu dokumentieren und zu verwalten.\\
  Die App ist besonders geeignet für Architekten, Maler, Bodenleger, Heizungsbauer und Elektriker, aber auch alle anderen Handwerker profitieren von der umfangreichen Funktionalität''
\end{quote}

\noindent
Nach dem Start der Applikation bietet sich die Möglichkeit ein neues Projekt anzulegen, oder bereits vorhandene Projekte zu bearbeiten.
Sobald das gewünschte Projekt ausgewählt wurde, bieten sich dem Benutzer über das Menü an der linken Seite diverse Funktionen (siehe \autoref{fig:bmenu}) an. \\

\begin{figure}[h]
  \centering
  \begin{subfigure}[t]{0.4\textwidth}
    \includegraphics[keepaspectratio, width=\textwidth]{bosch/menu}
    \caption{Navigationsmenü}\label{fig:bmenu}	
  \end{subfigure}
  \begin{subfigure}[t]{0.4\textwidth}
    \includegraphics[keepaspectratio, width=\textwidth]{bosch/bar}
    \caption{Aufmaße-Funktion}\label{fig:bbar}
  \end{subfigure}
  \caption{\mm{} bei ausgeklapptem Navigationsmenü und in der Aufmaße-Funktion}
\end{figure}

Im Folgenden wird der Menüpunkt ``Aufmaße'' und die damit verbundene Funktionalität weiter vorgestellt und evaluiert.
So bietet sich dem Benutzer nach Auswahl der Aufmaße-Funktion die Möglichkeit, ausgewählte Bilder direkt mit Messwerten zu beschriften.
Hierzu können entweder Bilder direkt aus der Gallerie importiert, oder mit der Kamera aufgenommen werden.
Sobald der Benutzer den Foto-Import erfolgreich abgeschlossen hat, öffnet sich eine neue Ansicht, welche das ausgewählte Bild und weitere Bedienelement, in Form einer Statusleiste am unteren Rand, zeigt (siehe \autoref{fig:bbar}). \\

In dieser Benutzeroberfläche kann der Nutzer mit Hilfe von vier verschiedenen Formen (Linie, Viereck, Winkel, Freiform) Aufmaße ins Bild anzeichnen, und über einen Klick auf die gewünschte Form, Messwerte eintragen.
Außerdem bietet die App die Option, eine ausgewählte Reihe von Laserentfernungsmesser mit der App zu verbinden.
Dies ermöglicht eine Übertraung der gemessenen Distanzen über eine Bluetooth-Schnittstelle direkt an die App. \\

Um das annotierte Bild außerhalb der App weiter zu benutzen, bietet diese den Export als \emph{PDF} und \emph{JPEG} an.
Die exportierte \emph{PDF} enthält im Gegensatz zu der \emph{JPEG} zusätzlich zu dem annotierten Bild auch noch eine Tabelle mit allen eingetragenen Messwerten \autoref{fig:bexport}. 
Zudem lassen sich annotierte Bilder in der App speichern und zu einem späteren Zeitpunkt wieder bearbeiten.

\begin{figure}[h]
  \centering
  \includegraphics[keepaspectratio, width=\textwidth]{bosch/export}
  \caption{Exportierte PDF}\label{fig:bexport}
\end{figure}

\subsubsection{Evaluation}

Durch die Anzeige der Statusleiste am unteren Bildschirmrand (siehe \autoref{fig:bbar}) gibt die App dem Benutzer zu jeder Zeit eine klare und sichtbare Rückmeldung über den aktuellen Systemzustand (Nielsen~\autoref{itm:N1}).

Zusätzlich bietet ein erklärender Text im oberen Bereich des Bildschirms eine konkrete Hilfestellung zu den möglichen Aktionen, die der Benutzer im aktuellen Systemzustand ausführen kann (Nielsen~\autoref{itm:N10}).

Darüber hinaus kann der Nutzer über ein Undo- bzw. Redo-Symbol fehlerhafte oder unabsichtliche Eingaben rückgängig machen, oder rückgängig gemachte Aktionen wiederholen (Nielsen~\autoref{itm:N3}).

Dadurch, dass Aktionen nur dann ausführbar sind, wenn man den richtigen Modus in der Statusleiste ausgewählt hat, werden Fehler präventiv vermieden. So lassen sich Messwerte erst dann einfügen, wenn zuvor eine Form markiert wurde, oder Aktionen rückgängig machen, wenn zuvor eine Benutzeraktion stattgefunden hat (Nielsen~\autoref{itm:N5}).

Schafft man es trotzdem einen Fehlerzustand des Systems herzustellen, wird dieser durch einen ``Toast'' \todo{def} erkennbar gemacht, und gibt dem Benutzer Feedback darüber, was zum Fehler geführt hat \todo{Bild vom Toast beim Bildimport} (Nielsen~\autoref{itm:N9}).

Durch die konsistente Benutzung bekannter Icons lassen sich die dahinter befindlichen Aktionen intuitiv erkennen, und die Gedächtnislast des Benutzers reduzieren. 
So wird zum Beispiel sowohl im ``Malen-Modus'' als auch im ``Text-Modus'' das Mülleimer-Icon in der oberen Statusleiste als Löschfunktion der markierten Form bzw.\ des ausgewählten Textes benutzt (Nielsen~\autoref{itm:N4} \& \autoref{itm:N6}).

Einen ``Expertenmodus'', der dem Benutzer voreingestelle Aktionen ändern, oder Abkürzungen für bestimmte Schritte nehmen lässt, gibt es in dieser App nicht (Nielsen~\autoref{itm:N7}).
So muss der Nutzer jedesmal, wenn er ein eine zuvor eingezeichnete Form beschriften will, in den ``Text-Modus'' wechseln, die gewünschte Form markieren, und anschließend die Messwerte eintragen. 
Alternativ hätte man hier zum Beispiel die Benutzung einer langen Klick-Aktion auf die gewünschte Form zum Beschrfiten nutzen können .

Durch das Hervorheben der markierten Form durch eine andere Farbe ermöglicht die App ein schnelles und effizientes Scannen des Bildschirminhaltes (Nielsen~\autoref{itm:N12}).

Das Malen von Formen durch das Klicken und anschließende Ziehen mit einem Finger auf dem Bildschirm ermöglicht eine einfache und effiziente Benutzung mit einer Hand (Nielsen~\autoref{itm:N17}).

Des Weiteren kann die App zu jeder Zeit pausiert werden, ohne dass eingetragene Messwerte oder Formen verloren gehen (Nielsen~\autoref{itm:N11}).

Im Kontrast dazu führen Bildschirmrotationen vom Hoch- ins Querformat zu einer verwirrenden Drehung des Bildes, welches sich entgegen aller Intuotion mit der Bildschirmausrichtung dreht, und nicht in der Ausrichtung, in der es aufgenommen wurde bleibt (Nielsen~\autoref{itm:N15}).

Zudem führt die teilweise fehlerhafte Umsetzung der Gesten-Unterstützung dazu, dass der Nutzer beim Zoomen des Bildes unabsichtlich eine Form malt, wenn er zuvor im ``Malen-Modus'' ist (Nielsen~\autoref{itm:N16}).
Dieser Punkt schmälert eine positive Benutzerfahrung enorm, da dies nach fast jeder Zoom-Aktion eine Undo-Aktion erzwingt, um die unabsichtlich-gemalte Form wieder zu entfernen (Nielsen~\autoref{itm:N13}).

Der Export des bearbeiteten Bildes führt dazu, dass alle eingetragenen Messwerte nicht mehr trivial auslesbar sind (da nicht als Meta-Daten verfügbar). Zudem würde sich die App nur schwer in eine bestehende Systemarchitektur integrieren lassen, da der Punkt der Aufmaßerfassung nur eine von vielen Funktionen der App ist. \todo{ref auf weitere Kriterien}

Zusammenfassend lässt sich also festhalten, dass die App \mm{} von der Bosch GmbH die Heuristiken nach Nielsen größtenteils erfüllt, sich jedoch nicht in eine bestehende Systemarchitektur integrieren lässt, und alle eingetragenen Messwerte nach Export des Bildes mühsam aus dem Bild extrahiert werden müssen, da diese dann nicht mehr als Meta-Daten vorliegen.

\subsection{Measure \& Sketch}
\subsubsection{Vorstellung}
Die App \ms{} von \emph{SameBits} entwickelt.
Zur Zeit des Downloads (20. Januar 2018) ist die Applikation laut Google Play-Store auf zwischen $100.000$ und $500.000$ Android-Geräten installiert.
Auch diese Applikation ist unter der Kategorie ``Effizienz'' gelistet, und wird vom Entwickler mit den folgenden Worten beschrieben \citep{BitsMS}:

\begin{quote}
  ``Die Must-Have Zeichenapp für alle echten Ingenieure, Architekten, Bauarbeiter, Immobilienmakler, Handwerker wund [sic] natürlich für alle Heimwerker!''
\end{quote}

\noindent
Beim initialen Start der App wird der Benutzer mit Hilfe eines Overlays auf die möglichen Aktionen, die er in diesem Zustand tätigen kann, hingewiesen.
Hier bietet sich die Option zwischen den bestehenden Projekten zu wechseln, oder ein Neues anzulegen.
Über den Knopf ``Neu'' (siehe \autoref{fig:msmenu} kann der Nutzer ein neues Bild aufnehmen, oder direkt eines aus der Galerie importieren. \\

\begin{figure}[h]
  \centering
	\begin{subfigure}[t]{0.4\textwidth}
		\includegraphics[keepaspectratio, width=\textwidth]{measure_sketch/menu}
		\caption{Startbildschirm}
		\label{fig:msmenu}	
	\end{subfigure}
	\begin{subfigure}[t]{0.4\textwidth}
		\includegraphics[keepaspectratio, width=\textwidth]{measure_sketch/menu}
		\caption{Aufmaße-Funktion} 
		\label{fig:msbar}	
	\end{subfigure}
  \caption{\ms{} beim Start der App und in der Aufmaße-Funktion}
\end{figure}

\noindent
\todo{Bilder mit Overlay}
Sobald man ein Bild ausgewählt hat, öffnet sich eine neue Benutzeröberfläche, in der, wie schon beim Start der App, durch ein Overlay alle möglichen Aktionen beschrieben werden (siehe \autoref{fig:msbar}).
In dieser Ansicht kann man das Bild mit drei verschiedenen Formen (Linie, Winkel, Freitext) annotieren. 
Zusätzlich können eingezeichnete Formen mit Messwerten beschriftet werden.\\

Weiterhin bietet sich die Gelegenheit, das bearbeitete Bild zur Galerie, \emph{Evernote} oder Universal \todo{gucken was gemeint ist} zu exportieren, oder direkt per E-Mail zu verschicken.
Auch bei dieser App kann man modifizierte Bilder speichern, und zu einem späteren Zeitpunkt zur Weiterverarbeitung wieder öffnen. \\

\subsubsection{Evaluation}

\begin{wrapfigure}{R}{0.4\textwidth}
	\includegraphics[keepaspectratio, width=0.4\textwidth]{measure_sketch/help_start}
	\caption{Hilfe-Overlay}
	\label{fig:mshelp}
\end{wrapfigure}

Auch diese App bedient sich eines Hilfe-Overlays beim ersten Start, überfordert den Benutzer jedoch mit zu viel Text. So erfüllt dieses Tooltip ihre Funktion als Hilfestellung nicht, sondern überfordert den Nutzer mit Text. Alternativ bietet sich auch bei dieser App wie in \autoref{fig:pmhelp} mit Icons zu arbeiten, sodass die Gedächtnisbelastung minimiert wird. \\

Die App kann als negativ-Beispiel bezüglich der Nielsen-Heuristiken betrachtet werden. Es gibt weder Undo- oder Redo-Button, noch wird in irgendeiner Weise hervorgehoben, welche Form aktuell ausgewählt ist. Dies führte beim Löschen oft zu Überraschungen. \\

 Außerdem unterstützt die App keinerlei Gesten zur Navigation im Bild. So lässt sich der abgebildete Bereich des Bildes weder zoomen, noch kann der Benutzer das Bild rotieren oder verschieben. Um Formen zu zeichnen bedient sich die App eine für den Benutzer unnatürliche Geste, denn hierzu muss der Nutzer gleichzeitig mit zwei Fingern die Form in die beiden gewünschten Richtungen ``aufziehen''. So fühlt sich der Zeichen-Prozess nicht nur unnatürlich an, sondern ist in der Größe der Form durch die Spannweite der Finger des Benutzers beschränkt. \\
 
 Zusätzlich schließt dies die Benutzung der App mit einer Hand aus, was Nielsen~\autoref{itm:16} klar widerspricht. Als Bildschirmausrichtung wird nur der Portrait-Modus unterstützt, was gerade die Bearbeitung von im Landscape aufgenommenen Bildern zu einer Herausforderung macht. \\


\subsection{Photo Measures}

\subsubsection{Vorstellung}
Die App \emph{Photo Measures} von \emph{Big Blue Pixel Inc.} hat zur Zeit des Downloads (20. Januar 2018) bei insgesamt 425 abgegeben Bewertungen eine durchschnittliche Bewertung von 4,3 von 5 Sternen im Google Play-Store \citep{PixelPM}.
Auch diese App wird wie die beiden zuvor vorgestellten Apps unter der Kategorie ``Effizienz'' gelistet.
Im Gegensatz zu den anderen Apps ist diese jedoch nicht kostenlos erhältlich, sondern kann für einen Preis von 3,99 Euro erworben werden. \todo{kaufen?} 
Die Beschreibung im Play-Store selbst lautet wie folgt \todo{cite aus playstore}:

\begin{quote}
  ``Photo Measures is the best and easiest way to save measures on your own photos on Android.
  [...] Whenever you need to save dimensions, sizes, angles or write down a detail you need to remember, Photo Measures will help you to be more efficient and more accurate.''
\end{quote}

\subsubsection{Evaluation}
\begin{wrapfigure}{R}{0.4\textwidth}
	\centering
	\includegraphics[keepaspectratio, width=0.4\textwidth]{photo_measures/help}
	\caption{Initialer Start}	
	\label{fig:pmhelp}
\end{wrapfigure}

Die App zeigt beim ersten Start ein helfendes Overlay, welches dem Nutzer genau erklärt, wie die App zu benutzen ist. Dieser Punkt fällt nach Nielsen unter \ref{itm:10} (Siehe Abbildung\ref{fig:pmhelp}) \\
 
Aktionen sind nur dann verfügbar, wenn sie benutzbar sind. Somit werden Fehler vorgebeugt, und der Benutzer weiß zu jeder Zeit, in welchem Systemzustand er sich befindet. Dies korrespondiert zu den Heuristiken~\ref{itm:1} und \ref{itm:5}. \todo{2 bilder mit bottom-bars} \\

Die App bedient sich einer Reihe universell bekannter Icons, sodass intuitiv erkennbar ist, welche Aktion sich hinter welchem Button verbirgt, ohne groß darüber nachdenken zu müssen. Zusätzlich stehen die entsprechenden Aktionen als Text unter den Icons. Dies kann nützlich sein, wenn ein Icon nicht auf Anhieb wiedererkannt wird. \ref{itm:4} \todo{ref auf pic von bottom-bars} \\

Des Weiteren gibt die App dem Benutzer die Möglichkeit, Formen, Größen und Farben anzupassen. Dies fördert eine flexible und effizente Benutzung. \ref{itm:7} \\

Ein durchaus schwerwiegender negativer Punkt liegt bei der Benutzerkontrolle \ref{itm:3} der App. So ist es dem Benutzer nicht möglich, über einen Undo- bzw. Redo-Button seine Aktionen zu revidieren. Dies ist gerade bei der Bearbeitung von Bildern, wo es viele aneinandergereihte Aktionen des Benutzers gibt, eine entscheidende Funktionalität, welche nicht nur die Gedächtnisbelastung des Benutzer senken, sondern auch den ``Joy of Use'' deutlich steigern kann. \\

Zudem bietet die App keine für den Nutzer erkennbare Ausstiegsmöglichkeit \ref{itm:6} an. Es gibt weder einen Zurück-Button, noch einen Button um das annotierte Bild explizit zu speichern. Die einzige Ausstiegsmöglichkeit erfolgt über die Zurück-Navigationstaste des Smartphones, welche das Bild auch zusätzlich speichert. Diese Lösungsvariante ist für den Nutzer nicht intuitiv verständlich. \todo{bild}

Die App erfüllt nahezu alle acht (\ref{itm:11}-\ref{itm:18}) Heuristiken für mobile Geräte. Zu jeder Zeit ist auf dem Bildschirm erkennbar, welche Form zur Zeit ausgewählt ist. Das Smartphone kann während der Benutzung pausiert bzw. gedreht werden, ohne dass Informationen verloren gehen, oder der Benutzer durch unbekannte Bausteine überrascht wird. \todo{screens} Hier fällt als einziger negativer Punkt die unzureichende Gesten-Unterstützung auf \ref{itm:13}. So malt der Benutzer unabsichtlich mit jeder Zoom-Geste eine Form in das Bild, welche danach wieder gelöscht werden muss, da es keine Undo-Funktion gibt. \\



\section{Auswertung der Evaluationsergebnisse}
Fazit und Überleitung zur Konzeption eigener App
