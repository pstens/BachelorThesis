\chapter{Fazit}
Das Ziel dieser Bachelorarbeit war es, eine effiziente und intuitive Software-Lösung in Form einer Android-App für die Aufmaßerfassung im Gerüstbau zu entwickeln.
Hierzu wurde eingangs die Problematik der bisherigen, analogen Aufmaßerfassung am Beispiel der \emph{Fa.} \vr{} veranschaulicht und anschließend ein optimierter, digitaler Prozess vorgestellt.
Potentielle Fehlerquellen des analogen Prozesses ließen sich vor allem auf zu grobe Schätzungen, Medienbrüche und der hohen kognitiven Belastung der Monteure zurückführen.
Um genau diese Fehlerquellen zu vermeiden, sollte eine Android-App entwickelt werden, die von den Monteuren bei der Aufmaßerfassung im Gerüstbau eingesetzt werden kann. \\

Hierzu wurden drei Apps aus dem Google-Play Store als mögliche Lösungsalternativen mit Hilfe verschiedener Bewertungskriterien bzgl. ihrer Usability, und Integration in eine bestehende Systemarchitektur evaluiert.
Zur Evaluation der Usability-Eigenschaften wurde eine erweiterte Version der Nielsen-Heuristiken angewandt, welche die Benutzung mobiler Endgeräte berücksichtigt.
Die Integrationsfähigkeit der Apps wurde mit Hilfe zweier selbst-aufgestellter Kriterien zum Export der Meta-Daten und deren späteren Weiterverarbeitung evaluiert.
Am Ende dieser Evaluation ließ sich festhalten, dass keine der drei Apps die Bewertungskriterien vollständig umsetzen konnte.
Die beiden Apps \mm{} und \pm{} schnitten im Vergleich zu \im{} bzgl. der Usability-Heuristiken unterdurchschnittlich schlecht ab.
Bei beiden Apps fiel insbesondere die fehlerhafte Umsetzung der Gesten-Unterstützung zum Zoomen des Bildes auf.
Im Gegensatz dazu hat die App \im{} die erweiterten Nielsen-Heuristiken relativ gut umgesetzt.
Hier fehlte jedoch eine Hilfestellung zur Benutzung der Zeichen-Funktion beim initialen Start der App, wodurch der Nutzer die eigentliche Zeichen-Oberfläche verlassen muss, um in einer Liste vordefinierter Fragen, die richtige Antwort zu finden.
Keiner der drei evaluierten Apps ließ sich in eine bestehende Systemarchitektur integrieren.
Hierbei scheiterten alle drei Apps beim Export des Bildes ohne die eingetragenen Messwerte als Meta-Daten zu verlieren.
Aufgrund dieser Evaluationsergebnisse wurde entschieden, eine eigene Android-App zu konzipieren. \\

Zur Konzeption der eigenen App wurde die Forschungsmethode des \hcdp{} nach \citet{Norman13} angewandt.
In insgesamt vier Iterationen des \hcdp{} wurde die App bzgl. ihrer Funktionalität und Usability so weit verbessert, dass es beim gemeinsamen Testen mit ausgewählten Testpersonen aus der \emph{Fa.} \vr{} keine neuen Usability-Probleme oder Anwenderwünsche mehr gab. \\

Schlüsselveränderungen zwischen dem Prototyp aus der ersten und vierten Iteration liegen vor allem in einer vereinfachten und intuitiveren Bedienoberfläche der App.
Hier wurde eine Statusleiste eingeführt, die den Nutzer jederzeit über den aktuellen Systemzustand der App informiert und einen schnellen Zugriff zu den wichtigsten Funktionen der App gibt.
Außerdem wurden insgesamt drei Hilfe-Overlays eingeführt, die dem Benutzer beim initialen Start der App und dem jeweiligen Wechsel in den Zeichen- bzw. Text-Modus kurz und präzise alle wichtigen Informationen zur Benutzung der App zeigen.
Funktional wurde die App im Laufe des \hcdp{} um eine \emph{Crop-Library} erweitert, welche es dem Benutzer ermöglicht, bereits beim Importieren Bilder in die gewünschte Form und Größe zu bringen. \\

Während des gesamten Entwicklungsprozesses der eigenen Android-App hat sich der \hcdp{} als gute Wahl für die Forschungsmethodik ausgezeichnet.
Die Aufteilung des gesamten Entwicklungsprozesses in kleine Iterationen hat positiv dazu beigetragen, dass nahe am Endanwender entwickelt und getestet werden konnte.
Auf diese Weise konnten Usability-Probleme oder Anwenderwünsche bereits früh identifiziert werden.

\todo{Forschungsfrage beantworten -> Ist mobile Aufmaßerfassung effizienter?}
