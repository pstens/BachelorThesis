\chapter{Implementierung}

Die gesammelten Eindrücke aus den Kapiteln Evaluation und Konzeption sollen im Folgenden dazu genutzt werden, einen ersten Prototyp zu entwickeln.
Hiezu bietet es sich an den Prototyp als \emph{Android Library} \todo{def} zu implementieren, da dies sowohl schnellere Kompilierzeiten als auch die Einbindung in eine bestehende Android-Applikation deutlich erleichtert.
Als Programmiersprache soll die von \emph{JetBrains} entwickelte Sprache \emph{Koltin} verwendet werden, welche 2017 als offiziell unterstützte Android-Sprache vorgestellt wurde.
Die Sprache bietet im Vergleich zu \emph{Java} einen besonders guten Umgang bezüglich der \emph{Null-Safety} von Variablen. \todo{noch ein bisschen mehr warum ausgerechnet Kotlin}


Als Architekturmuster soll das \emph{Model-View-Controller-Konzept} (kurz \emph{MVC}) verwendet werden.
Dies erlaubt eine saubere Entkopplung von Logik und User-Interface, was das Projekt auch bei einer wachsenden Anzahl von Codezeilen übersichtlich und verwaltbar gestaltet.

Dieses Kapitel entspricht der dritten Phase des \emph{Human-Centered Design Process}, und verfolgt das Ziel, einen ersten benutzbaren Prototyp zu schaffen, der die in Kapitel (Problematik) beschriebenen Probleme möglichst gut löst. Anschließend soll dieser erste Prototyp bzw. die Android-Library, die bei der Entwicklung des Prototyps entsteht, in die vorhandende Android-Applikation integriert werden, und bereits im Arbeitsalltag getestet werden.
Dies ermöglicht das Identifizieren von eventuell auftretenden Usability-Problemen, die während der initialen Beobachtung noch nicht aufgefallen sind.

Alle Erkenntnisse aus dieser ersten Test-Phase werden anschließend mit in die zweite Iteration des Zyklus genommen, um diese in einem verbesserten Prototyp zu lösen.

Der Ziel dieses Kapitels ist, einen funktionierenden Prototyp in Form einer Android-Library zu entwicklen, welcher sich in die bestehende Android-Applikation integrieren lässt und die Bewertungskriterien aus Kapitel (Evaluation) erfüllt.

\section{Implementation}\label{sec:pro1}
In \autoref{chap:concept} wurden Umsetzungsmöglichkeiten für die verschiedenen Bewertungskriterien aus \todo{ref auf sec} herausgearbeitet.
Diese sollen in die Implementierung des ersten Prototyps einfließen, und so einen Grundstein für eine positive Benutzererfahrung der App legen. \\

Der erste Prototyp wurde am 16. Dezember 2017 in Form fertiggestellt und in die bestehende App eingebunden.
Der Prototyp an sich wurde als \emph{Android-Library} programmiert, um den Quellcode möglichst übersichtlich und unabhängig von der bestehenden App schreiben zu können.
Die Implementierung als \emph{Android-Library} erlaubt zudem kürzerer Kompilierungszeiten und eine einfachere Einbindung in bestehende Android-Projekte. \\
\todo{jitpack nenne?}

Bei der Implementierung wurde das Entwurfsmuster des \emph{Model-View-Controllers} eingesetzt, welches den Quellcode in drei verschiedene Komponente unterteilt (siehe \autoref{fig:mvc}).
So gibt es einerseits das \emph{Datenmodell (model)}, die \emph{Präsentationskomponente (view)} und die \emph{Programmsteuerung (controller)}.
Ziel des Entwurfsmusters ist es, eine flexible Architektur zu schaffen, die bei Bedarf leicht erweitert bzw. wiederverwendet werden kann.

\begin{figure}[h]
  \centering
  \includegraphics[keepaspectratio, width=0.5\textwidth]{mvc}
  \caption{Interaktion innerhalb des Model-View-Controller Prinzips}
  \label{fig:mvc}
\end{figure}

\noindent
Zusätzlich wurde der Aufbau des Projekts in der \emph{Unified Modelling Language}, kurz \emph{UML}, modelliert.
Dies ermöglicht bereits vor der eigentlichen Implementierung wichtige Begriffe und mögliche Beziehungen festzulegen, und einen Überblick über die benötigten Klassen zu bekommen.

\begin{figure}[h]
  \centering
  \includegraphics[keepaspectratio, width=\textwidth]{prototype1/model}
  \caption{Datenmodell-Komponente als UML-Diagramm}
  \label{fig:model1}
\end{figure}

\noindent
Die Komponente des \emph{Datenmodells} (siehe \autoref{fig:model1}) sah dabei wie folgt aus (die beiden weiteren Komponenten befinden sich im Anhang):
In \autoref{fig:model1} ist zentral die abstrakte Klasse \emph{MeasureShape} zu erkennen, welche die Oberklasse der beiden Formen \emph{Line} und \emph{Rectangle} darstellt.
\emph{MeasureShape} vererbt die abstrakten Attribute \emph{points} und \emph{edges} an ihre Unterklassen, die diese Attribute in überschreiben müssen.
Außerdem muss die öffentliche Methode \emph{drawShape(...)} von beiden Unterklassen implementiert werden.
Dies sorgt dafür, dass jede Unterklasse selber dafür ``verantwortlich'' ist, wie und wohin ihre Form gezeichnet wird.
\emph{MeasureShape} selber implementiert die drei verschiedene \emph{Interfaces} \emph{Draggable, Resizable und Editable}, welche Schnittstellen zum Verschieben, Vergrößern und Editieren von Formen bereit stellen. \\
\todo{hier nur MeasureShape zeigen und Rest in Anhang}

\noindent
Eine weitere Klasse im \emph{Datenmodell} ist \emph{UserAction}.
Diese ist eine versiegelte (\emph{sealed}) Oberklasse, welche die verschiedenen Benutzeraktionen darstellt.
Versiegelt bedeutet in diesem Kontext, dass nur Klassen, welche im \emph{Scope} der Oberklasse liegen, von dieser erben können.
Folgende sechs Benutzeraktionen können in der Implementierung des ersten Prototyps über die Undo/Redo-Funktion rückgängig gemacht oder wiederhergestellt werden:

\begin{itemize}
  \item Verschieben von Formen (\emph{DraggedShape})
  \item Hinzufügen von Formen (\emph{AddedShape})
  \item Löschen von Formen (\emph{RemovedShape})
  \item Vergrößern bzw. verkleinern von Formen (\emph{ResizedPoint})
  \item Ändern des Textes (\emph{ChangedText})
  \item Ändern der Farbe (\emph{ChangedColor})
\end{itemize}

\noindent
Funktional beschränkt sich die Implementierung des ersten Prototyps zunächst auf das Zeichnen von einfachen Linien und Vierecken, sowie das anschließende Beschriften dieser, um Messwerte einzutragen.
Zum schnellen und präzisen Zeichnen der Formen wird eine Zoom-Linse (siehe \autoref{fig:draw1}), wie sie in allen drei Apps aus \autoref{chap:eval} umgesetzt wurde, verwendet.
Der Prototyp verfügt über zwei verschiedene Modi, den Zeichen- und Text-Modus, zwischen denen mit Hilfe des \emph{Floating Action Buttons} im unteren rechten Bildschirmbereich umgeschaltet werden kann (siehe \autoref{fig:all1}).
Zudem kann der Benutzer über einen weiteren \emph{Floating Action Button} im unteren linken Bildschirmbereich jederzeit ein neues Bild aufnehmen, oder ein bereits vorhandenen in die App importieren. \\

Undo- sowie Redo-Funktion befinden auf dedizierten \emph{Buttons} in der Menüleiste der App (siehe \autoref{fig:all1}).
Hier gibt es außerdem jeweils einen Button, um ausgewählte Formen zu löschen, die Zeichenfarbe zu ändern, oder eine andere Form zum Zeichnen auszuwählen. \\

\begin{figure}[h]
  \begin{subfigure}[t]{0.4\textwidth}
    \includegraphics[keepaspectratio, width=\textwidth]{prototype1/all}
    \caption{Erster Prototyp mit bereits eingezeichneter Linie}
    \label{fig:all1}
  \end{subfigure}
  \begin{subfigure}[t]{0.4\textwidth}
    \includegraphics[keepaspectratio, width=\textwidth]{prototype1/drawing}
    \caption{Zoom-Linse beim Zeichnen einen Vierecks}
    \label{fig:draw1}
  \end{subfigure}
  \centering
  \caption{Erster Prototyp bei eingezeichneter Linie und beim Zeichnen eines Vierecks}
\end{figure}

\noindent
Beim Auswählen des Farb-Icons in der Menüleiste öffnet sich ein modaler Dialog, der es dem Benutzer ermöglicht, die gewünschte Farbe auszuwählen (siehe \autoref{fig:color1}).
Diese Farbe wird als Standardfarbe für alle neuen Formen genutzt.
Falls vor dem Öffnen des Dialogs eine Form markiert wurde, wird diese ebenfalls mit der ausgewählten Farbe eingefärbt. \\

\begin{wrapfigure}{R}{0.5\textwidth}
  \centering
  \includegraphics[keepaspectratio, width=0.4\textwidth]{prototype1/color}
  \caption{Geöffneter Farbauswahl-Dialog}
  \label{fig:color1}
\end{wrapfigure}

Graue Indikatoren an den Eckpunkten der aktuell ausgewählte Form sollen dem Nutzer verdeutlichen, dass diese mit Hilfe der Indikatoren in ihrer Größe und Position bearbeitet werden kann.
Um Formen zu beschriften bietet sich dem Benutzer im Text-Modus die Möglichkeit, Kanten mittels eines Eingabe-Dialogs zu annotieren (siehe \autoref{fig:labeling1}).
Hierzu öffnet sich beim langen Klick auf die Kante einer Form im Text-Modus ein modaler Dialog, welcher die Messwerte des Nutzers entgegennimmt.
Eingetragene Messwerte werden anschließend neben der zuvor ausgewählten Kante im Bild dargestellt (siehe \autoref{fig:label1}). 
\todo{neues Bild wo Messwert besser sichtbar}

\begin{figure}[h]
  \begin{subfigure}[t]{0.4\textwidth}
    \includegraphics[keepaspectratio, width=\textwidth]{prototype1/labeling}
    \caption{Dialog zum Eintragen von Messwerten}
    \label{fig:labeling1}
  \end{subfigure}
  \begin{subfigure}[t]{0.4\textwidth}
    \includegraphics[keepaspectratio, width=\textwidth]{prototype1/label}
    \caption{Linie mit eingetragenem Messwert}
    \label{fig:label1}
  \end{subfigure}
  \centering
  \caption{Eintragen und Anzeigen von Messwerten im ersten Prototyp}
\end{figure}

\section{Test}\label{sec:test1}
\todo{Vielleicht noch ein bisschen ausschmücken}
Zum Testen des Prototyps wurde dieser in die App der beiden Geschäftsführer der Fa. \vr{} (siehe \autoref{table:testers}) eingebunden.
Testperson 1, André Vermeulen, repräsentiert hierbei den ``Poweruser'' der App, wohingegen Testperson 2, Sebastian Wiesbrock, die App eher selten nutzt.
Beiden Testern wurde weder eine verbale noch schriftliche Hilfestellung zur Verfügung gestellt.

\begin{table}[h]
  \centering
  \begin{tabular}{l | c | c}
    \hline
    \textbf{Testperson} & \textbf{1} & \textbf{2} \\
    \hline
    \textbf{Name} & André Vermeulen & Sebastian Wiesbrock \\
    \hline
    \textbf{Rolle} & Geschäftsleitung & Geschäftsleitung \\
    \hline
    \textbf{Alter} & ? & ? \\
    \hline
    \textbf{Nutzungsgrad der App} & hoch & niedrig \\
    \hline
  \end{tabular}
  \caption{Demografische Daten der Testpersonen}
  \label{table:testers}
\end{table}

\todo{Zeit kontrollieren}
Nachdem die beiden Tester den Prototyp zwei Tage in ihren Arbeitsalltag integriert haben, wurden am 18. Dezember die Testergebnisse gesammelt. \\

Nach Aussage beider Testpersonen, haben die \emph{Floating Action Buttons} sich bei der Benutzung der App als großes Hindernis herausgestellt.
So sei nicht intuitiv klar, dass die App über zwei verschiedene Modi, nämlich den Zeichen- und Text-Modus, verfügt.
Zudem sei unklar, dass man über einen Klick auf den rechten \emph{Floating Action Button} zwischen den beiden Modi hin und her wechseln kann. \\

Ein weiteres Problem ergab sich laut Testpersonen 1 bei der Benutzung der App auf seinem Tablet.
So seien sämtliche Texte nur schwer lesbar, und die Punkte zum Verändern der Formen so klein, dass sie nur mühsam und mit viel Konzentration mit dem Finger zu treffen seien. \\

Darüber hinaus schilderten beide Testpersonen, dass eingetragene Messwerte auf Bildern, die in dunkleren Lichtverhältnissen aufgenommen wurden, nur schlecht lesbar seien.
Hier setze sich die Textfarbe zu schlecht vom Hintergrund ab. \\

Zusätzlich zu den identifizierten Usability-Problemen kam bei beiden Testern der Wunsch nach neuen Funktionen auf, die sie sich als nützliche Erweiterungen hinsichtlich ihres Arbeitsalltags vorstellen konnten.
So wünschten sich beide Testpersonen einerseits die Möglichkeit, Formen mit bereits vorhandenen Gerüsttypen wie zum Beispiel ``Hängegerüst'' oder ``Standgerüst'' zu verbinden, und in den Meta-Daten des Bildes zu speichern.
Andererseits wurde angemerkt, dass es sinnvoll sei, Bilder vor dem Bearbeiten über eine weitere Oberfläche zunächst in die gewünschte Größe und Form schneiden zu können, da es oftmals auf der Baustelle vorkomme, dass die aufgenommenen Bilder nicht nur das gewünschte Gerüst, sondern auch andere Objekte, die für das Bild nicht relevant sind, beinhalten. \\

Die Ursachen und mögliche Lösungsideen der Probleme, die während dieser Testphase identifiziert wurden, sollen im nächsten Abschnitt in einer weiteren Iteration des \hcdp{} ausgewertet und mit Hilfe eines zweiten Prototyps gelöst werden.

\section{Zweite Iteration}\label{sec:pro2}

In \autoref{sec:pro1} haben sich während der Testphase des ersten Prototyps sowohl neue Probleme als auch fehlende Funktionalitäten identifizieren lassen.
Diese sollen bei der Entwicklungs eines zweiten Prototyps in diesem Abschnitt berücksichtigt werden. \\

\subsection{Observation}
Um einen Überblick über die Test-Ergebnisse aus dem vorherigen Abschnitt herzustellen und eventuelle Ursachen zu identifizieren, werden diese im Nachfolgenden zunächst aufgelistet und anschließend weiter ausgewertet:

\begin{enumerate}
  \item Systemzustand nicht intuitiv erkennbar (\emph{Floating Action Buttons})
  \item Text- und Formelemente auf Tablets nicht gut lesbar bzw. benutzbar
  \item Eingetragenen Messwerte auf dunekelem Hintergrund nicht gut erkennbar
  \item Zuordnen von Gerüsttypen zu Formen nicht möglich
  \item Zuschneiden und Rotieren des Bildes nicht möglich
\end{enumerate}

\noindent
Zunächst aufgefallen ist, dass beide Testpersonen in \autoref{subsec:test1} die \emph{Floating Actions Buttons} als Hindernis bei der Benutzung der App beschrieben haben.
Dadurch, dass die \emph{Buttons} zu jeder Zeit immer nur ein Icon anzeigen, und erst beim Klick auf diese erkennbar wird, welche weiteren Aktionen möglich sind, wird dem Nutzer nur wenig bis keine Rückmeldung über den aktuellen Systemzustand und ausführbare Aktionen gegeben.
Dies ist ein Verstoß gegen Nielsen-Heuristik \autoref{itm:N1} und spiegelt sich direkt in einer negativen Nutzungserfahrung der App wieder. \\

Ein weiterer Punkt, der in \autoref{subsec:test1} negativ aufgefallen ist, ist die Benutzung auf einem Tablet-Gerät.
So ist die Nutzung der App auf Endgeräten mit einer höheren Pixeldichte mühsam, und eine Quelle für Fehler.
Da Text- und Formelemente, die auf kleineren Bildschirm ohne Probleme zu erkennen und bedienen sind, auf größeren Bildschirmen nicht skaliert werden, sind diese bei einer höheren Pixeldichte nur mühsam erkennbar, und sorgen so für eine negative Benutzererfahrung und bieten gleichzeitig eine Quelle für potentielle Fehler. \\ 
\todo{Nielsen reffen}

Die schlechte Lesbarkeit von eingetragenen Messwerten bei Bildern mit einem dunkelen Hintergrund hängt vermutlich damit zusammen, dass die Textfarbe unabhängig von den Bildeigenschaften auf Grau festgelegt ist.
So führt die graue Textfarbe in Kombination mit einem fehlenden Texthintergrund, besonderns bei Bildern dunkelerem Hintergrung, zur schlechten Lesbarkeit der Messwerte. \\
\todo{Nielsen reffen}

Beim Testen der App ergab sich für beide Testpersonen außerdem das Problem, dass diese nach Funktionen gesucht haben, die nicht in der App vorhanden sind.
\todo{Wie fehlende Funktionen hier beschreiben?}
nach einer Funktion gesucht wurde, um eingetragene Formen mit bereits vorhandenen Gerüsttypen zu verknüpfen, oder Bilder beim Import in die gewünschte Größe und Form zu bringen.
Das Fehlen dieser Funktionen hat sich negativ auf das Benutzungserlebnis ausgewirkt, da die Benutzer fest davon ausgegangen sind, dass diese in der App integriert seien.

\subsection{Idea Generation}\label{subsec:idea2}
\todo{refs auf papers/guidelines wie bei Konzeption}
Um dem Benutzer jederzeit eine klare und einfache Rückmeldung über den aktuellen Systemzustand und die darin ausführbaren Aktionen zu geben, bietet es sich an, eine Art Statusleiste am unteren Bildschirmrand zu verankern.
Diese soll mit Hilfe von unterscheidbaren, aber intuitiv verständlichen Icons, über den aktuellen Modus (Zeichnen oder Text) informieren, und zugleich nicht benutzbare Aktionen nicht auswählbar gestalten. \\

Für eine bessere und einfachere Benutzung der App auf Tablet-Geräten bietet es sich einerseits an, eine zweite Benutzeroberfläche, die nur bei Geräten mit einer Displaybreite von beispielsweise mindestens $600$ Pixeln benutzt wird, einzubinden. 
Andererseits besteht die Möglichkeit, die Größe diverser Text- und Formelemente mit der Bildschirmgröße zu skalieren.
Dies hat den Effekt, dass Bildschirm-Elemente bei verschiedenen Bildschirmgrößen ungefähr gleich groß sind. \\

Die schlechte Erkennbarkeit von eingetragenen Messwerten auf Bildern mit dunkelem Hintergrund lässt sich durch die Benutzung eines Texthintergrundes lösen.
So könnte, wie bei \emph{Toasts}, die im Android System seit API X vorhanden sind, ein dunkeler Hintergrund in Kombination mit einer hellen Textfarbe dazu genutzt werden, eingetragene Messwerte unabhängig von den Lichtverhältnissen im Bild gut erkennbar zu machen. \\
\todo{ref auf toast}

Um Formen direkt den passenden Gerüsttypen zuzuordnen, bietet sich ein modaler Dialog an, der zum Beispiel bei einem langen Klick auf die gewünschte Form angezeigt wird, und in einem \emph{Dropdown} vorhandene Gerüsttypen zur Auswahl anzeigt.
Falls ein langer Klick auf die Form zu unintuitiv ist, würde sich eine Option in der oben beschriebenen Statusleiste anbieten, die nur dann auswählbar ist, wenn eine Form markiert ist. \\

Das Schneiden und Rotieren von Bildern kann einerseits durch die Benutzung der in der Android API vorhandenen \emph{Crop-Activity} realisiert werden, andererseits würde sich auch die Benutzung einer dedizierten \emph{Android-Library} für das Schneiden von Bildern anbieten.
Ersteres bringt die Gefahr mit sich, dass die Verfügbarkeit einer solchen \emph{Crop-Activity} im Android-System vom Gerätehersteller und der verwendeten Android-Version abhängig ist, sodass die Funktion nicht auf allen Geräten funktioieren wird. \todo{ref auf API docs}

\subsection{Prototyping}

Der zweite Prototyp wurde am 3. Januar 2018 fertiggestellt, und umfasst die Implementierung der in \autoref{subsec:idea2} gesammelten Ideen. \\

So wurde die Statusleiste durch eine \emph{Bottom-Bar} am unteren Bildschirmrand umgesetzt (siehe \autoref{fig:bar2}).

\begin{figure}[ht]
  \begin{subfigure}[t]{0.4\textwidth}
    \includegraphics[keepaspectratio, width=\textwidth]{prototype2/expanded_mode}
    \caption{Statusleiste im Zeichen-Modus mit Popup-Dialog zur Auswahl des Modus}
    \label{fig:mode2}
  \end{subfigure}
  \begin{subfigure}[t]{0.4\textwidth}
    \includegraphics[keepaspectratio, width=\textwidth]{prototype2/label_popup}
    \caption{Statusleiste im Text-Modus mit Popup-Dialog zur Beschriftung der ausgewählten Form}
    \label{fig:labelp2}
  \end{subfigure}
  \centering
  \caption{Bedienung der Statusleiste im zweiten Prototyp}
  \label{fig:bar2}
\end{figure}

\noindent
Diese besteht aus vier verschiedenen Icons, die nur dann auswählbar sind, wenn die entsprechende Aktion im aktuellen Systemzustand durchgeführt werden kann.
Das Icon ganz links ermöglicht das Wechseln zwischen dem Zeichen- und Textmodus (siehe \autoref{fig:mode2}). \\

Im Zeichenmodus kann per Klick auf das zweite Icon über ein \emph{Popup-Menü} die gewünschte Form ausgewählt werden. 
Im selben Modus kann beim Klick auf das dritte Icons (Farbpalette) die gewünschte Formfarbe im Voraus konfiguriert werden. Hierzu öffnet sich, wie schon beim ersten Prototyp, ein modaler Farbauswahl-Dialog.
In beiden Modi ermöglicht das vierte Icon (Mülleimer) das Löschen von ausgewählten Formen bzw. Texten. \\

Im Textmodus kann beim Klick auf das zweite Icon über ein \emph{Popup-Menü} entweder eine ausgewählte Form mit einer Kantenbeschriftung versehen, oder mit einem Gerüsttyp verknüpft werden (siehe \autoref{fig:labelp2}).
Das dritte Icon ermöglicht das Bearbeiten von bereits eingetragenen Messwerten und verknüpften Gerüsttypen.
Auch in diesem Modus sind die Icons nur dann auswählbar, wenn der aktuelle Systemzustand dies zulässt.
So sind das dritte und vierte Icons beispielsweise nur dann benutzbar, wenn zuvor eine Form ausgewählt worden ist. \\

Für eine einfache und fehlerfreie Benutzung auf Tablet-Geräten wurden sämtliche Größen mit Hilfe von dichteunabhängigen Pixeln modelliert, wie sie in den \emph{Android-Developer Guides} vorgeschlagen werden \citep{DP18}.
So werden hierbei alle Maße in der Einheit \emph{dp} angegeben, welche zur Laufzeit vom System in normale Pixel (\emph{px}) umgewandelt werden.
Die genaue Umformung lautet dabei wie folgt:
$$
px =  dp \times (\frac{dpi}{160})
$$

wobei $dp$ für ``dots per inch'' steht.
Dies stellt sicher, dass auf Geräten mit einer hohen Pixeldichte Elemente nicht zu klein dargestellt werden. \\
\todo{Bild vorher nachher Resize Points}

Messwerte sind mit einem grau-gefärbten Rechteck hinterlegt, welches dafür sorgt, dass sich die Texte besser vom Hintergrund abheben, und so auch bei schwierigen Bedingungen klar lesbar sind.  \\
\todo{Bild vorher nachher}

Das Zuordnen von Gerüsttypen zu gezeichneten Formen wurde mit Hilfe eines modalen Dialogs umgesetzt, der neben dem Gerüsttyp auch noch Textfelder für die verschiedenen Dimensionen des Gerüsts besitzt.
Hierdurch kann der Benutzer nicht nur den Gerüsttyp, sondern auch Maße des Gerüsts, welche im Bild aufgrund des Aufnahmewinkels eventuell nicht zu sehen sind, eintragen und in den Meta-Daten speichern. \\
\todo{Bild von Dialog}

Für das Schneiden und Rotieren von Bildern vor dem Annotieren wurde \emph{uCrop}, eine dedizierte Android-Library, welche auf \emph{Github} als Open-Source Projekt unter der \emph{Apache License Version 2.0} vorhanden ist, in das Projekt eingebunden \citep{UC18}.
Diese bietet im Gegensatz zu der \emph{Crop-Activity}, welche von manchen Android-Versionen zur Verfügung gestellt wird, Unterstützung für alle Gerätehersteller ab der Android-Version $14$ an.
Außerdem erlaubt diese Library das Anpassen sämtlicher Farben der Benutzeroberfläche, sodass Konsistenz beim Einbinden in den Prototyp bestehen bleibt, und der Nutzer nicht von zwei völlig verschiedenen Farbpaletten überrascht wird. \todo{Bilder}

\subsection{Testing}\label{subsec:test2}
Der zweite Prototyp wurde sechs Tage lang, bis zum 9. Januar 2018, von den beiden Geschäftsführern in ihrem Arbeitsalltag getestet.
Das anschließende Feedback ergibt sich aus einem Gespräch am 9. Januar. \\

Als deutlich positive Verbesserung wurde dabei von beiden Testpersonen die neue Statusleiste am unteren Bildschirmrand genannt.
Hierdurch sei die Benutzung der App um einiges leichter gefallen, als beim ersten Prototyp mit den \emph{Floating Action Buttons}. \\

Jedoch sei, so die beiden Tester, das initiale Einarbeiten in die App immer noch zu schwierig, und nicht intuitiv genug.
Hier fehlte beiden Testpersonen eine Hilfestellung, die beim Start der App die wichtigsten Funktionen zusammenfasst, und kurz erklärt wozu was benutzt werden kann. \\

Ein weiteres Problem, dass beim Testen des Prototyps aufgefallen sei, ist der Farbdialog.
Dieser sei nach Aussage einer Testperson, zu fortgeschritten und biete eine Auswahl an Farben, die ``[...] der normale Gerüstbauer niemals verwendet wird'' \todo{Zitat hier?}
Dies ist ein Problem, welches während der Test-Phase zum ersten Prototyp nicht als solches identifiziert wurde, sich jetzt aber doch als Problem herausgestellt hat.
Hierbei wäre es laut Testpersonen nämlich sinnvoller, den Benutzer ``[...] nicht mit so vielen Auswahlmöglichkeiten zu überfordern [...]'', sondern eine übersichtliche Menge an häufig benutzten Farben direkt auswählbar zu machen. \\

Außerdem wurde sich neben einem einfacheren Farbdialog eine Funktion gewünscht, Freitexte ins Bild einzutragen zu können. 
Dies sei laut der Aussage beider Testpersonen ein wichtiger Aspekt, da beim bisherigen Erstellen der Aufmaße oftmals weitere Notizen oder Kommentare auf Skizzen eingetragen werden, um besondere Punkte bzw. Abmachungen festzuhalten. \\

Zudem fehle die Möglichkeit, Linien mit nur einer Pfeilspitze zu zeichnen.
Dies sei laut beider Testpersonen wichtig, um Längen, die auf dem Bild nur einen Startpunkt haben, und in die Tiefe offen sind, zu kennzeichnen. \\

Des Weiteren seien verlinkte Gerüsttypen an Formen nicht intuitiv durch den Indikator, wie er in dem ersten Prototyp umgesetzt wurde, erkennbar.
Zusätzlich wurde angemerkt, dass die Textfelder im Dialog zum Verlinken des Gerüsttyps durchaus sinnvoll seien, aber nur selten genutzt wurden, da sich Messwerte gemerkt werden müssen, um diese anschließend in den Dialog einzutragen. \\

\section{Dritte Iteration}
Die beim Testen des zweiten Prototyps identifizierten Probleme sollen nun in einer weiteren Iteration des Human-Centered Design Process gelöst werden.
Hierzu werden wie schon im vorherigen Abschnitt die vier Phasen des Zyklus durchlaufen, und am Ende festgehalten, ob ein weitere Iteration des Zyklus notwending ist, um eventuelle neue Usability-Probleme zu lösen.
\subsection{Observation}\label{subsec:prob3}

Um auch in diesem Abschnitt einen Überblick über die gesammelten Test-Ergebnisse zu bekommen, werden die diese im Nachfolgenden zur Übersicht aufgelistet, und in den folgenden Unterabschnitten weiter evaluiert:

\begin{enumerate}
  \item Fehlende Hilfestellung beim initialen Start der App
  \item Überforderung bei Benutzung des Farbdialogs
  \item Einführen einer Freitext-Form
  \item Einführen einer Pfeil-Form 
  \item Wiedererkennbarkeit des Gerüsttyp-Indikators
  \item Gedächtnisbelastung beim Eintragen der Messwerte in Gerüsttyp-Dialog
\end{enumerate}

\subsection{Idea Generation}\label{subsec:idea3}
Um dem Benutzer beim Start der App einen Überblick über die beiden Modi und den darin enthalteten Funktionen zu geben, soll dem Nutzer eine Hilfestellung beim initialen Start angezeigt werden.
Dafür bietet sich entweder ein \emph{Onboarding-Pager} an, der den Benutzer mit Hilfe eines Fullscreen-Dialogs die verschiedenen Funktionen der App vor dem eigentlich Start der App zeigt, oder das Verwenden verschiedener Hilfe-Overlays, welche erst dann angezeigt werden, sobald der Benutzer diese braucht.
Hierbei könnte man zum Beispiel beim ersten Wechsel in den jeweiligen Modus eine Overlay anzeigen, welches die verschiedenen Aktionen in der Statusleiste erklärt. \\

Der Farbdialog kann in einem einfacheren Modus angezeigt werden, der standardmäßig nur eine kleine Menge voreingestellte Farben anzeigt, und bei Bedarf zu einem vollständigen Farbkreis erweitert werden kann. \\

Sowohl die Freitext-Form als auch die Pfeil-Form können mit Hilfe der bestehenden abstrakten Oberklasse \emph{MeasureShape} modelliert werden.
Wichtig bei der Freitext-Form wird es, wie sich in Abschnitt \todo{ref auf 2. proto} gezeigt hat, einen Hintergrund zu verwenden, damit der Text jederzeit lesbar ist. \\

Um den Indikator, dass Formen mit einem Gerüsttyp verknüpft sind, deutlicher und besser wiedererkennbar zu gestalten, könnte man ein Icon benutzen, welches neben der Indikator-Zahl angezeigt wird.
Alternativ könnte man auch einen weiteren Text hinzufügen, der die Verlinkung anzeigt.
Hierbei muss man jedoch bei der Implementierung darauf achten, dass man - besonders bei kleineren Geräten - nur eine begrenzte Menge Text auf dem Bildschirm gleichzeitig anzeigen sollte bevor dieser zu unübersichtlich wird. \\

Der Gerüsttyp-Dialog kann durch die Verwendung von Vorschlägen, die aus den eingetragenen Messwerten stammen, bei der Eingabe der Maße vereinfacht werden. \\

\subsection{Prototyping}
Der dritte Prototyp wurde am 16. Januar 2018 in die bestehende Android-App eingebunden.
Bei der Implementierung dieses Prototyps wurde versucht alle in \autoref{subsec:idea3} gennanten Ideen zur Lösung der Probleme aus \autoref{subsec:prob3} zu lösen.

So wurden drei Hilfe-Overlays, welche beim initialen Start, beim ersten Wechsel in den Textmodus, und beim ersten Wechsel in den Zeichenmodus implementiert.
Diese sollen dem Benutzer durch ihre Position und ihre kurzen erklärenden Text eine kurze, aber präzise Hilfestellung bei der Benutzung der App geben.
\todo{schreiben warum kein Onboarding screen}

Die verwendete ColorPicker-Library ermöglicht das Benutzen eines \emph{Preset-Mode}, welcher genau das gewünschte Verhalten, wie in \autoref{subsec:idea3} beschrieben, umsetzt.
Um den Dialog um einen weiteren \emph{Abbrechen-Button} zu ergänzen, musste die Library \emph{geforked} und anschließend musste dieser \emph{Fork} um den Button erweitert werden, da die Library an sich, eine solche Modifikation nicht vorgesehen hat. \todo{Fork erklären und Quelle}

Beide neuen Formen sind durch zwei Klassen, welche von der abstrakten Oberklasse \emph{MeasureShape} erben, modelliert und umgesetzt worden.
Dieser Prozess war für das Hinzufügen der Pfeil-Form trivial, da diese nahezu identisch zu der bereits vorhandenen Linien-Form ist.
Bei der Freitext-Form gab es jedoch ein paar Besonderheiten, die zu beachten waren.
So werden diese nicht wie die anderen Formen mit Hilfe einer Zeichen-Geste auf das Bild gezeichnet, sondern sollen beim Hinzufügen in der Mitte des derzeitigen Sichtbereichs erscheinen.
Außerdem muss die Größe einer Textform abhängig vom eingegebenen Text dynamisch berechnet, und beim Verändern erneut angepasst werden. \\

Um mit einem Gerüsttyp verlinkte Formen deutlicher zu kennzeichnen, wurde ein Icon \todo{icon zeigen} hinter dem Indikatortext angehangen.
Dies soll dafür sorgen, dass man beim Sehen des Indikators mit dem Icon direkt den Gerüsttyp-Dialog assoziiert.
Außerdem wurde die Position des Indikators so verändert, dass dieser jetzt immer neben den eingetragenen Messwerten, falls diese vorhanden sind, oder sonst genau an deren Position angezeigt wird. \\

Der Gerüsttyp-Dialog wurde um Vorschläge bei den Textfeldern ergänzt, welche die eingetragenen Messwerte der Form anzeigen.
So hat der Nutzer direkt eine Übersicht über die zuvor eingetragenen Messwerte, und muss nicht zwischen Dialog und Bild hin und her wechseln. \\

\subsection{Testing}
Der dritte Prototyp war 8 Tage in den Arbeitsalltag integriert, bis es am 24. Januar 2018 Feedback zur Implementierung gab.

Hierbei hat sich herausgestellt, dass alle Punkte, die sich während der Observation dieses Abschnittes als Probleme identifizieren lassen haben, gelöst wurden.
Jedoch sind bei der Implementierung neuer Funktionen noch nicht bekannte Usability-Probleme aufgetregen.
So machen die Freitext-Formen im parktischen Gebrauch das Bild zu unübersichtlich, da es keine Möglichkeit gibt, diese zu verkleinern oder auszublenden. \\

Außerdem hat sich als weiterer Negativpunkt identifizieren lassen, dass der Speichern-Dialog, bei dem man einen Titel für das annotierte Bild eintragen muss, sich nur mit dem Beschreibungs-Dialog, der in der vorhandenen Android-App nach dem Speichern des Bildes angezeigt wird, das Schreiben des gleichen Textes nach sich zieht.
Dies ist ein Punkt der für den Benutzer sowohl nervig, als auch irritierend ist, da er ohne erkennbaren Grund dazu aufgefordert wird, an zwei verschiedenen Stellen den selben Text einzugeben. \\

\todo{ andere punkte in bb app }

\chapter{Vierte Iteration - Übersichtlichkeit}
Die während der dritten Iteration in \autoref{chap:pro3} identifizierten Usability-Probleme und Anwenderwünsche sollen in diesem Kapitel weiter evaluiert werden.
Hierzu wird eine weitere Iteration des \hcdp{} ausgeführt.

\section{Observation}
Um die in \autoref{sec:test3} beschriebenen Probleme hinsichtlich ihrer Entstehung genauer zu untersuchen, werden diese zunächst aufgelistet und anschließend weiter beleuchtet.

\begin{itemize}
  \item Verwendung mehrerer Freitext-Formen im Bild zu unübersichtlich
  \item Doppelte Eingabe des Bildtitles beim Speichern irritierend
  \item Ungewollte Rotation von importieren Bildern
\end{itemize}

\noindent
Beide Testpersonen berichten in \autoref{sec:test3} von einer Unübersichtlichkeit bei der Verwendung von mehreren Text-Formen.
Dies hängt damit zusammen, dass die Schriftgröße des Textes dem aktuellen Zoom-Level der App angepasst wird.
So verkleinert sich der Text, wenn der Nutzer in das Bild hereinzoomt, und vergrößert sich wieder, sobald dieser herauszoomt.
Auf diese Weise soll sichergestellt werden, dass der Text unabhängig vom Zoom-Level jederzeit leicht lesbar ist.
Im herausgezoomten Zustand wird die Schriftgröße jedoch so stark vergrößert, dass die gesamte Freitext-Form zu viel Platz auf dem Bildschirm einnimmt.
Dies ist besonders auffällig, wenn der Nutzer mehrere Freitext-Formen erstellt, da diese anfangen sich zu überlappen und einen Großteil des Sichtbereiches zu verdecken.
(Nielsen~\autoref{itm:N12}) \\ 

Außerdem zeigen die Testergebnisse, dass es beim Speichern des bearbeiteten Bildes zu einer doppelten Eingabe des Titels kommt.
So muss der Benutzer einmal beim Speichern des Bildes innerhalb der \emph{Library} eine Beschreibung eingeben, und anschließend ein zweites Mal beim Hochladen des Bildes in der bestehenden App.
Diese doppelte Eingabe der exakt gleichen Beschreibung ist für den Benutzer nicht nachvollziehbar und sorgt darüber hinaus für unnötige Verwirrung.
(Nielsen~\autoref{itm:N1} \& \autoref{itm:N17}) \\

Zudem berichten beide Tester, dass sich im Hochformat aufgenommene Bilder ohne jegliche Interaktion beim Importieren in die App ins Querformat drehen.
Dieser Punkt ist beim lokalen Testen während der Implementierung des Prototyps auf keinem der benutzten Geräten aufgefallen.
Da beide Testpersonen ein \emph{Samsung}-Gerät verwenden, könnte es sich hierbei um einen Herstellerspezifischen Fehler handeln.
Auffällig ist auch, dass die beschriebene Rotation des Bildes nur dann geschieht, wenn das Bild beim Import weder skaliert noch zugeschnitten wurde.
Dieses Verhalten lässt sich im \emph{Quellcode} der \emph{uCrop-Library} verifizieren.
Hier wird vor dem Zuschneiden des Bildes geprüft, ob der Nutzer das Bild angepasst hat.
Ist dies nicht der Fall, so wird das Originalbild ohne jegliche Modifikation in die Zieldatei geschrieben.
Hierbei gehen auf \emph{Samsung}-Geräten die \emph{Exif}-Daten\urlnote{https://www.prophoto-online.de/digitalfotografie/exif-daten-10010057} verloren.
(Nielsen~\autoref{itm:N5}) \\ 

\section{Idea Generation}\label{sec:idea4}
Um das Bild auch bei der Verwendung von mehreren Freitext-Formen übersichtlich zu halten, bietet es sich an die Schriftgröße des Textes konfigurierbar zu machen.
Eine Alternative hierzu wäre es, die Formen aufklappbar zu gestalten, sodass der Nutzer selber entscheiden kann, welche Notiz zu welcher Zeit sichtbar sein soll, und welche zum Beispiel nur als \emph{Icon} auf dem Bild zu sehen ist. \\

Die doppelte Eingabe der Bildbeschreibung beim Speichern lässt sich durch das Einführen eines weiteren Parameters beim Starten der Library lösen.
So kann beim Benutzen der \emph{Library} über den zusätzliche Parameter entschieden werden, ob beim Speichern ein Speicher-Dialog innerhalb der \emph{Library} angezeigt werden soll oder nicht.
Hierdurch lässt sich der Speicher-Dialog für die bestehende App einfach ausblenden, und das doppelte Eintragen des Bildtitels verhindern. \\

Das ungewollte Rotieren von Bilder, die im Hochformat aufgenommen wurde, kann durch eine spezielle Überprüfung des Geräteherstellers verhindert werden.
Diese Überprüfung soll das Bild, wenn es sich um ein Geräte von \emph{Samsung} handelt, auch dann zuschneiden, wenn der Nutzer keine Modifikation am Bild vorgenommen hat.
Hierdurch soll sichergestellt werden, dass die \emph{Exif}-Daten nicht verloren gehen und das Bild nicht ungewollt rotiert wird.

\section{Prototyping}
Der vierte Prototyp wurde am 10. Februar 2018 ausgerollt und in die bestehende Android-App eingebunden, um die in \autoref{sec:idea4} genannten Lösungsideen umzusetzen. \\

Für eine bessere Übersichtlichkeit bei der Verwendung von mehreren Freitext-Formen gleichzeitig, wurde die Funktion hinzugefügt, dass Freitext-Formen über einen Doppelklick ein- bzw. aufgeklappt werden könne.
So werden diese im ``ausgeklappten'' Modus wie zuvor als Text mit Hintergrund angezeigt, beim Doppelklick transformiert sich diese Form jedoch zu einem kleinen Icon, welches keinerlei Text anzeigt, und so nur suggeriert , dass sich hinter dem Icon eine Freitext-Form befindet. 
Dies soll dem Benutzer die Möglichkeit geben, selber zu entscheiden, welcher Text wann auf dem Bildschirm sichtbar ist und welcher nicht. \\

\begin{figure}[h]
  \begin{subfigure}[t]{0.4\textwidth}
    \centering
    \includegraphics[keepaspectratio, width=\textwidth]{prototype4/text1}
    \caption{Zwei Freitext-Formen im dritten Prototyp}
  \end{subfigure}
  ~
  \begin{subfigure}[t]{0.4\textwidth}
    \centering
    \includegraphics[keepaspectratio, width=\textwidth]{prototype4/text2}
    \caption{Zwei Freitext-Formen im vierten Prototyp (eine minimiert)}
  \end{subfigure}
  \centering
  \caption{Vorher-Nachher-Vergleich: Anzeige zweier Freitext-Formen im dritten und vierten Prototyp}
  \label{fig:texts}
\end{figure}

Der Speicher-Dialog wurde durch das Hinzufügen eines weiteren Start-Parameters in der \emph{Library} ausgeblendet, und ist so bei der Benutzung in der bestehenden Android-App nicht mehr zu sehen.
Hier wird das Bild direkt auf dem Gerät gespeichert und anschließend in der bestehenden App nach einem Bildtitel gefragt. \\

Die ungewollte Rotation der Bilder auf \emph{Samsung}-Geräten wurde durch die Modifikation der \emph{uCrop-Library} umgesetzt.
Hierzu wurde ein \emph{Fork}\urlnote{https://help.github.com/articles/fork-a-repo/} der \emph{Library} erstellt und anschließend eine Überprüfung des Geräteherstellers im \emph{Quellcode} hinzugefügt.

\section{Testing}
Feedback zum Benutzererlebnis des vierten Prototyps gab es 7 Tage nach dem Roll-Out, am 17. Februar. \\

In dieser vierten Iteration des \hcdp{} sind keine weiteren Usability-Probleme identifiziert worden.
Der Prototyp ließe sich, so beide Testpersonen, intuitiv und einfach zum Erfassen der Aufmaße verwenden und würde auf langer Sicht viel Arbeit und Zeit sparen, da alle Informationen sofort digital und für alle Mitarbeiter zur Verfügung stehen. \\

Ein abschließender Anwenderwunsch bezieht sich auf das  Herunterladen bereits annotierter Bilder zum Bearbeiten auf einem anderen Gerät.
Da dieser Anwenderwunsch von der Android-App alleine nicht umgesetzt werden kann, wurde eine solche Funktionalität von der bestehenden \emph{API} am 17. Februar angefordert.
Genauer soll hier ein weiterer Endpunkt auf Seiten der \emph{API} definiert werden, welcher die hochgeladenen Meta-Daten des bearbeiteten Bildes formatiert als \emph{JSON}\urlnote{https://www.json.org/json-de.html} entgegennimmt.
Zusätzlich soll ein Endpunkt zum Bereitstellen des annotierten Bildes inklusive Meta-Daten implementiert werden. 
Auf diese Weise soll es dem Benutzer möglich sein, annotierte Bilder zu einem späteren Zeitpunkt auf dem gleichen oder einem anderen Gerät bearbeiten zu können.

