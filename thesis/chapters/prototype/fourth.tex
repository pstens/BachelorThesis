\chapter{Vierte Iteration}
Die während der dritten Iteration in \autoref{chap:pro3} identifizierten Usability-Probleme und Anwenderwünsche sollen in diesem Kapitel weiter evaluiert werden.
Hierzu wird eine weitere Iteration des \hcdp{} ausgeführt.

\section{Observation}
Um die in \autoref{sec:test3} beschriebenen Probleme hinsichtlich ihrer Entstehung genauer zu untersuchen, werden diese zunächst aufgelistet und anschließend weiter beleuchtet.

\begin{itemize}
  \item Verwendung mehrerer Freitext-Formen im Bild zu unübersichtlich
  \item Doppelte Eingabe des Bildtitles beim Speichern irritierend
\end{itemize}

\noindent
Beide Testpersonen berichten in \autoref{sec:test3} von einer Unübersichtlichkeit bei der Verwendung von mehreren Text-Formen.
Dies hängt wahrscheinlich damit zusammen, dass die Schriftgröße des Textes dem aktuellen Zoom-Level der App angepasst wird.
So verkleinert sich der Text, wenn der Nutzer in das Bild hereinzoomt, und vergrößert sich wieder, sobald dieser herauszoomt.
Auf diese Weise soll sichergestellt werden, dass der Text unabhängig vom Zoom-Level jederzeit leicht lesbar ist.
Im herausgezoomten Zustand wird die Schriftgröße jedoch so stark vergrößert, dass die gesamte Freitext-Form zu viel Platz auf dem Bildschirm einnimmt.
Dies ist besonders auffällig, wenn der Nutzer mehrere Freitext-Formen erstellt, da diese anfangen sich zu überlappen und einen Großteil des Sichtbereiches zu verdecken.
(Nielsen~\autoref{itm:N12}) \\ 

Außerdem zeigen die Testergebnisse, dass es beim Speichern des bearbeiteten Bildes zu einer doppelten Eingabe des Titels kommt.
So muss der Benutzer einmal beim Speichern des Bildes innerhalb der \emph{Library} eine Beschreibung eingeben, und anschließend ein zweites Mal beim Hochladen des Bildes in der bestehenden App.
Diese doppelte Eingabe der exakt gleichen Beschreibung ist für den Benutzer nicht nachvollziehbar und sorgt darüber hinaus für unnötige Verwirrung.
(Nielsen~\autoref{itm:N1} \& \autoref{itm:N17}) \\ 

\section{Idea Generation}\label{sec:idea4}
Um das Bild auch bei der Verwendung von mehreren Freitext-Formen übersichtlich zu halten, bietet es sich an die Schriftgröße des Textes konfigurierbar zu machen.
Eine Alternative hierzu wäre es, die Formen aufklappbar zu gestalten, sodass der Nutzer selber entscheiden kann, welche Notiz zu welcher Zeit sichtbar sein soll, und welche zum Beispiel nur als \emph{Icon} auf dem Bild zu sehen ist. \\
\todo{Collapsable Views?}

Die doppelte Eingabe der Bildbeschreibung beim Speichern lässt sich durch das Einführen eines weiteren Parameters beim Starten der Library lösen.
So kann beim Benutzen der \emph{Library} über den zusätzliche Parameter entschieden werden, ob beim Speichern ein Speicher-Dialog innerhalb der \emph{Library} angezeigt werden soll oder nicht.
Hierdurch lässt sich der Speicher-Dialog für die bestehende App einfach ausblenden, und das doppelte Eintragen des Bildtitels verhindern.
\todo{Konfigurierabare Libs?}

\section{Prototyping}
Der vierte Prototyp wurde am \todo{Release date nachschauen} ausgerollt und in die bestehende Android-App eingebunden, um die in \autoref{sec:idea4} genannten Lösungsideen umzusetzen. \\

Für eine bessere Übersichtlichkeit bei der Verwendung von mehreren Freitext-Formen gleichzeitig, wurde die Funktion hinzugefügt, dass Freitext-Formen über einen Doppelklick ein- bzw. aufgeklappt werden könne.
So werden diese im ``ausgeklappten'' Modus wie zuvor als Text mit Hintergrund angezeigt, beim Doppelklick transformiert sich diese Form jedoch zu einem kleinen Icon, welches keinerlei Text anzeigt, und so nur suggeriert , dass sich hinter dem Icon eine Freitext-Form befindet. 
Dies soll dem Benutzer die Möglichkeit geben, selber zu entscheiden, welcher Text wann auf dem Bildschirm sichtbar ist und welcher nicht. \\
\todo{Bild mit vielen Texten vorher nachher}

Der Speicher-Dialog wurde durch das Hinzufügen eines weiteren Start-Parameters in der \emph{Library} ausgeblendet, und ist so bei der Benutzung in der bestehenden Android-App nicht mehr zu sehen.
Hier wird das Bild direkt auf dem Gerät gespeichert und anschließend in der bestehenden App nach einem Bildtitel gefragt. \\

\section{Testing}
Feedback zum Benutzererlebnis des vierten Prototyps gab es X Tage nach dem Rollout. \\

In dieser vierten Iteration des \hcdp{} sind keine weiteren Usability-Probleme identifiziert worden.
Der Prototyp ließe sich, so beide Testpersonen, effizient und einfach zum Erfassen der Aufmaße verwenden und würde auf langer Sicht viel Arbeit und Zeit sparen, da alle Informationen sofort digital und für alle Mitarbeiter zur Verfügung stehen.
\todo{Hier abschließend Fazit?}
