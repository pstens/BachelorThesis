\section{Dritte Iteration}
Die beim Testen des zweiten Prototyps identifizierten Probleme sollen nun in einer weiteren Iteration des Human-Centered Design Process gelöst werden.
Hierzu werden wie schon im vorherigen Abschnitt die vier Phasen des Zyklus durchlaufen, und am Ende festgehalten, ob ein weitere Iteration des Zyklus notwending ist, um eventuelle neue Usability-Probleme zu lösen.
\subsection{Observation}\label{subsec:prob3}

Um auch in diesem Abschnitt einen Überblick über die gesammelten Test-Ergebnisse zu bekommen, werden die diese im Nachfolgenden zur Übersicht aufgelistet, und in den folgenden Unterabschnitten weiter evaluiert:

\begin{enumerate}
  \item Fehlende Hilfestellung beim initialen Start der App
  \item Überforderung bei Benutzung des Farbdialogs
  \item Einführen einer Freitext-Form
  \item Einführen einer Pfeil-Form 
  \item Wiedererkennbarkeit des Gerüsttyp-Indikators
  \item Gedächtnisbelastung beim Eintragen der Messwerte in Gerüsttyp-Dialog
\end{enumerate}

\subsection{Idea Generation}\label{subsec:idea3}
Um dem Benutzer beim Start der App einen Überblick über die beiden Modi und den darin enthalteten Funktionen zu geben, soll dem Nutzer eine Hilfestellung beim initialen Start angezeigt werden.
Dafür bietet sich entweder ein \emph{Onboarding-Pager} an, der den Benutzer mit Hilfe eines Fullscreen-Dialogs die verschiedenen Funktionen der App vor dem eigentlich Start der App zeigt, oder das Verwenden verschiedener Hilfe-Overlays, welche erst dann angezeigt werden, sobald der Benutzer diese braucht.
Hierbei könnte man zum Beispiel beim ersten Wechsel in den jeweiligen Modus eine Overlay anzeigen, welches die verschiedenen Aktionen in der Statusleiste erklärt. \\

Der Farbdialog kann in einem einfacheren Modus angezeigt werden, der standardmäßig nur eine kleine Menge voreingestellte Farben anzeigt, und bei Bedarf zu einem vollständigen Farbkreis erweitert werden kann. \\

Sowohl die Freitext-Form als auch die Pfeil-Form können mit Hilfe der bestehenden abstrakten Oberklasse \emph{MeasureShape} modelliert werden.
Wichtig bei der Freitext-Form wird es, wie sich in Abschnitt \todo{ref auf 2. proto} gezeigt hat, einen Hintergrund zu verwenden, damit der Text jederzeit lesbar ist. \\

Um den Indikator, dass Formen mit einem Gerüsttyp verknüpft sind, deutlicher und besser wiedererkennbar zu gestalten, könnte man ein Icon benutzen, welches neben der Indikator-Zahl angezeigt wird.
Alternativ könnte man auch einen weiteren Text hinzufügen, der die Verlinkung anzeigt.
Hierbei muss man jedoch bei der Implementierung darauf achten, dass man - besonders bei kleineren Geräten - nur eine begrenzte Menge Text auf dem Bildschirm gleichzeitig anzeigen sollte bevor dieser zu unübersichtlich wird. \\

Der Gerüsttyp-Dialog kann durch die Verwendung von Vorschlägen, die aus den eingetragenen Messwerten stammen, bei der Eingabe der Maße vereinfacht werden. \\

\subsection{Prototyping}
Der dritte Prototyp wurde am 16. Januar 2018 in die bestehende Android-App eingebunden.
Bei der Implementierung dieses Prototyps wurde versucht alle in \autoref{subsec:idea3} gennanten Ideen zur Lösung der Probleme aus \autoref{subsec:prob3} zu lösen.

So wurden drei Hilfe-Overlays, welche beim initialen Start, beim ersten Wechsel in den Textmodus, und beim ersten Wechsel in den Zeichenmodus implementiert.
Diese sollen dem Benutzer durch ihre Position und ihre kurzen erklärenden Text eine kurze, aber präzise Hilfestellung bei der Benutzung der App geben.
\todo{schreiben warum kein Onboarding screen}

Die verwendete ColorPicker-Library ermöglicht das Benutzen eines \emph{Preset-Mode}, welcher genau das gewünschte Verhalten, wie in \autoref{subsec:idea3} beschrieben, umsetzt.
Um den Dialog um einen weiteren \emph{Abbrechen-Button} zu ergänzen, musste die Library \emph{geforked} und anschließend musste dieser \emph{Fork} um den Button erweitert werden, da die Library an sich, eine solche Modifikation nicht vorgesehen hat. \todo{Fork erklären und Quelle}

Beide neuen Formen sind durch zwei Klassen, welche von der abstrakten Oberklasse \emph{MeasureShape} erben, modelliert und umgesetzt worden.
Dieser Prozess war für das Hinzufügen der Pfeil-Form trivial, da diese nahezu identisch zu der bereits vorhandenen Linien-Form ist.
Bei der Freitext-Form gab es jedoch ein paar Besonderheiten, die zu beachten waren.
So werden diese nicht wie die anderen Formen mit Hilfe einer Zeichen-Geste auf das Bild gezeichnet, sondern sollen beim Hinzufügen in der Mitte des derzeitigen Sichtbereichs erscheinen.
Außerdem muss die Größe einer Textform abhängig vom eingegebenen Text dynamisch berechnet, und beim Verändern erneut angepasst werden. \\

Um mit einem Gerüsttyp verlinkte Formen deutlicher zu kennzeichnen, wurde ein Icon \todo{icon zeigen} hinter dem Indikatortext angehangen.
Dies soll dafür sorgen, dass man beim Sehen des Indikators mit dem Icon direkt den Gerüsttyp-Dialog assoziiert.
Außerdem wurde die Position des Indikators so verändert, dass dieser jetzt immer neben den eingetragenen Messwerten, falls diese vorhanden sind, oder sonst genau an deren Position angezeigt wird. \\

Der Gerüsttyp-Dialog wurde um Vorschläge bei den Textfeldern ergänzt, welche die eingetragenen Messwerte der Form anzeigen.
So hat der Nutzer direkt eine Übersicht über die zuvor eingetragenen Messwerte, und muss nicht zwischen Dialog und Bild hin und her wechseln. \\

\subsection{Testing}
Der dritte Prototyp war 8 Tage in den Arbeitsalltag integriert, bis es am 24. Januar 2018 Feedback zur Implementierung gab.

Hierbei hat sich herausgestellt, dass alle Punkte, die sich während der Observation dieses Abschnittes als Probleme identifizieren lassen haben, gelöst wurden.
Jedoch sind bei der Implementierung neuer Funktionen noch nicht bekannte Usability-Probleme aufgetregen.
So machen die Freitext-Formen im parktischen Gebrauch das Bild zu unübersichtlich, da es keine Möglichkeit gibt, diese zu verkleinern oder auszublenden. \\

Außerdem hat sich als weiterer Negativpunkt identifizieren lassen, dass der Speichern-Dialog, bei dem man einen Titel für das annotierte Bild eintragen muss, sich nur mit dem Beschreibungs-Dialog, der in der vorhandenen Android-App nach dem Speichern des Bildes angezeigt wird, das Schreiben des gleichen Textes nach sich zieht.
Dies ist ein Punkt der für den Benutzer sowohl nervig, als auch irritierend ist, da er ohne erkennbaren Grund dazu aufgefordert wird, an zwei verschiedenen Stellen den selben Text einzugeben. \\

\todo{ andere punkte in bb app }
