\chapter{Forschungsmethodik}
Als Forschungsmethode wird in dieser Arbeit der \hcdp{} nach \citeauthor{Norman13} angewandt.
Diese Forschungsmethode wurde gewählt, da die Entwicklung nah am Anwender stattfindet und es sich um einen agilen Softwareentwicklungsprozess \todo{Fußnote?} handelt.
Zudem ist die Usability der entwickelten App wichtig, da die Endanwender nicht sehr IT-affin sind und die Aufmaßerfassung auf einem kleinem Display schwierig ist. \\

\citeauthor{Norman13} selbst definiert den Prozess in seinem Buch ``The Design of Everyday Things: Revised and Expanded Version'' wie folgt \citep[Abbildung 6.2]{Norman13}:

\begin{quote}
  ``Make observations on the intended target population, generate ideas, produce prototypes and test them.
  Repeat until satisfied.''
\end{quote}

\noindent
Die Idee des \hcdp{} ist es, eine ausgewählte Menge and Testpersonen aus der Zielgruppe in ihrem Alltag zu beobachten.
Hierbei sollen mögliche Usability-Probleme identifiziert und anschließend durch das Aufstellen verschiedener Ideen und die Entwicklung eines Prototyps gelöst werden.
Diesbezüglich beschreibt \citeauthor{Norman13} den Prozess als Zyklus, welcher sich aus vier verschiedenen Phasen zusammensetzt.
Dieser Zyklus wird iterativ so lange wiederholt, bis sich in der Testphase keine weiteren Usability-Probleme mehr identifizieren lassen oder man mit den bis dahin erzielten Ergebnissen zufrieden ist. \\

\begin{figure}[h]
  \centering
  \includegraphics[keepaspectratio]{hcp}
  \caption{The Human-Centered Design Process}
  \label{fig:hcp}
\end{figure}

\noindent
In \autoref{fig:hcp} sind die vier Phasen des Zyklus dargestellt:
\begin{enumerate}
  \item Observation (Beobachtung) \label{itm:observation}
  \item Idea Generation (Ideenfindung) \label{itm:idea}
  \item Prototyping (Entwicklung eines Prototyps) \label{itm:prototyping}
  \item Testing (Testen) \label{itm:testing}
\end{enumerate}

\noindent
In der ersten Phase des Zyklus (\emph{Observation}) wird eine Menge an ausgewählten Testpersonen bei der Bearbeitung von Aufgaben, die für das zu untersuchende Problemgebiet relevant sind, beobachtet.
Hierbei werden alle auftretenden Probleme notiert, und auf ihre eventuellen Ursachen untersucht.
Auf diese Weise ist es den Beobachtern möglich, ein tiefgründiges Verständnis für die angestrebten Ziele und den dabei anfallenden Problemen der Test-Personen zu erlangen.
Bezogen auf diese Arbeit findet die Phase der \emph{Observation} in Kapitel 3 und 4 statt. \\

In der zweiten Phase (\emph{Idea Generation}) werden Lösungsansätze zu den in der ersten Phase identifizierten Problemen aufgestellt.
Hierzu führt \citeauthor{Norman13} drei Richtlinien zur Orientierung an \citep[Seite 226]{Norman13}:

\begin{itemize}
  \item ``Generate numerous ideas'' (Generiere viele Ideen)
  \item ``Be creative without regard for constraints'' (Sei grenzenlos kreativ)
  \item ``Question everything'' (Hinterfrage alles)
\end{itemize}

\noindent
Zusammengefasst sagen diese drei Regeln aus, dass es während der \emph{Idea Generation} sinnvoll sei, viele kreative Lösungsansätze für die Usability-Probleme aufzustellen und zu hinterfragen.
Nach \citeauthor{Norman13} könne selbst eine anfangs nicht viel versprechende Idee zur Findung der finalen Lösung einen kleinen Anteil beitragen.
Die Phase der \emph{Idea Generation} findet in Kapitel 5, bei der Konzeption einer eigenen Android-App, statt. \\

Anschließend werden die Lösungsansätze in der dritten Phase (\emph{Prototyping}) in Form eines Prototyps umgesetzt.
Diese Phase korrespondiert zu Kapitel 6.1 dieser Arbeit. \\

In der vierten Phase des Zyklus (\emph{Testing}) wird der zuvor entwickelte Prototyp an ausgewählte Testpersonen verteilt.
Bei der Auswertung der Testergebnisse soll geprüft werden, ob ein Großteil der in Phase 1 identifizierten Probleme gelöst werden konnten.
Sollten keine weiteren Usability-Probleme während des \emph{Testing} auftreten, kann der \hcdp{} an dieser Stelle erfolgreich beendet werden.
Kommt es dennoch dazu, dass einzelne Probleme nicht optimal gelöst werden konnten oder neue Probleme identifiziert worden sind, wird eine weitere Iteration des Zyklus mit Hilfe der gewonnenen Testergebnisse gestartet.
Das Kapitel 6.2 beschreibt diese \emph{Testing} Phase.
Anschließend werden in den darauffolgenden Kapiteln 7, 8 und 9 drei weitere Iterationen, die nötig sind um einen optimalen Prototyp zu entwickeln, beschrieben. \\

Das Ziel dieses Entwicklungsprozesses ist es, den \hcdp{} so lange zu wiederholen, bis ein Prototyp entwickelt worden ist, der während der \emph{Testing}-Phase keine weiteren Usability-Probleme mehr aufweist.
