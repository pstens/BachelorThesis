\chapter{Probleme und Grenzen}
\section{Usability-Entscheidungen}
  Gab viele Stellen, wo es nicht genau >die< perfekte Lösung gab, sondern ein 
  Zusammenspiel verschiedener Faktoren dafür gesorgt hat, dass Feature X so geworden
  ist, wie es ist. \\
  
  Erst nach mehrmaligen Iterationen im Implementierungs/Testvorgang mit Testpersonen, 
  die aus der Domäne kommen, konnten alle Stolpersteine, die eine intuitive und 
  effiziente Benutzung der App ermöglichen beseitigt werden. \\
  
  Jedoch trotzdem ein Prozess, in der man nicht jede Testperson wunschlos glücklich machen kann,
  da jeder Mensch seine eigenen Gewohnheiten hat, sein Smartphone bzw. die mobilen Applikation darauf
  zu nutzen. \\

  Bei allen Usability-Verbersserungen sollte die eigentliche Aufgabe der Software-Lösung nicht in den
  Hintergrund rücken. Das effiziente Notieren bzw. Bearbeiten des Bildes sollte durch
  over-Engineering auf Seiten der UX nicht gefährdet sein. \\

  In Zukunft sogar den Prozess des Ausmessens per Augmented-Reality vorstellbar. Erste Schritte
  gibt es schon, aber meist nur unter Labor ähnlichen Voraussetzungen (gleiche Lichtverhältnisse, 
  guter Abstand zum Objekt, Objekt passt perfekt aufs Bild). Unter dem immer wechselnden Kontexts
  des Gerüstbaus ist dies zum jetzigen Zeitpunkt leider nur ein Traumgedanke. \\


