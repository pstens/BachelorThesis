\chapter{Probleme und Grenzen}
\todo{Section ausformulieren}
Während der gesamten Entwicklung des Prototyps haben sich verschiedene Usability-Probleme gezeigt, die in den einzelnen Zyklen des \hcdp{} so gut wie möglich gelöst werden sollten.
Nicht alle Probleme, die während der \emph{Testing}-Phasen identifiziert wurden, konnten mit Hilfe der Android-App gelöst werden.
In diesem Kapitel werden die Probleme und Grenzen der mobilen Bildbearbeitung zur Aufmaßerfassung im Gerüstbau näher beleuchtet.

\section{Hindernisse vor dem zu beaufmaßenden Objekt}
Während der Aufmaßerfassung des Gebäudes in \autoref{chap:prob} wurde dieses Problem besonders deutlich.
Hier verdeckte ein Baum eine gesamte Seite des Hauses (siehe \autoref{fig:tree}), sodass die Aufmaßerfassung mit Hilfe eines Bildes schwer umsetzbar war.
\todo{Bild von Andrè mit Baum vor Haus}
\todo{Was schreib ich hierzu noch alles?}

\section{Eingeschränkter Winkel zur Aufnahme des Bildes}

\todo{Was schreib ich hierzu noch alles?}

\section{Überladung des Bildes durch zu viele Formen und Texte}
-> Lösungsansatz durch Unterstützung verschiedener Ebenen
\todo{Was schreib ich hierzu noch alles?}
