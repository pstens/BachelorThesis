\chapter{Einleitung}
\section{Motivation}\label{subsec:motivation}
``Digitalisierung des Handwerks'' - Das ist der Titel einer im Jahr 2017 ausgeführten Studie vom Branchenverband \textit{Bitkom}. 
Diese Studie besagt, dass 69\% der 509 befragten Handwerksbetriebe die Digitalisierung als Chance für einen positive Umschwung sehen. 
Dabei stehen aus Sicht der Handwerker die Vorteile von ``optimaler Lagerung und Logistik'' (91\%), ``Zeitersparnis'' (81\%) und ``flexibler Arbeitsorganisation'' (78\%) im Vordergrund (Bitkom e.V., 2017). \\

Im November 2017 wurde passend dazu ein zweitägiges Groß-Seminar unter dem Titel ``Gerüstbau 4.0: Was bedeutet der digitale Wandel für den eigenen Betrieb?'' in Hannover vom Güterschutzverband Stahlgerüstbau e.V. veranstaltet. 
Mit Vorträgen zu Themen wie ``Geschäftsprozesse im Gerüstbau-Handwerk'', ``Virtuelle Projekträume und Cloudlösungen'' und ``Mobile Datenverwaltung'' wurden die Teilnehmenden über den digitalen Wandel informiert (Güteschutzverband Stahlgerüstbau e.V., kein Datum). \\

Der Gerüstbau-Dienstleister Fa. VERO Scaffolding EOOD Niederlassung Deutschland aus Paderborn (nachfolgend: \textsc{VERO}) kann nach der Einführung und Benutzung einer selbst entwickelten Android-Applikation im Mai 2014 die genannten Vorteile teilweise verzeichnen.
Die entwickelte Software-Lösung wird einerseits von den Monteuren zur Arbeitszeiterfassung, Einsicht von Auftragsinformationen, Baustellendokumentation mit Fotos und Text genutzt. Auf der anderen Seite nutzt die Geschäftsleitung bzw. Bauleitung vermehrt die administrativen Funktionen der App, wie die Verwaltung und Pflege von Kunden-, Projekt-, und Auftragsinformationen. \\

Die Aufmaßerfassung ist ein wichtiger Bestandteil als Vorbereitung zur Rechnungslegung im Handwerk, insbesondere im Gerüstbau. 
Hierbei werden die Maße (Länge, Breite und Höhe) der bereitgestellten Gerüste, idealerweise abgemessen, meistens jedoch abgeschätzt und später im Büro mit einer Preisliste nach Gerüsttyp (Fassadengerüst, Raumgerüst, Hängegerüst, ...) abgerechnet. \\

\todo{bilder von app}

\section{Umfeld}
Nach der Handwerkszählung aus dem Jahr 2015 sind 13\% (75.264) aller Handwerksunternehmen dem Bauhauptgewerbe zuzuordnen. Zu diesem Gewerbe zählt auch die Gerüstbaubranche, welche laut einer Statistik des Statistischen Bundesamtes im Jahr 2016 überwiegend von vielen kleinen Unternehmen dominiert wird (88,03\% der Unternehmen mit bis zu 20 Angestellten). 9,28\% der Unternehmen verzeichnen zwischen 21 und 50 Beschäftigte, und nur 2,69\% der Unternehmen verzeichnen über 50 Beschäftigte. (Statistisches Bundesamt, 2017) \\

Zu einem der kleineren Unternehmen gehört auch die Fa. VERO Scaffolding EOOD Niederlassung Deutschland in Paderborn mit 14 Beschäftigten, welche im Folgenden weiter vorgestellt wird.

\subsection{Unternehmen}
Das heutige Unternehmen VERO Scaffolding EOOD Niederlassung Deutschland geht aus im Jahr 1999 in Paderborn gegründeten VERO Gerüstbau GmbH hervor. 
Seit 2013 ist die VERO Gerüstbau GmbH die deutsche Niederlassung der bulgarischen VERO Scaffolding EOOD, ansässig in Stara Zagora. Der Standort in Bulgarien fokussiert sich hauptsächlich auf mehrjährige Großprojekte, wohingegen es in der deutschen Niederlassung in Paderborn viele kleinere Projekte für Kommunen, Automobilzulieferer, aber auch, wenn eher selten, private Haushalte handelt.
Die Belegschaft besteht aus zehn Monteuren, einem Auszubildenden, einer Bürofachkraft und zwei Geschäftsführern. Die Monteure sind zwischen 19 und 55 Jahren alt und erledigen in zwei bis vier Kolonnen bis zu vier Arbeitsaufträge am Tag.

\subsection{Android-App und Systemarchitektur}
Im Oktober 2013 begann der Entwicklungsstart einer Android-Applikation als mobile Software-Lösung für grundlegende Aufgaben der Monteure wie das Buchen von Arbeitszeiten, Ansehen von Arbeitsaufträgen und das Verfassen von schriftlichen Baudokumentationen. Die Applikation kommuniziert zur Synchronisation der getätigten Aktionen mit einer Server-Schnittstelle (REST-API) \todo{REST erklären}. Diese API ist wiederum an eine SQL-Datenbank, ein Dateisystem und externe Dienste gekoppelt, die es dem Unternehmen ermöglichen, die gesammelten Daten auch noch zu einem späteren Zeitpunkt anzupassen. \\

Zum jetzigen Stand (Januar 2018) ist die App um viele Funktionen erweitert. So gibt es nun beispielsweise die Möglichkeit neue Kunden, Projekte und Arbeitsaufträge direkt aus der App zu erstellen, aber auch eine Funktion, um seine Arbeit mit Fotos zu dokumentieren. 
\todo{eventuell ausführlicher beschreiben und fotos?}

\section{Ziel der Arbeit}
Das Ziel dieser Arbeit ist es herauszufinden, ob und in wie fern die mobile Bildbearbeitung als effizienter Ansatz zur Aufmaßerfassung im Gerüstbau verwerdnet werden kann.
Ein Teilziel dieser Arbeit umfasst die Entwicklung eines IT-Artefakts in Form einer Android-Applikation, welche in die bestehende Systemarchitektur der Fa. \emph{Vero} eingebunden werden kann, und es den Monteuren ermöglichen soll, vor Ort auf den Baustellen Aufmaße digital mit Hilfe der App zu erfassen.
Hierbei soll insbesondere die intuitive Benutzung und positive Benutzererfahrung der App im Vordergrund stehen.
Wünschenswert wäre es zudem, am Ende dieser Arbeit für die Monteure, welche die App zur Aufmaßerfassung benutzen, eine gesteigerte Effizienz im Vergleich zu der bisherigen Methode der Aufmaßerfassung zu erkennen.


\section{Aufbau der Arbeit}
Eingangs wird die in dieser Arbeit angewandte Forschungsmethodik des ``Human-Centered Design Process'' nach \citeauthor{Norman13} in Kapitel 2 vorgestellt und beschrieben. \\

Anschließend wird in Kapitel 3 der bisherige Prozess der Aufmaßerfassung im Gerüstbau am Beispiel der Fa \emph{Vero} vorgestllt und die auftretende Problematik identifiziert und evaluiert. 
Infolgedessen wird ein möglicher optimierter Prozess der Aufmaßerfassung beschrieben. \\

In Kapitel 4 werden zunächst die erweiterten Nielsen-Heuristiken vorgestellt, und nach einer Skala gewichtet.
Zudem werden noch weitere Bewertungskriterien, die für die Benutzung einer mobilen Software-Lösung zur Aufmaßerfassung im Gerüstbau wichting sind, eingeführt.
Weiterführend werden drei ausgewählte vorhandene Lösungsalternativen aus dem Google Play-Store vorgestellt und mittels der oben genannten Bewertungskriterien evaluiert.
Abschließend werden dieser wichtigsten Ergebnisse der Evaluation in einem Zwischenfazit festgehalten. \\

Die Konzeption der eigenen Software-Lösung wird in Kapitel 5 weiter ausgeführt.
Hierzu werden die in Kapitel 4 am stärksten gewichteten Kriterien aufgeführt, und praktischen Umsetzungmöglichkeiten vorgestellt.
Außerdem werden hierbei sowohl Bausteine, die bei der Evaluation der drei Lösungsalternativen besonderns positiv aufgefallen sind, als auch Best-Practices aus den Material-Guidelines von Google \citep{AndroidMG} berücksichtigt. \todo{ref auf papers?} \\

In Kapitel 6 wird die Implementierung eines ersten Prototyps vorgestellt, und anschließend in Hinblick auf Funktionalität und Usability evaluiert. Dies geschieht in einem iterativen Verfahren, wobei eine überarbeitete Implementierung des Prototypen in Zeitabständen von ungefähr 30 Tagen an eine ausgewählte Gruppe von Monteuren ausgerollt wird. 
Das Feedback zu dem jeweiligen Prototypen wird genutzt, um mögliche Probleme in der nächsten Iteration zu lösen. \\

Im Anschluss an den iterativen Entwicklungsprozess werden in Kapitel 7 eventuelle Probleme und Grenzen, die sich während der Implementierung und Evaluation gezeigt haben, aufgezeigt und bewertet.

Abschließend wird in Kapitel 8 ausgewertet, ob es eine erkennbare Verbesserung in Hinblick auf Effizienz und Kosten bei Benutzung der Software-Lösung zur Aufmaßerfassung im Vergleich zu der traditionellen Vorgehensweise gibt. \\
