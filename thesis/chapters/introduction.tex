\chapter{Einleitung}
\section{Motivation}\label{subsec:motivation}
``Digitalisierung des Handwerks'' - Das ist der Titel einer im Jahr 2017 ausgeführten Studie vom Branchenverband \textit{Bitkom}. 
Diese Studie besagt, dass 69\% der 509 befragten Handwerksbetriebe die Digitalisierung als Chance für einen positiven Umschwung der Betriebe sehen. 
Dabei stehen aus Sicht der Handwerker die Vorteile von ``optimaler Lagerung und Logistik'' (91\%), ``Zeitersparnis'' (81\%) und ``flexibler Arbeitsorganisation'' (78\%) im Vordergrund \citep{Bitkom17}. \\

Im November 2017 wurde zu dem Thema ein zweitägiges Groß-Seminar unter dem Titel ``Gerüstbau 4.0: Was bedeutet der digitale Wandel für den eigenen Betrieb?'' in Hannover vom Güterschutzverband Stahlgerüstbau e.V. veranstaltet. 
Mit Vorträgen zu Themen wie ``Geschäftsprozesse im Gerüstbau-Handwerk'', ``Virtuelle Projekträume und Cloudlösungen'' und ``Mobile Datenverwaltung'' wurden die Teilnehmenden über den digitalen Wandel informiert \citep{GSV17}. \\

Der Gerüstbau-Dienstleister \emph{Fa. VERO Scaffolding EOOD Niederlassung Deutschland} aus Paderborn (nachfolgend: \vr{}) kann nach der Einführung und Benutzung einer selbst entwickelten Android-Applikation im Mai 2014 die genannten Vorteile der \textit{Bitkom}-Studie teilweise verzeichnen.
Die entwickelte App wird einerseits von den Monteuren zur Arbeitszeiterfassung, Einsicht von Auftragsinformationen, und der Baustellendokumentation mit Fotos und Text genutzt.
Auf der anderen Seite nutzt die Geschäftsleitung bzw. Bauleitung die administrativen Funktionen der App, wie die Verwaltung und Pflege von Kunden-, Projekt-, und Arbeitsauftragsinformationen (\todo{ref auf bilder}). \\

Zu diesen administrativen Aufgaben gehört auch die Aufmaßerfassung.
Die Aufmaßerfassung ist ein wichtiger Bestandteil als Vorbereitung zur Rechnungslegung im Handwerk, insbesondere im Gerüstbau. 
Hierbei werden die Maße (Länge, Breite und Höhe) der bereitgestellten Gerüste, idealerweise abgemessen, meistens jedoch abgeschätzt und später im Büro mit einer Preisliste nach Gerüsttyp (Fassadengerüst, Raumgerüst, Hängegerüst, etc.) abgerechnet. \\
\todo{3 bilder von app}

\section{Umfeld}
Nach der Handwerkszählung vom statistischen Bundesamt im Jahr 2015 sind 13\% (75 264) aller Handwerksunternehmen dem Bauhauptgewerbe zuzuordnen.
Zu diesem Gewerbe zählt auch die Gerüstbaubranche, welche laut der Statistik im überwiegend von kleinen Unternehmen dominiert wird (88,03\% der Unternehmen mit bis zu 20 Angestellten). 
9,28\% der Unternehmen verzeichnen zwischen 21 und 50 Beschäftigte, und nur 2,69\% der Unternehmen verzeichnen über 50 Beschäftigte \citep{HZ17}.
Mit 14 Beschäftigten gehört die \vr{} zu einem der kleineren Unternehmen in der Gerüstbaubranche.
Im Folgenden soll das Unternehmen und die selbst entwickelte Android-App vorgestellt werden.

\subsection{Unternehmen}
Das heutige Unternehmen \vr{} geht aus der im Jahr 1999 in Paderborn gegründeten \emph{VERO Gerüstbau GmbH} hervor. 
Seit 2013 ist die \emph{VERO Gerüstbau GmbH} die deutsche Niederlassung der bulgarischen \emph{VERO Scaffolding EOOD}, ansässig in Stara Zagora, Bulgarien.
Der Standort in Bulgarien fokussiert sich hauptsächlich auf mehrjährige Großprojekte, wohingegen der Fokus in der deutschen Niederlassung in Paderborn auf kleineren Projekten für Kommunen, Automobilzulieferer, aber auch private Haushalte liegt.
Die Belegschaft besteht aus zehn Monteuren, einem Auszubildenden, einer Bürofachkraft und zwei Geschäftsführern.
Die Monteure sind zwischen 19 und 55 Jahren alt und erledigen in zwei bis vier Kolonnen bis zu vier Arbeitsaufträge am Tag.

\subsection{Android-App und Systemarchitektur}
Im Oktober 2013 begann der Entwicklungsstart einer Android-Applikation als mobile Software-Lösung für grundlegende Aufgaben der Monteure wie das Erfassen von Arbeitszeiten, Ansehen von Arbeitsaufträgen und das Verfassen von schriftlichen Baudokumentationen.
Die Applikation kommuniziert zur Synchronisation der gesammelten Daten mit einer Server-Schnittstelle (\emph{API}), welche auch vom Unternehmen entwickelt wurde.
Diese \emph{API} ist wiederum an eine SQL-Datenbank, ein Dateisystem und externe Dienste, wie einem E-Mail-Dienst, gekoppelt. \\

Im Vergleich zur anfänglichen Entwicklung aus dem Jahr 2013 verfügt die App zum jetzigen Stand (Januar 2018) über eine Vielzahl neuer Funktionen, die dem Nutzer den Arbeitsalltag so gut wie möglich erleichtern sollen.
Hierzu zählt zum Beispiel eine monatliche Stundenübersicht für die Mitarbeiter, die mobile Verwaltung von Stammdaten (wie etwa von Mitarbeitern und Ressourcen), sowie eine Vorplanungstafel, um Termine und Arbeitsaufträge übersichtlich zu planen.
\todo{Foto von Stundenübersicht und Vorplanungstafel}

\section{Ziel der Arbeit}
Im Rahmen dieser Arbeit soll die mobile Bildbearbeitung als effizienter Ansatz zur Aufmaßerfassung im Gerüstbau untersucht werden.
Ein Teilziel dieser Arbeit umfasst die Entwicklung eines IT-Artefakts in Form einer Android-Applikation, welche in die bestehende Systemarchitektur der \emph{Fa.} \vr{} eingebunden werden soll.
Weiterhin soll es den Monteuren ermöglichen, auf der Baustelle Aufmaße digital mit Hilfe der App zu erfassen.
Hierbei soll insbesondere die intuitive Benutzung und positive Benutzererfahrung der App im Vordergrund stehen.
Wünschenswert wäre es zudem, am Ende dieser Arbeit eine gesteigerte Effizienz im Vergleich zu der bisherigen analogen Methode der Aufmaßerfassung verzeichnen zu können.

\section{Aufbau der Arbeit}
Eingangs wird die in dieser Arbeit angewandte Forschungsmethodik des \hcdp{} nach \citeauthor{Norman13} in Kapitel 2 vorgestellt und die Anwendung in dieser Arbeit beschrieben. \\

Anschließend wird in Kapitel 3 der bisherige Prozess der Aufmaßerfassung im Gerüstbau am Beispiel der \emph{Fa.} \vr{} vorgestellt, und die potentiellen Schwachstellen identifiziert. 
Infolgedessen wird ein möglicher, optimierter Prozess der Aufmaßerfassung beschrieben. \\

In Kapitel 4 werden zunächst Bewertungskriterien nach Nielsen zur Evaluation vorgestellt und gewichtet.
Nachfolgend werden drei Apps als mögliche Lösungsalternativen aus dem Google Play-Store vorgestellt und mittels der zuvor definierten Bewertungskriterien evaluiert.
Abschließend werden die Ergebnisse der Evaluation in einer Tabelle festgehalten. \\

Die Konzeption der eigenen Software-Lösung wird in Kapitel 5 weiter ausgeführt.
Zu den in Kapitel 4 vorgestellten Kriterien werden praktischen Umsetzungsmöglichkeiten vorgestellt und diskutiert. \\

In Kapitel 6 wird die Implementierung eines ersten Prototyps vorgestellt.
Dieser Prototyp wird anschließend in die bestehende Android-App eingebunden, und für einen festgelegten Zeitraum von den Anwendern im Arbeitsalltag getestet.
Im Anschluss daran wird gemeinsam mit den Testpersonen der Prototyp evaluiert. \\

Basierend auf der Evaluation aus Kapitel 6 werden in den nachfolgenden Kapiteln 7, 8 und 9 drei weitere Iterationen des \hcdp{} durchgeführt.
Hier soll der Prototyp so weit verbessert werden, bis keine weiteren Usability-Probleme beim Testen mehr auffallen. \\

Infolgedessen werden eventuelle Probleme und Grenzen aufgezeigt und bewertet, die während den Iterationen identifiziert wurden. \\

Abschließend wird ausgewertet, ob es eine Verbesserung im Hinblick auf die Effizienz und Kosten bei der Benutzung der Android-App zur Aufmaßerfassung im Vergleich zur traditionellen, analogen Vorgehensweise gibt.
