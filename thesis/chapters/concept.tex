\chapter{Konzeption}

Bei der Konzeption der eigenen Android-App liegt der Fokus auf der Erfüllung aller in Kapitel 4 als relevant klassifizierten Kriterien.
Hierzu werden im Folgenden Implementierungsansätze vorgestellt, die für einen ersten Prototypen umgesetzt werden sollen.

\section{Sichtbarkeit des Systemzustandes}
Um dem Benutzer zu jeder Zeit einen Überblick über die App zu ermöglichen, soll eine Statusleiste, wie sie in Kapitel 4 bei allen drei Apps vorhanden war, eingesetzt.
Diese soll einerseits den aktuellen Modus anzeigen, andererseits soll erkennbar sein, welche Aktionen in dem aktuellen Modus ausführbar sind, und welche nicht.
\todo{Bottom bar ref}

\section{Übereinstimmung zwischen System und Welt}
Zusätzlich soll die App in einer natürlichen und logischen Reihenfolge benutzbar sein.
So soll dem Benutzer zuerst die Möglichkeit gegeben werden, ein gewünschtes Bild auszuwählen, welches im Anschluss daran in einer weiteren Benutzeroberfläche bearbeitet werden kann.

\section{Benutzerkontrolle und -freieht}
Gerade die fehlende Undo/Redo-Funktionalität hat sich bei den bei der App \pm{} als signifikantes Usability-Problem identifizieren lassen.
In der App soll diese Funktionalität über zwei dedizierte Icons in der Menübar angeboten werden, um potentielle Verwirrung wie bei der App \im{} zu vermeiden.
\todo{Menüleiste aus anderen Apps mit Redo/Undo}

\section{Konsistenz und Standards}
Um eine positive Benutzererfahrung zu generieren, und die Benutzung von nicht-intuitiven Icons zu vermeiden, werden bei der Implementierung die Design-Richtlinien von \emph{Google} befolgt.
Diese bieten sowohl Konzepte zur Umsetzung verschiedener Benutzeroberflächen-Elemente, als auch eine Vielzahl konsistent gestalteter Icons an.  \todo{ref auf Material-Guidelines}

\section{Fehlervorbeugung}
Zusätzlich soll das Deaktivieren von nicht-ausführbaren Aktionen dafür sorgen, dass Situationen, in denen Fehler entstehen können, präventiv vermieden werden.
Hierzu sollen nicht-benutzbare Icons ausgegraut und nicht auswählbar gestaltet werden.
\todo{disabled Icons in Apps}

\section{Wiedererkennung statt Erinnern}
Eine minimierte kongnitive Belastung soll durch die Verwedung von gleichen Icons für gleiche Aktionen erzielt werden.
So soll im ``Zeichen-Modus'' als auch im ``Text-Modus'' ein Mülleimer-Icon in der Menüleiste die Möglichkeit bieten, die markierte Form bzw. den ausgewählten Text zu löschen.

\section{Fleixibilität und Effizienz der Benutzung}
Der Benutzer soll in der Lage sein, die gezeichneten Formen im Nachhinein zu bearbeiten.
Hierzu soll es die Möglichkeit geben, dass der Benutzer bereits zu Beginn eine Standard-Farbe festlegen kann.
Diese soll dann für alle neu eingezeichneten Formen benutzt werden.
Zudem sollte es die Möglichkeit geben, einzelne Formen nachträglich anders färben zu können.
Abkürzungen für erfahrene Benutzer sollen durch das Einführen von verschiedenen Gesten geben werden.
So soll es zum Beispiel möglich sein, per Lang-Klick Geste auf eine Form Text eintragen zu können, anstatt erst in der Statusleiste die entsprechenden Button zu drücken.
\todo{Lang-Klick Geste}

\section{Erkennbarkeit, Diagnose und Erholung von Fehlern}
Durch das präventive Vermeiden von Situationen, in denen Fehler auftreten können, sollte der Nutzer zu keiner Zeit in die Lage geraten, sich von Fehlern erholen zu müssen.
Sollte es dennoch zu einer solchen Situation kommen, so soll eine kurze, aber informative Nachricht in Form eines ``Toast'' auf dem Bildschirm angezeigt werden, welches dem Benutzer über den Fehler informiert, und präzise aber verständliche Anweisungen gibt, wie er diesen Fehler lösen kann. \todo{Toast bild}.
Zu keiner Zeit sollte der Benutzer in eine endlose Schleife von Fehlerzuständen gelangen können, welche er nur durch das Beenden oder die Deinstallation der App unterbrechen kann.

\section{Hilfe und Dokumentation}
Wie in Kapitel 4 festgestellt, ist die Verwendung eines Hilfe-Overlays, welches über die eigentliche Benutzeroberfläche gelegt wird, sinnvoll. \todo{ref auf Kapitel}
Dieses soll zum initialen Start der App den Benutzer über die verschiedenen Modi und deren Funktionen informieren.
Beim jeweils ersten Wechsel in einen Modus soll ein anderes Hilfe-Overlay angezeigt werden, welches die verschiedenen Aktionen des ausgewählten Modus genauer erklärt.
\todo{Tap Taget}

\section{Adäquater Umgang mit Unterbrechungen}
Die App soll bei Unterbrechung der derzeitigen Aktivität, wie der Navigation zu einer anderen App, oder dem Wechsel auf den Home-Bildschirm, alle Informationen speichern, und beim Zurückkehren in die pausierte App wieder vollständig anzeigen.
Dies kann durch die Implementierung eines \emph{Saved States} erzielt werden.
\todo{Saved State}

\section{Fokussieren der Informationen}
Ein weiterer Wichtiger Umsetzungspunkt liegt beim Hervorheben wichtiger Informationen auf dem Bildschirm.
So sollen markierte Formen durch die Benutzung einer Akzent-Farbe als Schatten oder Umrandung schnell erkennbar sein.
\todo{Accent-Color}

\section{``Joy of Use''}
Die Bedienung der App soll intuitiv und einfach sein, um ein positives Nutzungerlebnis zu garantieren.
Hierzu sollen verschiedene Gesten zur Navigation im Bild unterstützt werden.
Unter anderem soll das Vergrößern und Verkleinern des Bildes durch Pinch und Doppel-Tap Gesten ermöglicht werden.
Zudem sollen Swipe-Gesten dazu genutzt werden, eine schnelle und intuitive Navigation im Bild zu verwirklichen.
\todo{Gesture Support, papers}

\section{Unterstützung verschiedener Bildschirmausrichtungen}
Insbesondere aufgrund der Zielgruppe im Gerüstbau soll die App im Querformat genau so einfach zu bedienen sein, wie im Hochformat.
So sollen Bilder, die im Querformat aufgenommen werden, auch in diesem Bearbeitet werden können, ohne dass sich die Ausrichtung des Bildes unerwartet ändert.

\section{Ergonomische Gestaltung der physischen Interatkion}
Das Unabsichtliche Auslösen von Funktionen soll in allen Fällen vermieden werden.
Hierzu muss insbesondere darauf geachtet werden, dass Swipe- und Zoom-Gesten verlässlich von Zeichen-Gesten unterschieden werden, um unabsichtliches Zeichnen von Formen, wie in Kapitel 4 beschrieben, zu vermeiden.
\todo{Verwendung verschiedener Gesture-Listener}

\section{Einfache Eingabe, Bildschirmlesbarkeit und Übersichtlichkeit}
Der Benutzer soll stets in der Lage sein, die gesamte App mit einer Hand bedienen zu können.
\todo{Wichtigte Kriterien für einhändige Handybenutzung}

% \subsection{Die 8 Goldenen Regeln von Shneiderman}

% \citeauthor{Shneiderman04} definieren in ihrem Buch ``Designing the User Interface: Strategies for Effective Human-Computer Interaction'' die sogenannten ``8 Goldenen Regeln des Interface Designs''.  Die acht Regeln lauten wie folgt:

% \begin{enumerate}
% \item Bemühe Dich um Konsistenz (``Strive for consistency'')
% \item Erlaube erfahrenen Benutzern die Nutzung von Abkürzungen (``Enable fequent users to use shortcuts'')
% \item Biete informative Rückkopplung (``Offer informative feedback'')
% \item Biete ein klares Ende von Teildialogen an (``Design dialogs to yield closure'')
% \item Biete Fehlervermeidung und einfache Fehlerbehandlung an (``Offer error protection and simple error handling'')
% \item \label{itm:undo} Erlaube eine einfach Rücknahme von Aktionen (``Permit easy reversal of actions'')
% \item Gib dem Benutzer die Kontrolle (``Support internal locus of control'')
% \item Minimiere Gedächtnisbelastung (``Reduce short-memory load'')
% \end{enumerate} 

% Diese Regeln sind ein weiterer guter Anhaltspunkt für die Entwicklung einer guten Software-Lösung, die den Faktor der Usability mit berücksichtigt. Vor allem der Punkt \autoref{itm:undo} sollte bei einer Applikation, deren Fokus auf den aneinander gereihten Aktionen des Benutzers beruht, erfüllt sein. Dies soll durch einen Undo- bzw. Redo-Button in der oberen Menüleiste gewährleistet werden. \\

% Die Konsistenz der Applikation soll durch die Benutzung der \citet{AndroidMG} gewährleistet werden. 
% So werden einerseits nur Icons aus der Standard Google-Library verwendet, andererseits werden diverese Guidelines die auf \citet{AndroidMG} vorgestellt werden, benutzt. 
% Hierzu zählt auch die sogenannte \emph{Feature-Discovery}.
% Dabei soll der Benutzer durch ein hinweisendes Overlay dazu aufgefordert werden, gewisse Funktionen der App zu erkunden. 
% Dies ist besonders bei einer App, in der es mehr als eine zentrale Funktion gibt, hilfreich und wichtig. \\

% \begin{figure}[h]
% \centering
% \includegraphics[keepaspectratio, width=0.5\textwidth]{fd}
% \caption{Feature-Discovery in Form eines Tap-Targets}
% \label{fig:fd}
% \end{figure}

% In \autoref{fig:fd} sieht man eine beispielhafte Implementierung einer solchen Feature-Discovery in Form eines Tap-Targets.
% Hierbei wird der Benutzer durch eine Hervorhebung bestimmter UI-Elemente auf Funktionen hingewiesen. 
% Zusätzlich wird ein kurzer erklärender Satz mit einem zusammenfassenden Titel angezeigt. \\

% \subsection{Gesture-Support}

% Mit der Einführung Applikationen für mobile Endgeräte hat sich ein weiteres, vorher nicht relevantes, Usability-Problem etabliert: 
% \emph{Die geringe Größe des Bildschirm}. \\

% Die relativ kleine physikalische Größe der Smartphone-Displays bringt verschiedene Probleme mit sich. 
% So muss einerseits die Funktionalität einer ganzen Desktop-Applikation auf ein viel kleineres Display passen, ohne den Content zu verändern, oder gar unleserlich zu machen.
% Andererseits muss dem Benutzer eine intuitive und effiziente Navigationsmöglichkeit gegeben sein, um zwischen verschiedenen Inhalten zu wechseln. \\

% Hierzu schreiben \citeauthor{Gutwin04}, dass die Navigationen auf kleinen Bildschirmen selbst im besten Fall deutlich langsamer sei als auf normal-großen Bildschirmen.
% Sie führen weiter an, dass eine Übersicht über den gesamten Systemzustand wertvoll sei, da dies dem Nutzer eine schnellere Navigation erlaube. \\
% Nach \citeauthor{Gutwin04} eigne sich die Technik des ``Two-Level Zooms'' besonders gut für Navigationsaufgaben, wohingegen Panning-Strategien eher negativ von den Test-Personen aufgenommen worden seien \citep[Seite 8]{Gutwin04}.  
