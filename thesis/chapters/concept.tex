\chapter{Konzeption}\label{chap:concept}
In diesem Kapitel werden die Evaluationsergebnisse aus \autoref{chap:eval} auf die Konzeption einer eigenen Android-App angewandt.
Hierzu werden Umsetzungsideen zu den einzelnen Bewertungskriterien, die zur Evaluation der Lösungsalternativen genutzt wurden, nacheinander vorgestellt.
Dieses Kapitel der Arbeit entspricht der dritten Phase (\emph{Idea Generation}) des \hcdp{} und dient dazu, Ideen zur Lösung der während der \emph{Observation} identifizieren Usability-Probleme aufzustellen.

\section{Sichtbarkeit des Systemzustandes}
\begin{wrapfigure}{R}{0.4\textwidth}
  \centering
  \includegraphics[keepaspectratio, width=0.4\textwidth]{concept/fab}
  \caption{Floating Action Button}
  \label{fig:fab}
\end{wrapfigure}

Um dem Benutzer zu jeder Zeit einen Überblick über den Systemzustand der App zu ermöglichen, soll ein \emph{Floating Action Button (FAB)} (siehe \autoref{fig:fab}) eingesetzt werden, da dieser nicht viel Platz auf dem Bildschirm einnimmt, für den Benutzer dennoch jederzeit leicht zugänglich ist.
So schreibt \citeauthor{SJ16} in seiner Usability-Studie zum Vergleich eines traditionellen Buttons mit einem \emph{FAB}, dass Aufgaben, nachdem sie einmal mit dem \emph{FAB} erfolgreich ausgeführt wurden, anschließend effizienter durchführbar seien als mit einem traditionellen Button \citep[Seiten 14--16]{SJ16}.

\section{Übereinstimmung zwischen System und Welt}
Zusätzlich soll die App in einer natürlichen und logischen Reihenfolge benutzbar sein.
So soll dem Benutzer zuerst die Möglichkeit gegeben werden, ein gewünschtes Bild auszuwählen, welches im Anschluss daran in einer weiteren Benutzeroberfläche bearbeitet werden kann.
Zudem soll die App in zwei Modi, den Zeichen- und Text-Modus, unterteilt werden, um eine klare Grenze zwischen dem Zeichnen und Eintragen von Messwerten zu schaffen.
Dieser Ablauf soll sicherstellen, dass der Benutzer bei der Verwendung der App nicht überfordert wird und sich intuitiv zurecht findet.

\section{Benutzerkontrolle und -freiheit}
Gerade die fehlende Undo- bzw. Redo-Funktionalität hat sich bei der App \pm{} in \autoref{subsec:pmeva} als signifikantes Usability-Problem identifizieren lassen.
Aufgrund dessen soll diese Funktionalität, wie bei \mm{} (siehe \autoref{fig:bicons}), über zwei dedizierte Icons in der Menüleiste der App angeboten werden. 
Dies soll potentielle Verwirrung wie in \autoref{subsec:imeva} bei der App \im{} vermeiden, und ein positives Nutzungserlebnis garantieren.

\section{Konsistenz und Standards}
Um eine positive Benutzererfahrung zu generieren, und die Benutzung von nicht-intuitiven Icons zu vermeiden, werden bei der Implementierung die Design-Richtlinien von \emph{Google} befolgt.
Diese sogenannten \mg{}\urlnote{https://material.io/guidelines/} bieten sowohl Konzepte zur Umsetzung verschiedener UI-Elemente, als auch eine Vielzahl konsistent gestalteter Icons an.

\section{Fehlervorbeugung}
Situationen, in denen Fehler auftreten können, sollen in der App präventiv vermieden werden.
Hierzu sollen Aktionen, die im aktuellen Systemzustand zu einer solchen Situation führen würden, deaktiviert werden.
Dies kann zum Beispiel durch das Ausgrauen der Icons von nicht-ausführbaren Aktionen umgesetzt werden, wie es auch bei \mm{} gemacht wird (siehe \autoref{fig:bicons}).

\section{Wiedererkennung statt Erinnern}
Eine minimierte kognitive Belastung soll durch die Verwendung von gleichartigen Icons für gleichartige Aktionen erzielt werden.
So soll bspw. im Zeichen-Modus als auch im Text-Modus ein Mülleimer-Icon in der Menüleiste die Möglichkeit bieten, die markierte Form bzw. den ausgewählten Text zu löschen.
Zur Auswahl dieser Icons können ebenfalls die \mg{} genutzt werden.

\section{Flexibilität und Effizienz der Benutzung}
Der Benutzer soll in der Lage sein, die gezeichneten Formen im Nachhinein zu bearbeiten.
Hierzu soll es die Möglichkeit geben, dass der Benutzer bereits zu Beginn eine Farbe festlegen kann.
Diese soll dann für alle neu eingezeichneten Formen benutzt werden.
Zudem sollten einzelne Formen im Nachhinein in einer anderen Farbe gefärbt und Texte bearbeitet werden können.
Die Unterstützung von verschiedenen Gesten sollen Abkürzungen für erfahrene Benutzer bereits stellen.
So soll es zum Beispiel möglich sein, per \emph{Langklick-Geste} auf eine Form Text eintragen zu können, anstatt erst den entsprechenden Button drücken zu müssen.

\section{Erkennbarkeit, Diagnose und Erholung von Fehlern}
Durch das präventive Vermeiden von Situationen, in denen Fehler auftreten können, sollte der Nutzer zu keiner Zeit in eine solche Situation gelangen können.
Sollte es dennoch dazu kommen, so soll eine kurze, aber informative Nachricht in Form eines \emph{Toasts} auf dem Bildschirm angezeigt werden.
Diese soll den Nutzer über den Fehler informieren und präzise aber verständliche Anweisungen geben, wie er diesen lösen kann.
Zu keiner Zeit sollte der Benutzer in eine Schleife von Fehlerzuständen gelangen können, welche er nur durch das Beenden oder die Deinstallation der App unterbrechen kann. 

\section{Hilfe und Dokumentation}
Zur Hilfe und Dokumentation sollen sogenannte \emph{Content-Descriptions}\urlnote{https://developer.android.com/guide/topics/ui/accessibility/apps.html} eingesetzt werden.
Diese ermöglichen es dem Benutzer, bei einem langen Klick auf ein Icon eine Beschreibung zur Funktion zu bekommen, die sich hinter dem Symbol verbirgt.
Hierdurch soll sichergestellt werden, dass der Nutzer nicht erst auf das Icon klicken und die Aktion ausführen muss, um zu sehen, mit welcher Funktion das Icon verknüpft ist.
Ein weiterer Vorteil bei der Benutzung von \emph{Content-Descriptions} liegt in der Unterstützung von Screen-Readern und der Navigation per Tastatur.

\section{Adäquater Umgang mit Unterbrechungen}
Die App soll bei Unterbrechung der derzeitigen Aktivität, wie der Navigation zu einer anderen App, oder dem Wechsel auf den Start-Bildschirm, alle Informationen speichern, und beim Zurückkehren in die pausierte App wieder vollständig herstellen.
Dies kann durch die Implementierung eines \emph{Saved States}\urlnote{https://developer.android.com/topic/libraries/architecture/saving-states.html} umgesetzt werden.
\emph{Saved States} werden eingesetzt, um Informationen während des \emph{Activity Lifecycles} (vgl. \autoref{fig:lifecycle}) zu speichern.
Der \emph{Activity Lifecycle} ist eine Besonderheit unter Android und stellt der \emph{Activity}\urlnote{https://developer.android.com/reference/android/app/Activity.html} eine Reihe verschiedener Methoden zur Verfügung.
Diese sollten von der \emph{Activity} überschrieben werden, um eine reibungslose Navigation des Benutzer in der App und zwischen verschiedenen Apps zu ermöglichen.
Auch wird hierdurch ein möglicher Datenverlust verhindert, der entstehen kann, wenn eine App mit einer höheren Priorität den Speicherbereich zugeordnet bekommt, der zuvor von der eigenen App benutzt wurde \citep{alifecycle}.

\section{Fokussieren der Informationen}
Ein weiterer Aspekt liegt beim Hervorheben wichtiger Informationen auf dem Bildschirm.
So sollen Formen durch die Benutzung einer Akzent-Farbe als Schatten oder Umrandung schnell als ``markiert'' erkennbar sein und dem Benutzer einen schnellen Überblick über den Systemzustand verschaffen.
Außerdem sollen ausgefüllte Kreise an den Eckpunkten der zur Zeit markierten Formen als \emph{Call for action} dienen, und dem Nutzer so darauf aufmerksam machen, dass die Form mit Hilfe dieser Eckpunkte in ihrer Position und damit ihrer Größe verändert werden kann.

\section{``Joy of Use''}
Die Bedienung der App soll intuitiv und einfach möglich sein, um ein positives Nutzungserlebnis zu garantieren.
Hierzu sollen verschiedene Gesten zur Navigation im Bild unterstützt werden (vgl. \autoref{fig:gesture}).
Unter anderem soll das Vergrößern und Verkleinern des Bildes durch eine \emph{Pinch-} und \emph{Doppelklick-Geste} ermöglicht werden.
Zudem soll eine \emph{Pan-Geste} dazu genutzt werden, eine schnelle und präzise Navigation im Bild zu verwirklichen.
\begin{figure}[h]
  \begin{subfigure}[t]{0.3\textwidth}
    \centering
    \includegraphics[keepaspectratio, width=\textwidth]{concept/pinch}
    \caption{\emph{Pinch}-Geste}
  \end{subfigure}
  ~
  \begin{subfigure}[t]{0.3\textwidth}
    \centering
    \includegraphics[keepaspectratio, width=\textwidth]{concept/doubletap}
    \caption{\emph{Doppelklick}-Geste}
  \end{subfigure}
  ~
  \begin{subfigure}[t]{0.3\textwidth}
    \centering
    \includegraphics[keepaspectratio, width=\textwidth]{concept/pan}
    \caption{\emph{Pan}-Geste}
  \end{subfigure}
  \centering
  \caption{Die drei verschiedenen Gesten zur Navigation im Bild}
  \label{fig:gesturea}
\end{figure}

\section{Unterstützung verschiedener Bildschirmausrichtungen}
Insbesondere weil die App zur Bearbeitung von Bildern eingesetzt werden soll, sollte sie im Querformat genau so einfach zu bedienen sein, wie im Hochformat.
So sollen Bilder, die im Querformat aufgenommen werden, auch in diesem Bearbeitet werden können, ohne dass sich die Ausrichtung des Bildes unerwartet ändert.

\section{Ergonomische Gestaltung der physischen Interaktion}
Das unabsichtliche Auslösen von Funktionen soll in allen Fällen vermieden werden.
Hierzu muss insbesondere darauf geachtet werden, dass \emph{Pan}- und \emph{Pinch-Gesten} verlässlich von Zeichen-Gesten unterschieden werden, um unabsichtliches Zeichnen von Formen, wie in \autoref{chap:eval} beschrieben, zu vermeiden.

\section{Einfache Eingabe, Bildschirmlesbarkeit und Übersichtlichkeit}
Der Benutzer soll stets in der Lage sein, die gesamte App mit einer Hand bedienen zu können.
Um die App übersichtlich zu gestalten, sollen häufig genutzte Aktionen in die Menüleiste der App integriert werden.
Hierzu gehören bspw. die Undo- bzw. Redo-Funktionalität, das Auswählen der verschiedenen Formen und das Löschen von markierten Formen bzw. ausgewählten Texten.

% \subsection{Die 8 Goldenen Regeln von Shneiderman}

% \citeauthor{Shneiderman04} definieren in ihrem Buch ``Designing the User Interface: Strategies for Effective Human-Computer Interaction'' die sogenannten ``8 Goldenen Regeln des Interface Designs''.  Die acht Regeln lauten wie folgt:

% \begin{enumerate}
% \item Bemühe Dich um Konsistenz (``Strive for consistency'')
% \item Erlaube erfahrenen Benutzern die Nutzung von Abkürzungen (``Enable fequent users to use shortcuts'')
% \item Biete informative Rückkopplung (``Offer informative feedback'')
% \item Biete ein klares Ende von Teildialogen an (``Design dialogs to yield closure'')
% \item Biete Fehlervermeidung und einfache Fehlerbehandlung an (``Offer error protection and simple error handling'')
% \item \label{itm:undo} Erlaube eine einfach Rücknahme von Aktionen (``Permit easy reversal of actions'')
% \item Gib dem Benutzer die Kontrolle (``Support internal locus of control'')
% \item Minimiere Gedächtnisbelastung (``Reduce short-memory load'')
% \end{enumerate} 

% Diese Regeln sind ein weiterer guter Anhaltspunkt für die Entwicklung einer guten Software-Lösung, die den Faktor der Usability mit berücksichtigt. Vor allem der Punkt \autoref{itm:undo} sollte bei einer Applikation, deren Fokus auf den aneinander gereihten Aktionen des Benutzers beruht, erfüllt sein. Dies soll durch einen Undo- bzw. Redo-Button in der oberen Menüleiste gewährleistet werden. \\

% Die Konsistenz der Applikation soll durch die Benutzung der \citet{AndroidMG} gewährleistet werden. 
% So werden einerseits nur Icons aus der Standard Google-Library verwendet, andererseits werden diverese Guidelines die auf \citet{AndroidMG} vorgestellt werden, benutzt. 
% Hierzu zählt auch die sogenannte \emph{Feature-Discovery}.
% Dabei soll der Benutzer durch ein hinweisendes Overlay dazu aufgefordert werden, gewisse Funktionen der App zu erkunden. 
% Dies ist besonders bei einer App, in der es mehr als eine zentrale Funktion gibt, hilfreich und wichtig. \\

% \begin{figure}[h]
% \centering
% \includegraphics[keepaspectratio, width=0.5\textwidth]{fd}
% \caption{Feature-Discovery in Form eines Tap-Targets}
% \label{fig:fd}
% \end{figure}

% In \autoref{fig:fd} sieht man eine beispielhafte Implementierung einer solchen Feature-Discovery in Form eines Tap-Targets.
% Hierbei wird der Benutzer durch eine Hervorhebung bestimmter UI-Elemente auf Funktionen hingewiesen. 
% Zusätzlich wird ein kurzer erklärender Satz mit einem zusammenfassenden Titel angezeigt. \\

% \subsection{Gesture-Support}

% Mit der Einführung Applikationen für mobile Endgeräte hat sich ein weiteres, vorher nicht relevantes, Usability-Problem etabliert: 
% \emph{Die geringe Größe des Bildschirm}. \\

% Die relativ kleine physikalische Größe der Smartphone-Displays bringt verschiedene Probleme mit sich. 
% So muss einerseits die Funktionalität einer ganzen Desktop-Applikation auf ein viel kleineres Display passen, ohne den Content zu verändern, oder gar unleserlich zu machen.
% Andererseits muss dem Benutzer eine intuitive und effiziente Navigationsmöglichkeit gegeben sein, um zwischen verschiedenen Inhalten zu wechseln. \\

% Hierzu schreiben \citeauthor{Gutwin04}, dass die Navigationen auf kleinen Bildschirmen selbst im besten Fall deutlich langsamer sei als auf normal-großen Bildschirmen.
% Sie führen weiter an, dass eine Übersicht über den gesamten Systemzustand wertvoll sei, da dies dem Nutzer eine schnellere Navigation erlaube. \\
% Nach \citeauthor{Gutwin04} eigne sich die Technik des ``Two-Level Zooms'' besonders gut für Navigationsaufgaben, wohingegen Panning-Strategien eher negativ von den Test-Personen aufgenommen worden seien \citep[Seite 8]{Gutwin04}.  
