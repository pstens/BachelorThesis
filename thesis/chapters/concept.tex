\chapter{Konzeption}

\section{Bewertungskriterien}
 
\section{Bewertung vorhandener Lösungalternativen}

Im Folgenden werden die 3 Alternativen nacheinander mit Hilfe der oben genannten Heuristiken/Kriterien bewertet, und abschließend ein vergleichendes Fazit in Form einer Tabelle \todo{ref auf tabelle?} gezogen.

\subsection{Photo Measures}

\textsc{Photo Measures} von Big Blue Pixel Inc. VERSION, RATING \todo{link zum Playstore} \\

\begin{wrapfigure}{R}{0.4\textwidth}
	\centering
	\includegraphics[keepaspectratio, width=0.4\textwidth]{photo_measures/help}
	\caption{Initialer Start}	
	\label{fig:pmhelp}
\end{wrapfigure}

Die App zeigt beim ersten Start ein helfendes Overlay, welches dem Nutzer genau erklärt, wie die App zu benutzen ist. Dieser Punkt fällt nach Nielsen unter \ref{itm:10} (Siehe Abbildung\ref{fig:pmhelp}) \\
 
Aktionen sind nur dann verfügbar, wenn sie benutzbar sind. Somit werden Fehler vorgebeugt, und der Benutzer weiß zu jeder Zeit, in welchem Systemzustand er sich befindet. Dies korrespondiert zu den Heuristiken~\ref{itm:1} und \ref{itm:5}. \todo{2 bilder mit bottom-bars} \\

Die App bedient sich einer Reihe universell bekannter Icons, sodass intuitiv erkennbar ist, welche Aktion sich hinter welchem Button verbirgt, ohne groß darüber nachdenken zu müssen. Zusätzlich stehen die entsprechenden Aktionen als Text unter den Icons. Dies kann nützlich sein, wenn ein Icon nicht auf Anhieb wiedererkannt wird. \ref{itm:4} \todo{ref auf pic von bottom-bars} \\

Des Weiteren gibt die App dem Benutzer die Möglichkeit, Formen, Größen und Farben anzupassen. Dies fördert eine flexible und effizente Benutzung. \ref{itm:7} \\

Ein durchaus schwerwiegender negativer Punkt liegt bei der Benutzerkontrolle \ref{itm:3} der App. So ist es dem Benutzer nicht möglich, über einen Undo- bzw. Redo-Button seine Aktionen zu revidieren. Dies ist gerade bei der Bearbeitung von Bildern, wo es viele aneinandergereihte Aktionen des Benutzers gibt, eine entscheidende Funktionalität, welche nicht nur die Gedächtnisbelastung des Benutzer senken, sondern auch den ``Joy of Use'' deutlich steigern kann. \\

Zudem bietet die App keine für den Nutzer erkennbare Ausstiegsmöglichkeit \ref{itm:6} an. Es gibt weder einen Zurück-Button, noch einen Button um das annotierte Bild explizit zu speichern. Die einzige Ausstiegsmöglichkeit erfolgt über die Zurück-Navigationstaste des Smartphones, welche das Bild auch zusätzlich speichert. Diese Lösungsvariante ist für den Nutzer nicht intuitiv verständlich. \todo{bild}

Die App erfüllt nahezu alle acht (\ref{itm:11}-\ref{itm:18}) Heuristiken für mobile Geräte. Zu jeder Zeit ist auf dem Bildschirm erkennbar, welche Form zur Zeit ausgewählt ist. Das Smartphone kann während der Benutzung pausiert bzw. gedreht werden, ohne dass Informationen verloren gehen, oder der Benutzer durch unbekannte Bausteine überrascht wird. \todo{screens} Hier fällt als einziger negativer Punkt die unzureichende Gesten-Unterstützung auf \ref{itm:13}. So malt der Benutzer unabsichtlich mit jeder Zoom-Geste eine Form in das Bild, welche danach wieder gelöscht werden muss, da es keine Undo-Funktion gibt. \\

\subsection{Measuring Master}

\textsc{Measuring Master} von Robert Bosch GmbH VERSION, RATING \\

\begin{wrapfigure}{R}{0.4\textwidth}
	\centering
	\includegraphics[keepaspectratio, width=0.4\textwidth]{bosch/help}
	\caption{Zeichen-Modus}	
	\label{fig:bhelp}
\end{wrapfigure}

Die App zeigt in einer Art Statusleiste am unteren Rand des Bildschirms den aktuellen Modus an, und gibt über einen auffordernden Text am oberen Bildschirmrand dem Nutzer eine Hilfestellung, was er im gerade ausgewählten Modus machen kann. \todo{screens} Hiermit deckt die App Nielsen \ref{itm:1} und \ref{itm:10} ausreichend ab. \\

Des Weiteren benutzt auch diese App universell verständliche Icons, um die wichtigsten Aktionen wiedererkennbar zu machen. So hat beispielsweise das Mülleimer-Icon in jedem Modus die Löschfunktion. \\

Im Gegensatz zu \textsc{Photo Measures} bietet diese App dem Benutzer die Möglichkeit seine Aktionen rückgängig zu machen, oder sie zu wiederholen. Dies ist ein deutlicher Vorteil seitens der Usability, da Fehler nicht so hart bestraft werden, als wenn keine Undo/Redo-Button vorhanden wären. \\

Fehler werden hier durch das Deaktivieren von Buttons, die im aktuellen Systemzustand nicht benutzbar sind, präventiv verhindert. Das Löschen von Formen ist beispielsweise nur dann möglich, wenn zuvor eine Form ausgewählt wurde.

Negativ fällt auch in dieser Alternative die fehlerhafte Gesten-Unterstützung auf. So sorgen Zoom-Gesten per Doppel-Tap zum unabsichtlichen Zeichnen einer Form, welche im Nachhinein wieder gelöscht werden muss. Außerdem verletzt die App Nielsen~\ref{itm:15}, da Änderungen in der Bildschirmausrichtung dafür sorgen, dass das Bild nicht wie erwartet seine Ursprungsausrichtung beibehält, sonder auch rotiert wird. 

\begin{figure}[h]
	\begin{subfigure}[b]{0.5\textwidth}
		\includegraphics[keepaspectratio, width=0.9\linewidth]{bosch/portrait}
		\caption{App im Portrait-Modus}
		\label{fig:bportait}	
	\end{subfigure}
	~
	\begin{subfigure}[b]{0.5\textwidth}
		\includegraphics[keepaspectratio, height=0.7\linewidth]{bosch/landscape}
		\caption{App im Landscape-Modus}
		\label{fig:blandscape}	
	\end{subfigure}
	\caption{Bildschirmrotation Bosch-App}
	\label{fig:borientation}
\end{figure}

\subsection{Measures \& Sketch}

\textsc{Measures \& Sketch} von HERSTELLER, VERSION, RANKING \\

\begin{wrapfigure}{R}{0.4\textwidth}
	\includegraphics[keepaspectratio, width=0.4\textwidth]{measure_sketch/help}
	\caption{Hilfe-Overlay}
	\label{fig:mshelp}
\end{wrapfigure}

Auch diese App bedient sich eines Hilfe-Overlays beim ersten Start, überfordert den Benutzer jedoch mit zu viel Text. So erfüllt dieses Tooltip ihre Funktion als Hilfestellung nicht, sondern überfordert den Nutzer mit Text. Alternativ bietet sich auch bei dieser App wie in \autoref{fig:pmhelp} mit Icons zu arbeiten, sodass die Gedächtnisbelastung minimiert wird. \\

Die App kann als negativ-Beispiel bezüglich der Nielsen-Heuristiken betrachtet werden. Es gibt weder Undo- oder Redo-Button, noch wird in irgendeiner Weise hervorgehoben, welche Form aktuell ausgewählt ist. Dies führte beim Löschen oft zu Überraschungen. \\

 Außerdem unterstützt die App keinerlei Gesten zur Navigation im Bild. So lässt sich der abgebildete Bereich des Bildes weder zoomen, noch kann der Benutzer das Bild rotieren oder verschieben. Um Formen zu zeichnen bedient sich die App eine für den Benutzer unnatürliche Geste, denn hierzu muss der Nutzer gleichzeitig mit zwei Fingern die Form in die beiden gewünschten Richtungen ``aufziehen''. So fühlt sich der Zeichen-Prozess nicht nur unnatürlich an, sondern ist in der Größe der Form durch die Spannweite der Finger des Benutzers beschränkt. \\
 
 Zusätzlich schließt dies die Benutzung der App mit einer Hand aus, was Nielsen~\autoref{itm:16} klar widerspricht. Als Bildschirmausrichtung wird nur der Portrait-Modus unterstützt, was gerade die Bearbeitung von im Landscape aufgenommenen Bildern zu einer Herausforderung macht. \\

\section{Bewertung der Lösungsalternativen}

\todo{Punkte auf Konsistenz überprüfen}
\begin{sidewaystable}[ht]
	\centering
	\caption{Vergleich der Lösungsalternativen}
	\vspace*{10px}
	\label{tab:nielsen}
	\begin{tabular}{r|l|c|c|c|}
	\cline{2-5}
    	        				    &								& Photo Measures 	& Measuring Master 	& Measure \& Sketch \\ \cline{2-5} 
	Nach \cite{Nielsen94} 	& \autoref{itm:1}				&       \po 		&    \po 			&       \xmark      \\ \cline{2-5} 
    	             				& \autoref{itm:2} 				&       \po  		&    \po  			&       \po		    \\ \cline{2-5}
    	             				& \autoref{itm:3} 				&       \xmark 		&    \po			&       \xmark      \\ \cline{2-5} 
    	             				& \autoref{itm:4} 				&       \po  		&    \po			&       \xmark      \\ \cline{2-5}
    	            				& \autoref{itm:5} 				&       \po  		&    \xmark			&       \xmark      \\ \cline{2-5} 
    	            				& \autoref{itm:6} 				&       \xmark 		&    \po  			&       \xmark      \\ \cline{2-5} 
    	             				& \autoref{itm:7} 				&       \po  		&    \xmark			&       \xmark      \\ \cline{2-5} 
    	             				& \autoref{itm:8} 				&       \nl  		&    \po  			&       \xmark      \\ \cline{2-5} 
    	             				& \autoref{itm:9} 				&       \po   		&    \po  			&       \nl	        \\ \cline{2-5} 
    	            				& \autoref{itm:10} 				&       \po  		&    \po 			&       \xmark      \\ \cline{2-5} 
    	             				& \autoref{itm:11} 				&       \po   		&    \po 			&       \xmark      \\ \cline{2-5} 
    	             				& \autoref{itm:12} 				&       \po   		&    \po 			&       \xmark      \\ \cline{2-5} 
    	             				& \autoref{itm:13}			 	&       \xmark  	&    \xmark			&       \xmark      \\ \cline{2-5} 
    	            				& \autoref{itm:14} 				&       \po   		&    \po  			&       \po		    \\ \cline{2-5}
    	            				& \autoref{itm:15} 				&       \po   		&    \xmark			&       \xmark  	\\ \cline{2-5}   
    	            				& \autoref{itm:16} 				&       \po   		&    \po  			&       \xmark      \\ \cline{2-5} 
    	             				& \autoref{itm:17} 				&       \po  		&    \po  			&       \xmark		\\ \cline{2-5} 
    	             				& \autoref{itm:18} 				&       \nl  		&    \nl 			&       \nl		    \\ \cline{2-5} 
	Eigene Kriterien 				& \autoref{itm:integration}		&      	\xmark		&    \xmark			&       \xmark      \\ \cline{2-5}
	    	         				& \autoref{itm:export}   		&      	\xmark		&    \xmark			&       \xmark      \\ \cline{2-5}

	    	             
	\end{tabular}
	\\
	\vspace*{10px}
	\begin{tabular}{l}
		\po~wird erfüllt \\
		\nl~wurde nicht berücksichtigt \\
		\xmark~wird nicht erfüllt
	\end{tabular}
\end{sidewaystable}

\clearpage
\section{Vorgehensweise bei Implementierung eigener Software-Lösung}
\todo{Alle Aussagen mit Zitaten belegen?}
\todo{Schreiben was ich in dieser section mache}
\todo{Bezug auf Ergebnisse von alternativen} 
 \subsection{Die 8 Goldenen Regeln von Shneiderman}
 
\citeauthor{Shneiderman04} definieren in ihrem Buch ``Designing the User Interface: Strategies for Effective Human-Computer Interaction'' die sogenannten ``8 Goldenen Regeln des Interface Designs''.  Die acht Regeln lauten wie folgt:

\begin{enumerate}
	\item Bemühe Dich um Konsistenz (``Strive for consistency'')
	\item Erlaube erfahrenen Benutzern die Nutzung von Abkürzungen (``Enable fequent users to use shortcuts'')
	\item Biete informative Rückkopplung (``Offer informative feedback'')
	\item Biete ein klares Ende von Teildialogen an (``Design dialogs to yield closure'')
	\item Biete Fehlervermeidung und einfache Fehlerbehandlung an (``Offer error protection and simple error handling'')
  \item \label{itm:undo} Erlaube eine einfach Rücknahme von Aktionen (``Permit easy reversal of actions'')
	\item Gib dem Benutzer die Kontrolle (``Support internal locus of control'')
	\item Minimiere Gedächtnisbelastung (``Reduce short-memory load'')
\end{enumerate} 

Diese Regeln sind ein weiterer guter Anhaltspunkt für die Entwicklung einer guten Software-Lösung, die den Faktor der Usability mit berücksichtigt. Vor allem der Punkt \autoref{itm:undo} sollte bei einer Applikation, deren Fokus auf den aneinander gereihten Aktionen des Benutzers beruht, erfüllt sein. Dies soll durch einen Undo- bzw. Redo-Button in der oberen Menüleiste gewährleistet werden. \\

Die Konsistenz der Applikation soll durch die Benutzung der \citet{AndroidMG} gewährleistet werden. 
So werden einerseits nur Icons aus der Standard Google-Library verwendet, andererseits werden diverese Guidelines die auf \citet{AndroidMG} vorgestellt werden, benutzt. 
Hierzu zählt auch die sogenannte \emph{Feature-Discovery}.
Dabei soll der Benutzer durch ein hinweisendes Overlay dazu aufgefordert werden, gewisse Funktionen der App zu erkunden. 
Dies ist besonders bei einer App, in der es mehr als eine zentrale Funktion gibt, hilfreich und wichtig. \\

\begin{figure}[h]
    \centering
    \includegraphics[keepaspectratio, width=0.5\textwidth]{fd}
    \caption{Feature-Discovery in Form eines Tap-Targets}
    \label{fig:fd}
\end{figure}

In \autoref{fig:fd} sieht man eine beispielhafte Implementierung einer solchen Feature-Discovery in Form eines Tap-Targets.
Hierbei wird der Benutzer durch eine Hervorhebung bestimmter UI-Elemente auf Funktionen hingewiesen. 
Zusätzlich wird ein kurzer erklärender Satz mit einem zusammenfassenden Titel angezeigt. \\

\subsection{Gesture-Support}

Mit der Einführung Applikationen für mobile Endgeräte hat sich ein weiteres, vorher nicht relevantes, Usability-Problem etabliert: 
\emph{Die geringe Größe des Bildschirm}. \\

Die relativ kleine physikalische Größe der Smartphone-Displays bringt verschiedene Probleme mit sich. 
So muss einerseits die Funktionalität einer ganzen Desktop-Applikation auf ein viel kleineres Display passen, ohne den Content zu verändern, oder gar unleserlich zu machen.
Andererseits muss dem Benutzer eine intuitive und effiziente Navigationsmöglichkeit gegeben sein, um zwischen verschiedenen Inhalten zu wechseln. \\

Hierzu schreiben \citeauthor{Gutwin04}, dass die Navigationen auf kleinen Bildschirmen selbst im besten Fall deutlich langsamer sei als auf normal-großen Bildschirmen.
Sie führen weiter an, dass eine Übersicht über den gesamten Systemzustand wertvoll sei, da dies dem Nutzer eine schnellere Navigation erlaube. \\
Nach \citeauthor{Gutwin04} eigne sich die Technik des ``Two-Level Zooms'' besonders gut für Navigationsaufgaben, wohingegen Panning-Strategien eher negativ von den Test-Personen aufgenommen worden seien \citep[Seite 8]{Gutwin04}.  

\todo{Tap-Target prompts}

\subsection{Iterativer Entwicklungsprozess}
  3 Iterationen jeweils Mitte Dezember, Januar und Februar \\
  Feedback in Form von Gitlab-Issues und Beobachten von Testpersonen (Usability-Experimente) \\
  1. Iteration (Dezember): App mit floating buttons (screenshots vorher-nachher)
  2. Iteration (Januar): App mit status bar (screenshots vorher-nachher)
  3. Iteration (Februar): App mit Help-Overlay \& verbesserter Statusbar (screenshots vorher-nachher)
\subsection{DIN und ISO-Normen}
\subsection{Material Design-Guidelines}
\subsection{ABC-Modell}
\subsection{Gesture-Support (papers)}
  Zoom-Area und Panning am besten geeignet um Content auf kleinen Displays darzustellen \\
  
