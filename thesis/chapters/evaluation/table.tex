\begin{sidewaystable}[ht]
  \centering
  \caption{Vergleich der Lösungsalternativen}
  \vspace*{10px}
  \label{tab:nielsen}
  \begin{tabular}{|r|l|c|c|c|}
    \hline
    &								          &       \mm{}   &   \emph{ImageMeter}	  &       \pm{}     \\
    \toprule
    \multirow{10}{*}{\rot{\textbf{Nach \cite{Nielsen94}}}}
    & \hyperref[itm:N1]{1. Sichtbarkeit des Systemzustandes}				&       \po 		&    \po 			&       \po       \\ \cline{2-5} 
    & \hyperref[itm:N2]{2. Übereinstimmung zwischen System und realer Welt} 				&       \po  		&    \po  		&       \po		    \\ \cline{2-5}
    & \hyperref[itm:N3]{3. Benutzerkontrolle- und freiheit} 				&       \po 		&    \xmark	  &       \xmark    \\ \cline{2-5} 
    & \hyperref[itm:N4]{4. Konsistenz und Standards} 				&       \po  		&    \po			&       \po    \\ \cline{2-5}
    & \hyperref[itm:N5]{5. Fehlervorbeugung} 				&       \po  		&    \po			&       \po       \\ \cline{2-5} 
    & \hyperref[itm:N6]{6. Wiedererkennung statt Erinnern} 				&       \po 		&    \xmark		&       \po    \\ \cline{2-5} 
    & \hyperref[itm:N7]{7. Flexibilität und Effizienz der Benutzung} 				&       \xmark  &    \po			&       \po       \\ \cline{2-5} 
    & \hyperref[itm:N8]{8. Ästethisches und minimalistisches Design} \cc    &       \ccnl   &    \ccnl    &       \ccnl     \\ \cline{2-5}
    & \hyperref[itm:N9]{9. Erkennbarkeit, Diagnose und Erholung von Fehlern} 				&       \po   	&    \po 			&       \xmark    \\ \cline{2-5} 
    & \hyperref[itm:N10]{10. Hilfe und Dokumentation} 			&       \po  		&    \xmark		&       \po       \\
    \midrule
    \multirow{8}{*}{\rot{\textbf{Erweitert}}}
    & \hyperref[itm:N11]{11. Adäquater Umgang mit Unterbrechungen} 			&       \po   	&    \po 			&       \po       \\ \cline{2-5} 
    & \hyperref[itm:N12]{12. Fokussieren von Informationen} 			&       \po   	&    \po 			&       \po    \\ \cline{2-5} 
    & \hyperref[itm:N13]{13. ``Joy of Use''}			 	&       \xmark  &    \po  		&       \xmark       \\ \cline{2-5} 
    & \hyperref[itm:N14]{14. ``Don't lie to the user''} \cc   &       \ccnl   &    \ccnl    &       \ccnl     \\ \cline{2-5}
    & \hyperref[itm:N15]{15. Unterstützung verschiedener Bildschirmausrichtungen} 			&       \xmark  &    \po		  &       \po     	\\ \cline{2-5}   
    & \hyperref[itm:N16]{16. Ergonomische Gestaltung der physischen Interaktion} 			&       \xmark  &    \po  		&       \xmark    \\ \cline{2-5} 
    & \hyperref[itm:N17]{17. Einfache Eingabe, Lesbarkeit und Übersichtlichkeit} 			&       \po  		&    \po  		&       \po   		\\ \cline{2-5} 
    & \hyperref[itm:N18]{18. Stelle Privatheit sicher} \cc   &       \ccnl   &    \ccnl    &       \ccnl     \\
    \midrule
    \multirow{6}{*}{\rot{\textbf{Weitere}}}
    &&&& \\
    & \hyperref[itm:integration]{Integration der App in eine vorhandene Systemarchitektur}  &      	\xmark	&    \xmark   &       \xmark    \\
    &&&& \\ \cline{2-5}
    &&&& \\ 
    & \hyperref[itm:export]{Export des annotierten Bildes und der Meta-Daten}    &      	\xmark	&    \xmark		&       \xmark  \\
    &&&& \\ \bottomrule
  \end{tabular} \\
  \vspace*{10px}
  Bei der Bewertung nicht berücksichtigte Kriterien sind in \textcolor{gray}{grau} hinterlegt und mit \nl{} gekennzeichnet. \\
  Gut umgesetzte Kriterien sind mit \po{} und schlecht umgsetzte Kriterien mit \xmark{} markiert.
\end{sidewaystable}
