\documentclass[a4paper]{article}

\usepackage[utf8]{inputenc}
\usepackage{url}
\usepackage{times}
\usepackage[numbers,sort&compress]{natbib}
\usepackage[ngerman]{babel}
\usepackage{graphicx}
\usepackage{fancyhdr}
\usepackage{lastpage}


\setlength{\bibsep}{2pt}
\graphicspath{{resources/}} 

% Macros
\newcommand{\figwidth}{\columnwidth*3/4}


% Header and footer
\pagestyle{fancy}
\fancyhf{}
\rhead{Patrick Stens}
\lhead{Bachelor-Thesis Proposal}
\rfoot{Page \thepage\ of \pageref{LastPage}}


\begin{document}
\renewcommand{\refname}{Literature}

% Title page
\begin{titlepage}
	\centering
	\includegraphics[width=\textwidth]{logo}
	\par\vspace{5cm}

	{\scshape\Large Bachelor-Thesis Proposal\par}
	\vspace{1cm}

	{\huge\bfseries Mobile Bildbearbeitung als Ansatz für die effiziente Aufmaßerfassung im Gerüstbau\par}
	\vspace{2cm}

	{\large VERO Scaffolding EOOD Niederlassung Deutschland\par}

	{\Large Patrick Stens\par}

	\vfill

	Beaufsichtigt von\par
	Prof. Dr.~Gerd \textsc{Szwillus}	

\end{titlepage}

\section{Motivation}
Die Aufmaßerfassung ist ein ganz entscheidender Teil der Gerüstebau-Industrie. 
Hierbei werden ersten Kosten für benötigte Materialen überschlagen, und dem Kunden anschließend ein Kostenvoranschlag gemacht. \\ 
Der bisherige Ablauf der Aufmaßerfassung sieht dabei wie folgt aus:
\begin{enumerate}
	\item Monteur fährt zum Arbeitsauftrag 
	\item Berechnet die ungefähren Maße
	\item Schreibt sich die Maße im besten Falle auf
	\item Fährt zurück ins Büro
	\item Gibt die Daten zur Eintragung ins System an die Sekretärin weiter
\end{enumerate}

Diese Verfahren hat sich im Laufe der Zeit als fehleranfällig und zeitaufwenig herausgestellt.
Durch zu grobe Schätzungen, aber auch die hohe Gedächtnisbelastung der Monteure kommt es bei der Aufmaßerfassung immer wieder zu Kosten, die man vermeiden hätte können. 
Genau hier soll das Smartphone ins Spiel kommen bla bla bleh
\end{document}
