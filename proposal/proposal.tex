\documentclass[a4paper]{article}
\listfiles

\usepackage[utf8]{inputenc}
\usepackage{url}
\usepackage{times}
\usepackage[numbers,sort&compress]{natbib}
\usepackage[ngerman]{babel}
\usepackage{graphicx}
\usepackage{fancyhdr}
\usepackage{lastpage}
\usepackage{lipsum}


\setlength{\bibsep}{2pt}
\graphicspath{{resources/}} 

% Macros
\newcommand{\figwidth}{\columnwidth*3/4}
\newcommand{\Vero}{\textsc{VERO Scaffolding EOOD Niederlassung Deutschland}}
\newcommand{\vero}{\textsc{VERO}}



% Header and footer
\pagestyle{fancy}
\fancyhf{}
\rhead{Patrick Stens}
\lhead{Bachelor-Thesis Proposal}
\rfoot{Seite \thepage\ von \pageref{LastPage}}
\begin{document}
\renewcommand{\refname}{Literatur}

% Title page
\begin{titlepage}
    \centering
    \includegraphics[width=\textwidth]{logo}
    \par\vspace{5cm}

    {\scshape\Large Bachelor-Thesis Proposal\par}
    \vspace{1cm}

    {\huge\bfseries Mobile Bildbearbeitung als Ansatz für die effiziente Aufmaßerfassung im Gerüstbau\par}
    \vspace{2cm}

    {\large \par}

    {\Large Patrick Stens\par}

    \vfill

    Angeboten von\par
    \Vero

    \vfill

    Beaufsichtigt von\par
    Prof. Dr.~Gerd \textsc{Szwillus}	

\end{titlepage}

\section*{Motivation}
Die Aufmaßerfassung ist ein wichtiger Bestandteil als Vorbereitung zur Rechnungslegung im Handwerk, insbesondere im Gerüstbau. \\
Hierbei werden die Maße (Länge, Breite und Höhe) der bereitgestellten Gerüste, idealerweise abgemessen, meistens jedoch abgeschätzt und später im Büro mit einer Preisliste nach Gerüsttyp (Fassadengerüst, Raumgerüst, Hängegerüst, ...) abgerechnet. \\
Der bisherige Ablauf einer solchen Aufmaßerfassung sieht dabei für den Monteur in etwa wie folgt aus:
\begin{enumerate}
    \item Fahrt zum Kunden
    \item Erstellen des Aufmaßes
    \item Aufschreiben der gesammelten Daten auf Papier
    \item Abnahme vom Kunden
    \item Rückfahrt ins Büro
    \item Eintragen der Daten ins System
    \item Rechnungsstellung
\end{enumerate}


Dieses Verfahren hat sich als fehleranfällig und zeitaufwändig herausgestellt.
Durch zu grobe Schätzungen, aber auch die hohe Gedächtnisbelastung der Monteure kommt es bei der Aufmaßerfassung immer wieder zu Fehlern und damit Kosten, die sich durch effizientere Informationsverarbeitung vor Ort auf der Baustellen vermeiden lassen.
Genau diese Problematik soll durch die Entwicklung einer Applikation für Android-Endgeräte gelöst werden. \\

Dabei soll die Software dem Benutzer es ermöglichen, bereits auf der Baustelle Bilder zu machen, diese mit Hilfe von verschiedenen geometrischen Formen zu bearbeiten, und anschließend die gewünschten Informationen (Maße und Gerüstart) direkt zu hinterlegen.
Weiterführend soll die Möglichkeit bestehen, die bearbeiteten Bilder an eine \textit{API} zu senden, welche anhand der eingebenden Maße und Gerüstarten direkt eine Aufmaß zur Freigabe für den Kunden generiert.\\

Für die Umsetzung einer solchen Lösung stellen sich für den Entwickler verschiedene Fragen, die sich sowohl auf die Wechselwirkung zwischen Benutzer und Smartphone oder Tablet, als auch auf die robuste Implementierung einer Bearbeitungsumgebung für die aufgenommen Bilder beziehen: \\

\begin{itemize}
    \item Wie setzt man eine Bearbeitungsumgebung auf dem Smartphone oder Tablet um, die dem Benutzer intuitiv alle möglichen Bearbeitungsoptionen aufzeigt, das Bild dennoch zu jeder Zeit gut erkennbar bleibt?
    \item Auf welche Weise ermöglicht man eine effiziente und zuverlässige Bearbeitungsmethode der aufgenommenen Bilder?
    \item Gibt es spezielle äußerliche Einflüsse, die erfüllt sein müssen, um ein gutes Ergebnis zu garantieren (Belichtung, Aufnahmewinkel)?
    \item Wo liegen die Grenzen der mobilen Aufmaßerfassung?
    \item Wie lassen sich Meta-Informationen zum Bild (z.B. Beschriftung von Linien) für eine API oder nachgelagerte Dienste aufbereiten.
\end{itemize}

\section*{Ziele}
Das Hauptziel des vorgeschlagenen Themas sollte es sein, eine effiziente und robuste Software-Lösung in Form einer Android-Applikation für die Aufmaßerfassung im Gerüstbau zu erstellen. 
Die Applikation sollte dem Benutzer unnötigen Arbeitsaufwand abnehmen, und die Erfassung der notwendigen Daten effizient mittels einer entsprechenden Oberfläche möglich machen.
Hierzu sollte die Benutzeroberfläche intuitiv verständlich, gleichzeitig aber auch so komplex gestaltet werden, dass auf der einen Seite keine lange Einarbeitungszeit von Nöten ist, auf der anderen Seite aber die gewünschte Funktionalität jedoch nicht eingeschränkt wird.

\section*{Umsetzung}
Ein Hauptteil der Bachelor-Arbeit wird es sein, einen ersten Prototypen für die mobile Aufmaßerfassung zu implementieren.
Hierzu können originale Fotos von Baustellen genutzt werden, die von der Firma \vero{} bereitgestellt werden.
Des Weiteren steht die Android-Applikation \textit{BauBuddy}, sowie die dazugehörige \textit{API} zur Verfügung, um den Prototypen bereits in ein bestehendes Software-System einbinden und testen zu können. \\
Diese erste Prototyp-Implementierung sollte es dem Benutzer ermöglichen, ausgewählte Fotos mit Hilfe von verschiedenen geometrischen Formen zu bearbeiten, Informationen zu hinterlegen und abschließend wieder abzuspeichern.
Sobald ein erster Prototyp entwickelt ist, wird sich ein weiterer Teil der Arbeit damit beschäftigen, die Bearbeitungsumgebung so benutzerfreundlich und intuitiv wie möglich zu gestalten.
Zu jeder Zeit ergibt sich die Möglichkeit, den Prototypen bereits im Arbeitsalltag der Monteure der Firma \vero{} auszutesten, um mögliche Schwachstellen frühzeitig erkennen und verbessern zu können.
Zusätzlich bietet sich die Option, bereits einige Monteure mit der Applikation auszustatten, um anschließend auswerten zu können, wie groß der Einfluss der mobilen Lösung auf die Effizienz ist.

\section*{Umriss der Kapitel}
\begin{itemize}
    \item Einleitung 
    \item Problemstellung
    \item Konzeption des Prototyps
    \item Implementierung
    \item Evaluation
        \begin{itemize}
            \item Funktionalität
            \item Usability
        \end{itemize}
    \item Verbesserungsmöglichkeiten
    \item Probleme und Grenzen
    \item A/B-Test
    \item Auswertung des Tests in Hinblick auf Effizienz/Kosten
\end{itemize}

\section*{Zeitplanung}
\begin{itemize}
    \item[\textbf{01.11.2017}] Literatur und Recherche
    \item[\textbf{01.12.2017}] Start der Implementierung
        \begin{itemize}
            \item Bestimme Werkzeuge und Methoden für die mobile Bildbearbeitung
            \item Implementiere ersten Prototyp
            \item Gestalte eine intuitive Benutzeroberfläche 
            \item Teste Prototyp in der Firma
            \item Werte Testergebnisse aus
            \item Verbessere Prototyp mit Hilfe der Ergebnisse
        \end{itemize}
    \item[\textbf{14.02.2018}] Abschließen der Implementierung
        \begin{itemize}
            \item Aufzeigen der Probleme und Grenzen
            \item Durchführung eines A/B-Tests
            \item Auswerten des Tests und Ergebnisse veranschaulichen
        \end{itemize}
    \item[\textbf{31.03.2018}] Fertigstellung der Thesis
\end{itemize}
\end{document}
